%
% slides.tex -- slides
%
% (c) 2017 Prof Dr Andreas Müller, Hochschule Rapperswil
%
\theoremstyle{definition}
\newtheorem{trigopol}{Trigonometrische Polynome}
\newtheorem{fourier}{Fourier-Reihe}
\newtheorem{leastsquares}{Kleinste Quadrate}
\newtheorem{loesung}{Lösung}
\newtheorem{geometrie}{Schwerpunkt}
\newtheorem{produkte}{Produkte}
\newtheorem{basis}{Basiseigenschaft}
\newtheorem{frage}{Frage}
\newtheorem{antwort}{Antwort}
\newtheorem{dft}{Diskrete Fourier-Transformation}

\begin{document}

\setlength{\abovedisplayskip}{3pt}
\setlength{\belowdisplayskip}{3pt}

\ifthenelse{\boolean{presentation}}{
\begin{frame}
\titlepage
\end{frame}
}{}

\ifthenelse{\boolean{presentation}}{
\section{Astronomische Schwankungen}

\begin{frame}
\frametitle{Astronomische Einflüsse}
\begin{enumerate}[<+->]
\item
Strahlungsleistung der Sonne nimmt mit der Zeit zu\uncover<6->{: {\color{blue}
nur wichtig für die Klimarekonstruktion über sehr lange Zeiträume}}%
\uncover<11->{, keine Periode}
\item
Sonnenfleckenzyklus führt zu Schwankungen der Strahlungsleistung%
\uncover<7->{: {\color{blue}kurzfristig beobachtbar}}%
\uncover<12->{, {\color{red}Periode 11 Jahre}}
\item
Störungen durch andere Planeten führen zu periodischen Schwankungen
in den Erdbahn-Parametern, vor allem die Exzentrizität der Erdbahn%
\uncover<8->{: {\color{blue}Höhere Strahlung im Perihel wenn $e$
kleiner}}%
\uncover<13->{, {\color{red}Periode 400\,kyr}}
\item
Anziehung durch den Mond: Erdachse präzediert%
\uncover<9->{: {\color{blue}Mehr Strahlung auf der Nordhalbkugel wenn der
Sonne zugewandt}}%
\uncover<14->{, {\color{red}Periode 23\,kyr}}
\item
Störungen führen zu Schwankungen der Neigung der Erdachse%
\uncover<10->{: {\color{blue}Mehr Strahlung am Pol}}%
\uncover<15->{, {\color{red}Periode 42kyr}}
\end{enumerate}
\uncover<16->{$\Rightarrow$ Welche Einflüsse sind im Klima nachweisbar?}
\end{frame}
}{}

\section{Fourier}

%
% Idee von Fourier
%
\begin{frame}
\frametitle{Die Idee von Fourier}
\begin{trigopol}
Ein trigonometrisches Polynom ist eine Funktion der Form
\begin{align*}
f_n(t)
&=
a_0 + a_1\cos t+b_1\sin t + a_2\cos 2t + b_2\sin 2t + \dots + a_n\cos nt
\\
&=
a_0 + \sum_{k=1}^{n-1} a_k \cos kt + \sum_{k=1}^{n-1}b_k\sin kt
+ a_n\cos nt
\end{align*}
\end{trigopol}
\begin{fourier}
Eine periodische Funktion $f(t)$ kann durch eine Fourier-Reihe beliebig
genau approximiert werden:
\[
f(t)
=
a_0 + \sum_{k=1}^{\infty}  a_k \cos kt + \sum_{k=1}^{\infty}b_k\sin kt
\]
\end{fourier}
\end{frame}

%
% Kleinste Quadrate
%
\begin{frame}
\frametitle{Kleinste Quadrate}
\begin{columns}
\begin{column}{0.47\hsize}
\begin{leastsquares}
Zu vorgegebenen Funktionswerten $y_j=f(t_j)$ in den Punkten
\[
t_j = 2\pi \frac{j}{N},\qquad j=0,\dots,N-1
\]
finde Koeffizienten $\hat a_k$ und $\hat b_k$ derart, dass
\[
L=\sum_{j=0}^{N-1} \bigl(y_j - f(t_j)\bigr)^2
\]
minimal wird, wobei $f(t)$ ein trigonometrisches Polynom ist.
\end{leastsquares}
%\begin{dft}
$\hat{a}_k$ und $\hat{b}_k$ heissen {\em diskrete Fourier-Transformation}
%\end{dft}
\end{column}
\begin{column}{0.5\hsize}
\begin{loesung}
$a_k$ und $b_k$ so bestimmen, dass
\begin{align*}
\frac{\partial L}{\partial a_k}&=0&
\frac{\partial L}{\partial b_k}&=0
\end{align*}
gilt.
Berechnung der Ableitungen:
\begin{align*}
\frac{\partial L}{\partial a_k}
&=
\sum_{j=0}^{N-1}2\bigl(y_j-f(t_j)\bigr)
\frac{\partial f(t_j)}{\partial a_k}
\\
\frac{\partial f(t_j)}{\partial a_k}
&=
\cos kt_j
,\;
\frac{\partial f(t_j)}{\partial b_k}
=
\sin kt_j
\end{align*}
und analog für $\partial L/\partial b_k$
\begin{align*}
\frac{\partial L}{\partial b_k}
&=
\sum_{j=0}^{N-1}2\bigl(y_j-f(t_j)\bigr)
\frac{\partial f(t_j)}{\partial b_k}
\end{align*}
\end{loesung}
\end{column}
\end{columns}
\end{frame}

%
% Trigonometrische Summen
%
\begin{frame}
\frametitle{Spezielle trigonometrische Summen}
\begin{columns}
\begin{column}{0.48\hsize}
\begin{geometrie}
Die Punkte $(\cos kt_j,\sin kt_j)$ sind gleichmässig auf einem Kreis verteilt:
\\[3pt]
Schwerpunkt der roten Punkte ist $0$,
ausser für $k=0$
\end{geometrie}
\end{column}
\begin{column}{0.48\hsize}
\def\r{1}
\begin{tikzpicture}[>=latex,thick]
\draw (0,0) circle[radius={\r}];
\draw[->] (-1.3,0)--(1.3,0);
\draw[->] (0,-1.3)--(0,1.3);
\foreach\i in {0,1,...,16}{
\fill[color=red] ({\r*cos(360*\i/17)},{\r*sin(360*\i/17)}) circle[radius=1.5pt];
}
\node at (-0.7,-1.2) {$N=17$};
\node at (1.5,0) [right] {\begin{minipage}{3cm}\raggedright
\begin{align*}
\sum_{j=0}^{N-1} \cos kt_j&=N\delta_{k0}\\
\sum_{j=0}^{N-1} \sin kt_j&=0
\end{align*}
\end{minipage}};
\end{tikzpicture}
\end{column}
\end{columns}
\begin{produkte}
\vspace{-20pt}
\begin{tikzpicture}[>=latex,thick]
\node at (0,0) {
\begin{minipage}{\hsize}
\[
\left.
\begin{aligned}
\cos\alpha\cos\beta&={\textstyle\frac12}(\cos(\alpha-\beta)+\cos(\alpha+\beta))
\\
\sin\alpha\sin\beta&={\textstyle\frac12}(\cos(\alpha-\beta)-\cos(\alpha+\beta))
\\
\sin\alpha\cos\alpha&={\textstyle\frac12}(\sin(\alpha-\beta)+\sin(\alpha+\beta))
\end{aligned}
\right\}
\Rightarrow
\left\{
\begin{aligned}
\sum_{j=0}^{N-1}\cos kt_j\cos lt_j&=\frac{N}2\delta_{kl}
\\
\sum_{j=0}^{N-1}\sin kt_j\sin lt_j&=\frac{N}2\delta_{kl}
\\
\sum_{j=0}^{N-1}\sin kt_j\cos lt_j&=0
\end{aligned}
\right.
\]
\end{minipage}
};
\node at (-5,-2) {Im Folgenden: $N=2n$ gerade};
\end{tikzpicture}
\end{produkte}
\end{frame}

%
% Lösung a_0
%
\begin{frame}
\frametitle{Lösung $a_0$ und $a_n$}
\begin{align*}
\frac{\partial L}{\partial a_0}
&=
\sum_{j=0}^{N-1} 2\biggl(y_j-a_0
-\sum_{k=1}^{n-1}
a_k\cos kt_j
-\sum_{k=1}^{n-1}
b_k\sin kt_j\bigr)
-
a_n\cos nt_j
\biggr)\frac12=0
\\
0
&=
\sum_{j=0}^{N-1} y_j - Na_0
-\sum_{k=1}^{n-1}
a_k\underbrace{\sum_{j=0}^{N-1}\cos kt_j}_{\displaystyle=0}
-\sum_{k=1}^{n-1}
b_k\underbrace{\sum_{j=0}^{N-1}\sin kt_j}_{\displaystyle=0}
-
a_n\underbrace{\sum_{j=0}^{N-1}\cos nt_j}_{\displaystyle=0}
\\
\Rightarrow\;
\hat a_0&=\frac{1}{N}\sum_{j=0}^{N-1} y_j,
\qquad
\text{analog findet man}
\quad
\hat a_n
=
\frac{1}{N}\sum_{j=0}^{N-1} y_j \cos nt_j
=
\frac{1}{N}\sum_{j=0}^{N-1} (-1)^jy_j
\end{align*}
\end{frame}

%
% Lösung a_k
%
\begin{frame}
\frametitle{Lösung $a_k$, $0<k<n$}
\begin{align*}
\frac{\partial L}{\partial a_k}
&=
\sum_{j=0}^{N-1} 2\biggl(y_j-a_0-\sum_{l=1}^{n-1}
\bigl(a_l\cos lt_j + b_l\sin lt_j\bigr)
-a_n\cos nt_j
\biggr)\cos kt_j=0
\\
\sum_{j=0}^{N-1}y_j\cos kt_j
&=
a_0\underbrace{\sum_{j=0}^{N-1}\cos kt_j}_{\displaystyle=0}
+
a_n
\underbrace{
\sum_{j=1}^{N-1}\cos nt_j
}_{\displaystyle=0}
\\
&\qquad
+
\sum_{l=1}^{n-1}
a_l
\underbrace{
\sum_{j=1}^{N-1}
\cos lt_j
\cos kt_j
}_{\displaystyle=\frac{N}{2}\delta_{kl}}
+
\sum_{j=1}^{N-1}
b_l
\underbrace{
\sum_{j=1}^{N-1}
\sin lt_j
\cos kt_j
}_{\displaystyle=0}
\\[-5pt]
\Rightarrow\;
\hat a_k&=\frac{2}{N}\sum_{j=0}^{N-1}y_j\cos kt_j
\end{align*}
\end{frame}

%
% Lösung b_k
%
\begin{frame}
\frametitle{Lösung $b_k$, $0<k<n$}
\begin{align*}
\frac{\partial L}{\partial b_k}
&=
\sum_{j=0}^{N-1} 2\biggl(y_j-a_0-\sum_{l=1}^{n-1}
\bigl(a_l\cos lt_j + b_l\sin lt_j\bigr)
-a_n\cos nt_j
\biggr)\sin kt_j=0
\\
\sum_{j=0}^{N-1}y_j\cos kt_j
&=
a_0\underbrace{\sum_{j=0}^{N-1}\sin kt_j}_{\displaystyle=0}
+
a_n\underbrace{\sum_{j=0}^{N-1}\sin kt_j\cos nt_j}_{\displaystyle=0}
\\
&\qquad
+
\sum_{l=1}^{n-1}
a_l
\underbrace{
\sum_{j=1}^{N-1}
\cos lt_j
\sin kt_j
}_{\displaystyle=0}
+
\sum_{l=1}^{n-1}
b_l
\underbrace{
\sum_{j=1}^{N-1}
\sin lt_j
\sin kt_j
}_{\displaystyle=\frac{N}{2}\delta_{kl}}
\\[-5pt]
\Rightarrow\;
\hat b_k&=\frac{2}{N}\sum_{j=0}^{N-1}y_j\sin kt_j
\end{align*}
\end{frame}

%
% Schreibweise als Skalarprodukt
%
\begin{frame}
\frametitle{Skalarprodukt für $\hat a_k$ und $\hat b_k$}
\begin{align*}
\hat a_k
&=
\sum_{j=0}^{N-1} y_j \cos kt_j
&&\Rightarrow&
\hat a_k
&=
\frac{2}{N}
\underbrace{
\begin{pmatrix}y_0\\y_1\\\vdots\\y_{N-1}\end{pmatrix}
}_{\displaystyle\vec{y}}
\cdot
\underbrace{
\begin{pmatrix}\cos kt_0\\\cos kt_1\\\vdots\\\cos kt_{N-1}\end{pmatrix}
}_{\displaystyle \vec{c}_k}
\\
\intertext{analog:}
\hat b_k
&=
\sum_{j=0}^{N-1} y_j \sin kt_j
&&\Rightarrow&
\hat b_k
&=
\frac{2}{N}
\begin{pmatrix}y_0\\y_1\\\vdots\\y_{N-1}\end{pmatrix}
\cdot
\underbrace{
\begin{pmatrix}\sin kt_0\\\sin kt_1\\\vdots\\\sin kt_{N-1}\end{pmatrix}
}_{\displaystyle \vec{s}_k}
\\
\end{align*}
\end{frame}

\begin{frame}
\frametitle{Skalarprodukt für $\hat a_0$ und $\hat a_n$}
\begin{align*}
\vec{c}_0
&=
\begin{pmatrix}
1\\1\\\vdots\\1
\end{pmatrix}
=\vec{1}
&&\Rightarrow&
\hat a_0
&=
\frac1{N}
\sum_{j=0}^{N-1} y_j\cdot 1
=
\frac{1}{N}
\begin{pmatrix}y_0\\y_1\\\vdots\\y_{N-1}\end{pmatrix}
\cdot
\begin{pmatrix}1\\1\\\vdots\\1\end{pmatrix}
\frac{1}{N}
=
\vec{y}\cdot\vec{c}_0
=
\vec{y}\cdot\vec{1}
\\
\vec{c}_n
&=
\begin{pmatrix}1\\-1\\\vdots\\\pm1\end{pmatrix}
&&\Rightarrow&
\hat a_n
&=
\frac{1}{N}\sum_{j=1}^{N-1}y_j\cdot \cos nt_j
=
\frac{1}{N}\vec{y}\cdot\vec{c}_n
\end{align*}
\end{frame}

\begin{frame}
\frametitle{Orthogonalität}
Die Vektoren $\vec{c}_k$ und $\vec{s}_k$ sind orthogonal.
Für $0<k,l<n$ gilt:
\begin{equation}
\begin{aligned}
\vec{s}_k\cdot \vec{s}_l
&=
\sum_{j=0}^{N-1} \sin kt_j \sin lt_j = \frac{N}{2}\delta_{kl},
&
\vec{c}_k\cdot \vec{c}_l
&=
\sum_{j=0}^{N-1} \cos kt_j \cos lt_j = \frac{N}{2}\delta_{kl}
\\
\vec{s}_k\cdot \vec{c}_l
&=
\sum_{j=0}^{N-1} \sin kt_j \cos lt_j = 0
\end{aligned}
\label{ortho1}
\end{equation}
Die Regeln für $\vec{c}_0$ und $\vec{c}_n$ sind etwas speziell:
\begin{equation}
\begin{aligned}
\vec{c}_0\cdot \vec{c}_k&=N\delta_{k0},
&
\vec{c}_n\cdot \vec{c}_k&=N\delta_{kn}
\\
\vec{c}_0\cdot\vec{s}_k&=0
&
\vec{c}_n\cdot\vec{s}_k&=0
\end{aligned}
\label{ortho2}
\end{equation}
\begin{basis}
Mit $N=2n$ bilden die $\vec{c}_k$ mit $0\le k\le n$ und die
$\vec{s}_l$ mit $1\le l\le n-1$ eine Basis.
Die Koeffizienten $\hat a_k$ und $\hat b_l$ sind daher
eindeutig bestimmt.
\end{basis}
\end{frame}

%
% Approximationsfehler
%
\begin{frame}
\frametitle{Approximationsfehler}
\begin{frage}
Für $N=2n$ ist $y_j=f_n(t_j)$ für alle Stützstellen $j=0,1,\dots N-1$.
Um wieviel vergrössert sich der Fehler, wenn man einige Terme weglässt?
\end{frage}

Weggelassene Terme:
\[
\vec{e}(t)
=
\sum_{k\in K}a_k\vec{c}_k
+
\sum_{l\in L}b_l\vec{s}_l
\]

\begin{antwort}
Länge von $\vec{e}(t)$ mit Orthonormalität ausrechnen:
\begin{equation}
|\vec{e}(t)|^2
=
\sum_{k\in K} a_k^2\, \vec{c}_k\cdot\vec{c}_k
+
\sum_{l\in L} b_l^2\, \vec{s}_l\cdot\vec{s}_l
\label{fehler}
\end{equation}
Die Skalarprodukte sind durch
\eqref{ortho1}
und
\eqref{ortho2}
bestimmt.
\end{antwort}
\end{frame}

%
% Zusammenfassung
%
\begin{frame}
\frametitle{Fourier-Zusammenfassung}
$N=2n$ Datenpunkte $\vec{y}$ lassen sich exakt durch das
trigonometrische Polynom
\[
f_n(t)
=
a_0 + a_1\cos t+b_1\sin t + a_2\cos 2t + b_2\sin 2t + \dots + a_n\cos nt
\]
wiedergeben, wenn man
\begin{align*}
a_0
&=
\frac1N\vec{y}\cdot\vec{c}_0
=
\frac1N\sum_{j=0}^{N-1} y_j
&
a_k
&=
\frac2N\vec{y}\cdot\vec{c}_k
=
\frac2N\sum_{j=0}^{N-1} y_j\cos kt_j
\\
a_n
&=
\frac1N\vec{y}\cdot\vec{c}_n
=
\frac1N\sum_{j=0}^{N-1} y_j\cos nt_j
&
b_k
&=
\frac2N\vec{y}\cdot\vec{s}_k
=
\frac2N\sum_{j=0}^{N-1} y_j\sin kt_j
\end{align*}
mit $0<k<n$
wählt.
\bigskip

Der Fehler beim Weglassen von Termen wird durch \eqref{fehler}
bestimmt.

\end{frame}

\section{Anwendung}
%
% 
%
\begin{frame}
\frametitle{Anwendung}
\begin{enumerate}[<+->]
\item
Periodische Phänomene mit Hilfe von Fourier-Theorie verwenden
\item
Anomalie $\mathstrut = \mathstrut$ Abweichung von den modellierten
periodischen Phänomenen.
\item
Zeiträume grösser als die Perioden:
\begin{align*}
\int_0^{2\pi}
f(t)\,dt
&=
a_0\int_0^{2\pi} \,dt
+
\sum_{k=1}^\infty a_k\underbrace{\int_0^{2\pi}\cos kt\,dt}_{\displaystyle=0}
+
\sum_{k=1}^\infty b_k\underbrace{\int_0^{2\pi}\sin kt\,dt}_{\displaystyle=0}
\\
&=
2\pi a_0
\end{align*}
D.~h.~$2\pi a_0$ sind langfristige Mittelwerte
$\rightarrow$ Fourier verfeinert Gleichgewichtsmodelle
\end{enumerate}
\end{frame}

\end{document}
