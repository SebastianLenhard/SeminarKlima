%
% themen.tex -- XXX
%
% (c) 2017 Prof Dr Andreas Müller, Hochschule Rapperswil
%
\theoremstyle{definition}
\newtheorem{aufgabe}{Aufgabe}

\begin{document}

\begin{frame}
\frametitle{Statistik extremer Ereignisse}

\begin{aufgabe}
Erklären Sie wie man aus der Statistik extremer Ereignisse schliessen
kann, dass der Klimawandel real ist, und illustrieren Sie dies wenn
möglich mit Daten zum Beispiel von Erdrutschen oder Hochwassern aus
der Schweiz.
\end{aufgabe}

\begin{itemize}
\item Buch Kapitel 19
\item Gedacht als ``Bauingenieurthema''
\end{itemize}

\end{frame}

\begin{frame}
\frametitle{3-Box-Modell der thermohalinen Zirkulation}

In der Vorlesung wurde das 2-Box-Modell der thermohalinen Zirkulation
zwischen Nordatlantik und Äquator beschrieben.
Insbesondere wurde ein Kipp-Phänomen gefunden.

\begin{aufgabe}
Entwickeln sie das 3-Box-Modell der thermohalinen Zirkulation, welches
Nord- und Südatlantik als separate Boxen modelliert, und versuchen das
Verhalten des Modells bei ansteigender Temperatur vorherzusagen.
\end{aufgabe}

\begin{itemize}
\item Buch, Aufgaben zu Kapitel 6
\end{itemize}

\end{frame}

\begin{frame}
\frametitle{0-dimensionales Modell}

\begin{aufgabe}
Konstruieren sie ein vertikales Modell der Atmosphäre und versuchen
Sie das Verhalten des Modells bei Parameteränderung zu bestimmen.
\end{aufgabe}
\end{frame}

\begin{frame}
\frametitle{Klimamodelle auf anderen Planeten}
\begin{aufgabe}
Finden Sie ein Modell für einen Planeten, in welchem die Strahlungsbilanz
nicht wie auf der Erde durch Wolkenbildung stark beeinflusst wird, sondern
durch Staub in der Atmosphäre.
\end{aufgabe}
\begin{itemize}
\item
Anton Petrov's Youtube Channel ``What the Math'' und das
Game ``Universe Sandbox''
\end{itemize}
\end{frame}

\begin{frame}
\frametitle{Eis}
\begin{aufgabe}
Konstruieren Sie ein Modell der Eisbedeckung der Erde mit dem
Rückkopplungseffekt über die planetare Albedo.
\end{aufgabe}
\end{frame}

\begin{frame}
\frametitle{Vegetation}
\begin{aufgabe}
Finden Sie ein Modell für die sich änderne Albedo und Feuchtigkeit 
mit sich ändernder Vegetation.
\end{aufgabe}
\end{frame}

\begin{frame}
\frametitle{Methan und Permafrost}
Einige der Auslöschungsereignisse in der Erdgeschichte gehen einher
mit der Freisetzung des sehr potenten Treibhausgases Methan beim Auftauen
des Permafrost.
Dieser Effekt könnte die Klimaerwärmumng beschleunigen und eventuell
sogar unaufhaltsam machen.

\begin{aufgabe}
Finden Sie ein Modell für die Methanfreisetzung aus dem Permafrost und 
untersuchen sie, ob dieser Prozess umkehrbar ist, und in welchem Zeitrahmen.
\end{aufgabe}
\end{frame}

\begin{frame}
\frametitle{Neigung der Erdachse}
Die Milankovic-Zyklen der Eiszeiten zeigen, dass die Neigung der Erdachse
einen wesentlichen Einfluss auf das Klima der Erde hat.
Paradox daran ist jedoch, dass die von der Erde empfangene Gesamtstrahlung
nicht ändert.

\begin{aufgabe}
Untersuchen sie die Koppelung der Neigung der Erdachse an die globale
Temperatur und die Entstehung der Eiszeiten.
\end{aufgabe}
\end{frame}

\begin{frame}
\frametitle{Verzögerte Differentialgleichung}
Matlab stellt keine Funktionen zur Lösung von verzögerten
Differentialgleichungen vor, die zur Modellierung der
El Niño Southern Oscillation (ENSO) verwendet werden.
\begin{aufgabe}
Lösen Sie die verzögerte ENSO-Differentialgleichung numerisch und
untersuchen Sie ihr Verhalten bei Parameter-Änderungen experimentell.
\end{aufgabe}
\end{frame}

\begin{frame}
\frametitle{Klimamodelle ausprobieren}

Die meisten Klimamodelle sind nicht öffentlich im Source-Code verfügbar,
sondern nur auf Anfrage.
Anders das Modell des Institut Pierre Simon Laplace.

\url{http://forge.ipsl.jussieu.fr/igcmg}

\begin{aufgabe}
Bringen Sie das Modell zum Laufen und berichten Sie über die 
Parametrisierungsmöglichkeiten und
Schwierigkeiten.
\end{aufgabe}
\end{frame}

\begin{frame}
\frametitle{Lorenz-Attraktor}

Die Lorenz-Gleichung beschreibt ein stark vereinfachtes Modell der
globalen Zirkulation und stellt sich als chaotisch heraus.

\begin{aufgabe}
Visualisieren Sie die Bahn des Lorenzsystems im Zustandsraum
mit hoher Auflösung und versuchen Sie, Selbstähnlichkeiten zu finden.
\end{aufgabe}

\end{frame}

\begin{frame}
\frametitle{Tesla Roadster}

Jedes Klimamodell muss das Datenassimilationsproblem lösen, mit dem
aus Beobachtungsdaten die Zustandsvariablen des Modells bestimmt werden.
Dieses Problem ist jedoch nicht ein spezifisches Problem von Klimamodellen.

\begin{aufgabe}
Lösen Sie das Datenassimilationsproblem für die Bahn des Tesla Roadster von
Elon Musk.
\end{aufgabe}

Vereinfachende Annahmen:
\begin{itemize}
\item Die Erde bewegt sich auf einer Kreisbahn
\item Die Bahn des Tesla liegt in der Bahnebene der Erde
\end{itemize}

\end{frame}

\end{document}
