%
% main.tex -- Paper zum Thema Lorenz-Attraktor
%
% (c) 2018 Matthias Baumann und Oliver Dias, Hochschule Rapperswil
%

% Settings

\newenvironment{centerFigure}[1][]{
	\begin{figure}[h]
	\begin{center}
	\def\tempa{#1}
}{
	% Tests whether tempa set and add caption, if unset use content in {} brackets \detokenize{}
	\if\tempa\empty\else\caption{\tempa}\let\tempa\undefined\fi  % remove tempa again
	\end{center}
	\end{figure}
}

\lstset{language=Java} 

% Document

\chapter{Lorenz-Attraktor\label{chapter:lorenz}}
\lhead{Lorenz-Attraktor}
\begin{refsection}
\chapterauthor{Matthias Baumann und Oliver Dias}

\section{Abschnitt}
\rhead{Abschnitt} %TODO Remove?

\subfile{lorenz/intro}
\subfile{lorenz/attraktor}
\subfile{lorenz/chaostheorie}
\subfile{lorenz/implementation}
\subfile{lorenz/loesungen}

\section{Schlussfolgerung}
\rhead{Schlussfolgerung}
% \subfile{lorenz/outro}

\printbibliography[heading=subbibliography]
\end{refsection}
