% !TeX root = documentation.tex
% !TeX spellcheck = de_DE
\section{Schlussfolgerung}\label{outro}
Wie wir beschrieben haben, spielt das Chaos eine grosse Rolle in der Wetterprognose und ist der Grund, wieso keine langfristige, verlässliche Prognose gemacht werden kann. Denn selbst kleinste Verfälschungen der Messdaten führen nach einer gewissen Zeit zu einem komplett anderen Resultat. Es kann sein, dass mit einem minimal anderen Input die Resultatmenge gerade auf die andere Seite des Butterflys ausschlagen könnte. Da wir nicht beliebig genau messen können, stellt uns das vor diese Einschränkung der zeitlich limitierten Prognose.

Um Rückschluss auf das Paper von Lorenz mit dem Flügelschlag des Schmetterlings zu nehmen, müssen wir uns überlegen, inwiefern ein Schmetterlingsschlag eine Auswirkung haben kann. Gemäss dem Lorenz-Modell wäre es möglich, dass ein solch kleines Event zu so grossen Auswirkungen wie ein Tornado führen kann, da es sich ja genau um eine winzige Parameteränderung handelt. Hingegen ein Rückschluss zu ziehen, was genau ein Tornado ausgelöst hat, ist praktisch unmöglich, selbst wenn alle benötigten Daten vorhanden wären. Genauso könnte ein Schmetterlingsschlag einen allfälligen Tornado verhindert haben. Zu dieser Erkenntnis ist Herr Lorenz gekommen nachdem er den hier besprochenen Artikel veröffentlicht hat. Er hat diese Schlussfolgerung in einem separaten Folgeartikel\cite{lorenz63} niedergeschrieben. Dieser Grundsatz ist noch heute das entscheidende Merkmal der Chaostheorie.

Doch kann das Modell auch auf die Realität angewendet werden und stimmt nun diese Schlussfolgerung? Für das Berechnen des Lorenz-Attraktors wurden viele Vereinfachungen gemacht. So werden viele relevante Eigenschaften des Wetters wie zum Beispiel die Luftfeuchtigkeit, die Einflüsse der Wolken oder auch die Albedo nicht miteinbezogen. 

Es ist also kein realistisches Modell. Wir kommen zum Schluss, dass das Wetter zwar ein chaotisches Modell ist, aber durch das, dass es ein solch grosses und komplexes System ist, ein Schmetterlingsschlag schlichtweg zu irrelevant ist, um ein Tornado auszulösen. 
