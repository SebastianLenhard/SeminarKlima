%
% rossby.tex
%
% (c) 2018 Prof Dr Andreas Müller, Hochschule Rapperswil
%

\section{Rossby-Wellen\label{section:elnino:rossby}}
In der Untersuchung der Kelvin-Wellen haben wir angenommen, dass die
Geschwindigkeit $v$ entlang der Längenkreise verschwindet.
Die Strömung wird im allgemeinen nicht parallel zum Äquator sein.
Trotzdem beobachten wir zum Beispiel, dass die Westwindströmung 
zwischen verschiedenen geographischen Breiten mäandriert.
Woher kommt diese Wellenbewegung?

\subsection{Zirkulation\label{subsection:rossby:zirkulation}}
Aus der Diskussion der globalen Zirkulation wissen wir, dass die
Strömung in Äquatornähe dominiert wird durch eine mittlere Strömung
konstante Strömung mit Geschwindigkeit $U$ in Ost-West-Richtung.
Wir suchen daher eine Beschreibung der Abweichungen von dieser
mittleren Strömung.
Unter Verwendung der gleichen Notation wie im
Abschnitt~\ref{section:elnino:kelvin} schreiben wir daher
\[
u=U+u',\qquad v=U+v'\qquad\text{mit $u,v\ll U$}.
\]


Wir nehmen im folgenden wieder an, dass die Strömung quellenfrei ist.
Dann lässt sich wie früher gezeigt, dass die Strömung mit einer
Strömungsfunktion $\psi$ als
\[
u=-\frac{\partial \psi}{\partial y},\qquad
v=\frac{\partial\psi}{\partial x}
\]
beschrieben werden kann.

Die Drehimpulsdichte in der Strömung ist gegeben durch die Zirkulation
\[
\zeta
=
\frac{\partial v}{\partial x} - \frac{\partial u}{\partial y}
=
\Delta \psi.
\]
Da der Drehimpuls erhalten ist, muss die Zirkulation in einem
Luftpacket abnehmen, wenn es sich auf der Nordhalbkubel nach Norden
bewegt.
Die Quelle dieser Zirkulationsänderung ist die Erddrehung und damit
die Corioliskraft $f$ und der mathematische Ausdruck der Drehimpulserhaltung
ist die Erhaltung der Grösse $\zeta+f$.

\subsection{Bewegungsgleichung\label{subsection:rossby:bewegungsgleichung}}
Die Grösse $\zeta+f$ ist erhalten, ihre zeitliche Ableitung
\[
\frac{d(\zeta f)}{dt}=0
\]
verschwindet also.
$f$ hängt nur von $y$ ab, aber die Funktion $\zeta$ hängt von $t$,
$x$ und $y$ ab.
Wir berechnen die Ableitung mit Hilfe der Kettenregel:
\begin{align*}
0
=
\frac{d(\zeta+f)}{dt}
&=
\frac{\partial\zeta}{\partial t}
+
\frac{\partial\zeta}{\partial x}\cdot \frac{dx}{dt}
+
\frac{\partial(\zeta+f)}{\partial y}\cdot\frac{dy}{dt}
\\
&\simeq
\frac{\partial\zeta}{\partial t}
+
(U+u)\frac{\partial\zeta}{\partial x}
+
v\biggl(\frac{\partial\zeta}{\partial y} + \frac{\partial f}{\partial y}\biggr).
\end{align*}
Da $u\ll U$ ist, kännen wir in erster Näherung $u$ im zweiten Term 
vernachlässigen.
Und da $\partial\zeta/\partial y$ ebenfalls sehr klein ist, können
wir dies im letzten Term im Vergleich zu $\partial f/\partial y$
vernachlässigen.
Schliesslich können wir wie in Abschnitt~\ref{section:elnino:kelvin}
die $\beta$-Ebenen-Approximation verwenden, und $\partial f/\partial y$
durch $\beta$ ersetzen.
Schliesslich können wir wieder $v=\partial\psi/\partial x$ ersetzen.
Wir erhalten so
\[
0
=
\frac{\partial\zeta}{\partial t}
+
U\frac{\partial\zeta}{\partial x}
+
\beta\frac{\partial\psi}{\partial x}
\]
als Bewegungsgleichung.

Indem wir $\zeta=\Delta \psi$ schreiben, erhalten wir so die 
Bewegungsgleichung
\begin{equation}
\frac{\partial\Delta\psi}{\partial t}
+
U\frac{\partial\Delta\psi}{\partial x}
+
\beta\frac{\partial\psi}{\partial x}
=
0.
\label{rossby:gleichung}
\end{equation}


\subsection{Wellenlösungen\label{subsection:rossby:loesungen}}
Gesucht sind Lösungen der Gleichung~\eqref{rossby:gleichung}
in Form kleiner Abweichungen.
Wir erwarten Wellenlösungen und schreiben sie daher in der Form
\[
\psi_{kl}(t,x,y)
=
\sin(kx+ly-\omega t).
\]
Die Ableitungen, die wir für die Bewegungsgleichung
\eqref{rossby:gleichung} benötigen, sind
\begin{align*}
\frac{\partial}{\partial t} \psi_{kl}(t,x,y)
&=
-\omega \cos(kx+ly-\omega t) = -\omega\psi_{kl}(t,x,y),
\\
\frac{\partial}{\partial x} \psi_{kl}(t,x,y)
&=
k
\cos(kx+ly-\omega t) = k \psi_{kl}(t,x,y),
\\
\Delta\psi_{kl}
&=
-(k^2+l^2)\cos(kx+ly-\omega t)=-(k^2+l^2)\psi_{kl}(t,x,y).
\end{align*}
Setzen wir dies in die Differentialgleichung ein, erhalten wir 
\begin{align*}
0
&=
\omega(k^2+l^2)\psi_{kl}(t,x,y)
-
Uk(k^2+l^2)\psi_{kl}(t,x,y)
+
\beta k\psi_{kl}(t,x,y)
\\
&=
\bigl((\omega-Uk)(k^2+l^2)+\beta k\bigr)\psi_{kl}(t,x,y)
\end{align*}
Für eine Lösung muss also die Dispersionsrelation
\[
(\omega -Uk)(k^2+l^2) +\beta k=0
\]
gelten.
Aufgeläst nach $\omega$ ist dies
\begin{align*}
\omega(k^2+l^2)
&=
Uk(k^2+l^2) -\beta k
\\
\Rightarrow\qquad
\omega
&=
Uk
-
\frac{\beta k}{k^2+l^2}.
\end{align*}
Eine Wellenlösung mit Wellenzahlen $k$ und $l$ hat daher 
die Ausbreitungsgeschwindigkeit 
\begin{equation}
c=\frac{\omega}{k} = U-\beta\frac{1}{k^2+l^2}
\label{rossby:phasengeschwindigkeit}
\end{equation}
entlang der $x$-Koordinate.
Da der Nenner immer positiv ist, ist $c$ kleiner als $U$, solche
Wellen können sich immer nur in West-Ost-Richtung ausbreiten.

\subsection{Gruppengeschwindigkeit}
Die Phasengeschwindigkeit 
\eqref{rossby:phasengeschwindigkeit}
hilft uns nicht, die Dynamik des El Niño zu verstehen, da wir dazu
den Transport von Energie verstehen müssen.








