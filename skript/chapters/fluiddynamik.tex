%
% fluiddynamik.tex
%
% (c) 2018 Prof Dr Andreas Müller, Hochschule Rapperswil
%
\chapter{Fluiddynamik\label{chapter:fluiddynamik}}
\lhead{Fluiddynamik}
\rhead{}
Die Atmosphäre und die Ozeane unterschieden sich in ihren
für das Studium von Wetter und Klima wesentlichen Eigenschaften
ganz beträchtlich.
Das Wasser der Ozeane ist fast inkompressibel, seine Dichte hängt aber
von der Temperatur und dem Salzgehalt ab.
Wasser hat eine sehr grosse Wärmekapazität, ausserdem kann Wärme durch
Verdunstung aus den Ozeanen in die Atmosphäre übergehen, wobei gleichzeigit
die Salzkonzentration steigt.

Die Atmosphäre auf der anderen Seite hat eine wesentlich geringere
Dichte und Wärmekapazität, ihre Temperatur kann sich daher sehr viel
schneller ändern.
Sie ist stark kompressibel.
Wegen der geringeren Dichte kann die Atmosphäre sehr viel höhere
Strömungsgeschwindigkeiten erreichen.

Trotz dieser grossen Unterschiede lassen sich Atmosphäre und Ozeane
beide als Fluide mit den gleichen partiellen Differentialgleichungen
beschreiben, die im folgenden hergeleitet werden sollen.
Die Unterschiede äussern sich vor allem in den Zustandsgleichungen,
die die Zustandsgrössen Druck, Temepratur, Dichte und Saltzgehalt
miteinander in Beziehung setzen.
Im ersten Abschnitt dieses Kapitels sollen die Grundgleichungen
der Fluiddynamik zusammengestellt werden.
Im zweiten Teil wird am Beispiel des Lorenz-Systems gezeigt, dass
die Gleichungen der Fluiddynamik trotzdem nur beschränkt eine exakte
Prognose des Wetters gestatten können.

%
% hydrodynamik.texo
%
% (c) 2018 Prof Dr Andreas Müller, Hochschule Rapperswil
%
\section{Fluiddynamik}
\rhead{Fluiddynamik}
%Die Atmosphäre und die Ozeane unterschieden sich in ihren
%für das Studium von Wetter und Klima wesentlichen Eigenschaften
%ganz beträchtlich.
%Das Wasser der Ozeane ist fast inkompressibel, seine Dichte hängt aber
%von der Temperatur und dem Salzgehalt ab.
%Wasser hat eine sehr grosse Wärmekapazität, ausserdem kann Wärme durch
%Verdunstung aus den Ozeanen in die Atmosphäre übergehen, wobei gleichzeigit
%die Salzkonzentration steigt.
%
%Die Atmosphäre auf der anderen Seite hat eine wesentlich geringere
%Dichte und Wärmekapazität, ihre Temperatur kann sich daher sehr viel
%schneller ändern.
%Sie ist stark kompressibel.
%Wegen der geringeren Dichte kann die Atmosphäre sehr viel höhere
%Strömungsgeschwindigkeiten erreichen.
%
%Trotz dieser grossen Unterschiede lassen sich Atmosphäre und Ozeane
%beide als Fluide mit den gleichen partiellen Differentialgleichungen
%beschreiben, die im folgenden hergeleitet werden sollen.
%Die Unterschiede äussern sich vor allem in den Zustandsgleichungen,
%die die Zustandsgrössen Druck, Temepratur, Dichte und Saltzgehalt
%miteinander in Beziehung setzen.
%
In diesem Abschnitt gehen wir davon aus, dass das Fluid beschrieben wird
durch Funktionen der Raumkoordinaten $(x,y,z)$ und der Zeit $t$,
wobei wir meistens darauf verzichten, die unabhängigen Variablen
auszuschreiben.
Die Temperatur $T$ ist also zu lesen als die Funktion $T(x,y,z,t)$.
Die Newtonschen Bewegungsgleichungen stellen eine Verbindung zwischen
Masse, Beschleunigung und Kraft her, wir können daher davon ausgehen,
dass die Bewegungsgleichungen eines Fluides nur die Dichte $\varrho$ und 
den Geschwindigkeitsvektor $\vec{v}$ involvieren.
Den Zusammenhang zwischen Druck, Temperatur, Dichte und möglicherweise
weiteren Eigenschaften wird durch Zustandsgleichungen vermittelt.

\subsection{Kontinuitätsgleichung}
Die Kontinuitätsgleichung drückt aus, dass
Materie nicht einfach neu entstehen oder verschwinden kann.
Um sie herzuleiten, betrachten wir ein Volumen $V$ des Fluids.
Die Masse im Inneren des Volumens wird bestimmt durch das Volumenintegral
\[
m
=
\iiint_V \varrho \,dx\,dy\,dz.
\]
Ein kleiner Quader mit den Abmessungen $\Delta x$, $\Delta y$
und $\Delta z$ enthält die Masse 
\[
m = \varrho \Delta x \,\Delta y \, \Delta z.
\]
Wenn sich die Masse in dem Quader ändert, dann muss Materie durch die
Wände zu- oder abfliessen.
Wir berechnen daher für jede Wand des Quaders, wie gross der Massefluss
durch die Wand in einer Zeiteinheit $\Delta t$ ist.

Durch ein Rechteck mit Abmessungen $\Delta y \times \Delta z$ senkrecht
zur $x$-Achse fliesst in der Zeit $\Delta t$ das Volumen
$v_x\Delta x\,\Delta y\,\Delta z$ und damit die Masse
\begin{equation}
\varrho v_x\,\Delta y\,\Delta z.
\label{skript:massenausdruck}
\end{equation}
Die Dichte $\varrho$ und die Geschwindigkeit $v_x$ sind dabei an der
Koordinate $x$ zu nehmen.
Durch die Wand des Quaders bei $x+\Delta x$ fliesst eine Masse, die
ebenfalls durch den Ausdruck \eqref{skript:massenausdruck}
beschrieben werden kann, jedoch für die $x$-Koordinaten $x+\Delta x$.
Um die Massenänderung im Quader zu bestimmen, sind diese beiden Ausdrücke
als mit entgegengesetzten Vorzeichen zu berücksichtigen.

Die Massenänderung ist daher
\begin{align}
\Delta m
&=
\varrho(x,y,z,t) v_x(x,y,z,t)\,\Delta y\,\Delta z\,\Delta t
-
\varrho(x+\Delta x,y,z,t) v_x(x+\Delta x,y,z,t)\,\Delta y\,\Delta z\,\Delta t
\notag
\\
&\quad
+
\varrho(x,y,z,t) v_y(x,y,z,t)\,\Delta x\,\Delta z\,\Delta t
-
\varrho(x,y+\Delta y,z,t) v_y(x,y+\Delta y,z,t)\,\Delta x\,\Delta z\,\Delta t
\notag
\\
&\quad
+
\varrho(x,y,z,t) v_z(x,y,z,t)\,\Delta x\,\Delta y\,\Delta t
-
\varrho(x,y,z+\Delta z,t) v_z(x,y,z+\Delta z,t)\,\Delta x\,\Delta y\,\Delta t.
\notag
\intertext{Wir fassen die Terme zu gegenüberliegenden Wänden zusammen wobei
wir das Produkt $\Delta x\,\Delta y\,\Delta z$ ausklammern können.
Wir teilen ausserdem durch $\Delta t$, um die zeitliche Massenänderungsrate
zu erhalten.}
\frac{\Delta m}{\Delta t}
&=
-
\bigg(
\frac{\varrho(x+\Delta x,y,z,t)v_x(x+\Delta x,y,z,t)-\varrho(x,y,z,t)v_x(x,y,z,t)}{\Delta x}
\notag
\\
&\qquad
+\frac{\varrho(x,y+\Delta y,z,t)v_y(x,y+\Delta y,z,t)-\varrho(x,y,z,t)v_y(x,y,z,t)}{\Delta y}
\notag
\\
&\qquad
+\frac{\varrho(x,y,z+\Delta z,t)v_y(x,y,z+\Delta z,t)-\varrho(x,y,z,t)v_y(x,y,z,t)}{\Delta z}
\bigg)
\Delta x\,\Delta y\,\Delta z\,\Delta t.
\notag
\intertext{Da $\Delta m=\varrho\Delta x\,\Delta y\,\Delta z$ können wir
auf beiden Seiten durch $\Delta x\,\Delta y\,\Delta z$ dividieren.
Um die zeitliche Änderung zu bestimmen, müssen wir ausserdem durch
$\Delta t$ dividieren.
Lassen wir die Inkremente $\Delta x$, $\Delta y$, $\Delta z$ und
$\Delta t$ gegen $0$ gehen, werden aus den Differenzenquotienten
Ableitungen.
Wir erhalten daher die {\em Kontinuitätsgleichung}
\index{Kontinuitätsgleichung} }
\frac{\partial \varrho}{\partial t}
&=
-
\biggl(
\frac{\partial \varrho v_x}{\partial x}
+
\frac{\partial \varrho v_y}{\partial y}
+
\frac{\partial \varrho v_z}{\partial z}
\biggr).
\label{skript:kontinuitaetsgleichung}
\end{align}
\index{Kontinuitätsgleichung}%
Die rechte Seite kann mit Hilfe des {\em Nabla-Operators}
\index{Nabla-Operator}
\[
\nabla
=
\begin{pmatrix}
\frac{\partial}{\partial x}\\
\frac{\partial}{\partial y}\\
\frac{\partial}{\partial z}
\end{pmatrix}
\]
kürzer geschrieben werden.
Der Nabla-Operator wird wie ein Vektor behandelt.
Für eine (skalare) Funktion $f$ ist $\nabla f$ ein Vektor,
der {\em Gradient}
\index{Gradient} der Funktion $f$.
Das Skalarprodukt $\nabla\cdot\vec{v}$ ist ein Skalar, die
{\em Divergenz}
\index{Divergenz}
eines Vektorfeldes $\vec{v}$, sie wird manchmal auch 
$\operatorname{div}\vec{v}$ geschrieben.
Aus 
\eqref{skript:kontinuitaetsgleichung}
wird dann
\[
\frac{\partial \varrho}{\partial t}
=
-\nabla\cdot (\varrho\vec{v})
\]
geschrieben werden.
%So erhält die Kontinuitätsgleichung die kompakte Form
%\[
%\frac{\partial}{\partial t}\varrho = -\nabla\cdot (\varrho\vec{v}).
%\]

\subsection{Inkompressible Strömung}
Bei einem inkompressiblen Fluid ist die Dichte eine Konstante, alle
\index{Fluid!inkompressibel}
\index{inkompressibel}
Ableitungen von $\varrho$ verschwinden.
Die Kontinuitätsgleichung wird damit zu
\[
\frac{\partial\varrho}{\partial t}
=
-\nabla\cdot(\varrho\vec{v})
=
-\nabla\varrho\cdot\vec{v}
-\varrho\nabla\vec{v}
=
-\varrho\nabla\vec{v}
=
0.
\]
In einer inkompressiblen Strömung verschwindet daher die Divergenz
des Geschwindigkeitsfeldes.

\subsubsection{Verallgemeinerung}
Die Herleitung der Kontinuitätsgleichung für die Massedichte funktioniert
auch für jede andere Erhaltungsgrösse, die im Fluid mit einer Dichte
$a(x,y,z,t)$ vorhanden ist und mit der Strömung mittransportiert wird.
Die {\em verallgemeinerte Kontinuitätsgleichung} für die Erhaltungsgrösse $a$
\index{Kontinuitätsgleichung!verallgemeinerte}
ist daher
\begin{equation}
\frac{\partial a}{\partial t}
=
-
\nabla(a\vec{v}).
\label{skript:verallgemeinerte kontinuitaetsgleichung}
\end{equation}

\subsection{Bewegungsgleichung}
Das zweite Newtonsche Gesetz $F=ma$ besagt, dass Kraft und Beschleunigung
proportional sind.
Dies gilt jedoch nur, wenn die Masse unveränderlich ist.
Genauer besagt Newtons zweites Gesetz, dass die Kraft die
zeitliche Änderung des Impulses ist, also
\[
F=
\frac{d}{dt}(m\vec v).
\]
Ein Volumen des Fluides kann wegen veränderlicher Dichte seine
Masse verändern.
Kräfte auf das Fluid ändern daher die Impulsdichte des Fluids.

\subsubsection{Impulsdichte}
Die Impulsdichte des Fluids wird an jeder Stelle durch die Grösse
$\vec{p}=\varrho\vec{v}$ gegeben.
Das zweite Newtonsche Gesetz besagt dann, dass die Änderung von $\vec p$
durch die äusseren Kräfte $\vec{b}$ bestimmt wird, die auf das Fluid wirkt.
Der Impuls in einem Volumen kann aber auch ändern, dass das Fluid Impuls
in das Volumen hinein- oder aus dem Volumen heraustransportiert.
Jede Komponente des Impulses ist eine Erhaltungsgrösse, für die ohne
Wirkung äusserer Kräfte die verallgemeinerte Kontinuitätsgleichung
\eqref{skript:verallgemeinerte kontinuitaetsgleichung}
gilt.
Für die $x$-Komponente des Impulses gilt daher die Gleichung
\[
\frac{\partial \varrho v_x}{\partial t}
=
-\nabla \cdot(\varrho v_x\,\vec{v})
+\varrho b_x,
\]
und analog für die anderen Komponenten $\varrho v_y$ und $\varrho v_z$ 
der Impulsdichte.

\subsubsection{Innere Kräfte}
Damit sind aber innere Kräfte im Fluid noch nicht berücksichtigt.
Das Fluid widersetzt sich zum Beispiel der Kompression, dies äussert
sich im Druck, der jeweils senkrecht auf den Wänden des Volumens wirkt.
In einem zähen Medium sind aber auch Kräfte parallel zu den Wänden
\index{Zähigkeit}
möglich, sogenannte {\em Scherkräfte}.
\index{Scherkraft}
Im Allgemeinen wirkt auf ein $\Delta y\times\Delta z$-Rechteck senkrecht
zur $x$-Achse die Kraft
\[
\vec{\tau}_x
\,\Delta y\,\Delta z
=
\begin{pmatrix}
\tau_{xx}\\
\tau_{xy}\\
\tau_{xz}
\end{pmatrix}
\,\Delta y\,\Delta z
\]
und analog für die Wände senkrecht auf der $y$- bzw.~$z$-Achse.
Die diagonalen Komponente $\tau_{ii}$ beschreiben die Druckkraft
\index{Druck}
auf die jeweilige Seitenfläche, während die ausserdiagonalen Elemente
Scherkräfte beschreiben.

Die Matrix $\bm{\tau}$ mit Komponenten $\tau_{ij}$ heisst auch der
{\em Cauchy-Spannungstensor}.
\index{Cauchy-Spannungstensor}
\index{Spannungstensor}
Wir werden weiter unten (Seite~\pageref{skript:spannungstensor symmetrisch})
zeigen, dass $\tau_{ij}$ symmetrisch sein muss,
Dass $\tau_{ij}$ ein Tensor ist, ist für die weiteren Erörterungen nicht
von Bedeutung, wir werden daher diesen Begriff verwenden, ohne ihn wirklich
zu definieren.

Die resultierende Kraft $\vec{F}$ auf einen Quader mit den Kantenlängen
$\Delta x$, $\Delta y$ und $\Delta z$  hat daher die $i$-Komponente
\begin{align*}
F_x
&=
(
\tau_{xx}(x+\Delta x,y,z,t)
-
\tau_{xx}(x,y,z,t)
) \Delta y\,\Delta z
\\
&\qquad
+
(
\tau_{yx}(x,y+\Delta y,z,t)
-
\tau_{yx}(x,y,z,t)
) \Delta x\,\Delta z
\\
&\qquad
+
(
\tau_{zx}(x,y,z+\Delta z,t)
-
\tau_{zx}(x,y,z,t)
)\Delta x\,\Delta z
\\
&=
\bigg(
\frac{
\tau_{xx}(x+\Delta x,y,z,t)
-
\tau_{xx}(x,y,z,t)
}{\Delta x}
+
\frac{
\tau_{yx}(x,y+\Delta y,z,t)
-
\tau_{yx}(x,y,z,t)
}{\Delta y}
\\
&\qquad
+
\frac{
\tau_{zx}(x,y,z+\Delta z,t)
-
\tau_{zx}(x,y,z,t)
}{\Delta z}
\bigg)
\Delta x\,\Delta y\,\Delta z.
\end{align*}
Die Kraftdichte $f_i$ erhalten wir nach Division durch
$\Delta x\,\Delta y\,\Delta z$ und Grenzübergang, sie ist
\begin{equation}
f_x
=
\frac{\partial \tau_{xx}}{\partial x}
+
\frac{\partial \tau_{yx}}{\partial y}
+
\frac{\partial \tau_{zx}}{\partial z}.
\label{skript:spannungskraftdichte}
\end{equation}
Wir können damit die vollständige Bewegungsgleichung für das Fluid
hinschreiben, sie lautet
\begin{equation}
\frac{\partial \varrho v_x}{\partial t}
=
-\nabla\cdot (\varrho v_x\vec{v})
+
\varrho b_x
+
f_x.
\label{skript:navier-stokes komponente}
\end{equation}
\subsubsection{Vektorschreibweise}
Die Schreibweise
\eqref{skript:navier-stokes komponente}
für die Bewegungsgleichungen ist sehr schwerfällig und passt nicht
zu der deutlich elegantere vektoriellen Schreibweise zum Beispiel
der Kontinuitätsgleichung.
Die linke Seite von
\eqref{skript:navier-stokes komponente}
und der mittlere Term auf der rechten Seite können natürlich sofort
in eine vektorielle Schreibweise überführt werden, nicht jedoch die
anderen zwei Terme.

Der Term $\nabla \cdot(\varrho v_x\vec{v})$ ist ausgeschrieben
\[
\nabla \cdot(\varrho v_x\vec{v})
=
\frac{\partial}{\partial x}
(\varrho v_xv_x)
+
\frac{\partial}{\partial y}
(\varrho v_xv_y)
+
\frac{\partial}{\partial z}
(\varrho v_xv_z).
\]
Dieser Ausdruck sieht ganz ähnlich aus wie der Ausdruck
\eqref{skript:spannungskraftdichte}
für die $x$-Kompontente der Kraftdichte der inneren Kräfte.
Wir können die Ähnlichkeit formal noch etwas klarer machen.
Schreiben wir $A_{xy} = \varrho v_xv_y$, dann ist
\[
\nabla \cdot(\varrho v_x\vec{v})
=
\frac{\partial A_{xx}}{\partial x}
+
\frac{\partial A_{xy}}{\partial y}
+
\frac{\partial A_{xz}}{\partial z}
=
\sum_{i}\frac{\partial A_{xi}}{\partial i}.
\]
Da es offenbar auf die Reihenfolge der Indizes von $A$ nicht ankommt,
ist dies auch das gleiche wie
\[
\nabla \cdot(\varrho v_x\vec{v})
=
\frac{\partial A_{xx}}{\partial x}
+
\frac{\partial A_{yx}}{\partial y}
+
\frac{\partial A_{zx}}{\partial z}
=
\sum_{i}\frac{\partial A_{ix}}{\partial i}.
\]
Wir können daher die Wirkung des Nabla-Operators $\nabla$ auf einer
symmetrischen Matrix $A$ wie folgt definieren:

\begin{definition}
\label{skript:definition divergenz}
\index{Divergenz!einer Matrix}
Ist $A_{ij}$ eine symmetrische Matrix, dann ist die {\em Divergenz}
$\nabla\cdot A$
von
$A$ der Vektor mit den Komponenten
\[
(\nabla\cdot A)_x
=
\sum_{i}\frac{\partial A_{ix}}{\partial i}.
\]
\end{definition}
Falls die Matrix $\tau_{ij}$ symmetrisch ist, kann diese Definition
auch auf $\bm{\tau}$ angewendet werden.
Die $x$-Komponente der Divergenz von $\bm{\tau}$ ist dann
\[
(\nabla\cdot \bm{\tau})_x
=
\frac{\partial \tau_{xx}}{\partial x}
+
\frac{\partial \tau_{yx}}{\partial y}
+
\frac{\partial \tau_{zx}}{\partial z}
=
f_x.
\]
Dies ist genau der letzte Term in der Gleichung
\eqref{skript:navier-stokes komponente}.

Wir brauchen jetzt nur noch eine kompaktere Notation für die Matrix
$\varrho v_xv_y$.

\begin{definition}
\index{Kronecker-Produkt}
Das {\em Kronecker-Produkt} zweier Vektoren $\vec{a}$ und $\vec{b}$ ist die
Matrix $\vec{a}\otimes\vec{b}=\vec{a}\vec{b}^t$ mit den Komponenten
\[
(\vec{a}\otimes\vec{b})_{ij}
=
a_ib_j
=
(
\vec{a}
\vec{b}^t
)_{ij}
\]
Abgekürzt erlauben wir die Schreibweise $\vec{a}\otimes\vec{b}=\vec{a}\vec{b}$.
\end{definition}

Mit diesen Notationen bekommen wir jetzt die Bewegungsgleichungen in
Vektorform.
Sie lauten
\begin{equation}
\frac{\partial \varrho\vec v}{\partial t}
=
-\nabla\cdot(\varrho\vec{v}\vec{v})
+ \varrho\vec{b}
+ \nabla\cdot \bm{\tau}.
\label{skript:navier-stokes1}
\end{equation}
Dies ist die {\em Navier-Stokes Gleichung}.
\index{Navier-Stokes Gleichung}
Die drei Terme beschreiben die Impulsänderung durch den Zu- oder Abtransport
von Impuls durch die Strömung, durch die äusseren Kräfte bzw.~die inneren
Spannungen.

\subsubsection{Symmetrie des Spannungstensors}
\label{skript:spannungstensor symmetrisch}
\begin{figure}
\centering
\includegraphics{chapters/2/drehmoment.pdf}
\caption{Drehmoment um die $z$-Achse der Schwerkräfte auf einen Würfel
mit Kantenlänge $2l$. Gezeigt sind nur die Komponenten von $\bm{\tau}$,
die zu einem Drehmoment führen.
\label{skript:drehmoment}}
\end{figure}
In diesem Abschnitt wollen wir nachweisen, dass der Spannungstensor
symmetrisch ist.
Dazu betrachten wir das Drehmoment, welches die Scherkräfte auf einen
kleinen Würfel mit Kantenlänge $2l$ ausüben (Abbildung \ref{skript:drehmoment}).

Der Würfel hat die Masse $m=\varrho(2l)^3$.
Das Trägheitsmoment eines Würfels mit Masse $m$ und Kantenlänge $2l$
ist
\[
I_z
=
\frac1{12}m((2l)^2+(2l)^2)
=
\frac1{12}\varrho 8l^3\cdot8l^2
=
\frac{16}{3}\varrho l^5.
\]
Der Drehimpuls um die $z$-Achse ist $L_z=I_z\omega$.

Aus Abbildung~\ref{skript:drehmoment} kann man die Scherkräfte auf den
Seitenflächen ablesen, sie sind $\tau_{xy}4l^2$ bzw.~$\tau_{yx}4l^2$,
ihr Hebelarm ist $l$.
Das resultierende Drehmoment um die $z$-Achse ist daher
\[
M_z = 8l^3\tau_{xy} - 8l^3\tau_{yx}.
\]
Die Bewegungsgleichungen eines starren Körpers besagen jetzt, dass
für die Winkelgeschwindigkeit der Drehung des Würfels um die $z$-Achse
die Gleichung
\[
\frac{dL_z}{dt}
=
M_z
\qquad\Rightarrow\qquad
I_z\dot\omega
=
M_z
\qquad\Rightarrow\qquad
\dot\omega
=
\frac{M_z}{I_z}
=
\frac{8l^3(\tau_{xy}-\tau_{yx})}{\frac{16}{3}l^5}
=
\frac{3}{2l^2}(\tau_{xy}-\tau_{yx}).
\]
Wir nehmen an, es sei $\tau_{xy}\ne\tau_{yx}$.
Lässt man $l$ gegen $0$ gehen, folgt die Aussage, dass die
Winkelgeschwindigkeit eines sehr kleinen Würfels im Fluid sich mit beliebig
schnell anwachsender Winkelgeschwindigkeit drehen müsste.
Dieses unphysikalische Resultat erlaubt zu schliessen, dass
$\tau_{xy}=\tau_{yx}$ sein muss und dass nur ein
symmetrischer Spannungstensor ein physikalisches Fluid beschreibt.

\subsubsection{Druck und Spannungen}
Die Diagonalelemente des Spannungstensors $\bm{\tau}$ beschreiben
Normalkräfte auf ein Volumenelement des Fluids.
Im Gleichgewicht sind sie alle gleich gross und stimmen mit dem
negativen {\em (hydrostatischen) Druck} überein, wir setzen daher
\index{Druck}
\index{hydrostatischer Druck}
\[
p=-\frac13\operatorname{Spur}\bm{\tau}.
\]
Wir können daher $\bm{\tau}$ zerlegen in eine Diagonalmatrix
mit Elementen $-p$ auf der Diagonalen und eine spurlose Matrix
\[
\bm{\tau} = -pE + \bm{\sigma},
\]
$E$ ist die Einheitsmatrix.
Die spurlose symmetrische Matrix $\bm{\sigma}$ heisst auch
{\em Spannungsdeviator}.
\index{Spannungsdeviator}

Für die Bewegungsgleichung brauchen wir die Divergenz beider Terme.
Die Druckterme sind alle gleich, nach
Definition~\ref{skript:definition divergenz} ist
\[
(\nabla\cdot(pE))_x
=
\sum_i
\frac{\partial p\delta_{xi}}{\partial i}
=
\frac{\partial p}{\partial x}
\qquad\Rightarrow\qquad
\nabla\cdot(pE)
=
\nabla p.
\]
Damit wird die Bewegungsgleichung 
\begin{equation}
\frac{\partial \varrho\vec{v}}{\partial t}
=
-\nabla\cdot(\varrho\vec{v}\vec{v})
+\varrho\vec b
-\nabla p
+\nabla\cdot\bm{\sigma}
\label{skript:navier-stokes2}
\end{equation}
Die Scherkräfte sind in einem newtonschen Fluid proportional zu
den Schergeschwindigkeiten.
Man kann zeigen (siehe \cite[p.~172]{skript:kaperengler}), dass $\bm{\sigma}$
geschrieben werden kann als
\[
\bm{\sigma}
=
2\nu\biggl(\bm{\varepsilon} - \frac13(\nabla\cdot\vec{v})E\biggr)
\qquad\text{mit}\qquad
\bm{\varepsilon}=\frac12\bigl(\nabla\vec{v}+(\nabla\vec{v})^t\bigr).
\]
Die spezielle Form von $\bm{\varepsilon}$ ist notwendig, damit die Matrix
$\bm{\varepsilon}$ symmetrisch wird.
Der zweite Term im Ausdruck von $\bm{\sigma}$ ist nötig, damit die Spur
\[
\operatorname{Spur}{\bm{\sigma}}
=
2\nu(\operatorname{\varepsilon} - \nabla\cdot\vec{v})
=
2\nu
\frac12
\biggl(
\sum_i \frac{\partial v_i}{\partial i}
+
\frac12
\sum_i \frac{\partial v_i}{\partial i}
-
\nabla\cdot\vec{v}
\biggr)
=
0
\]
von $\bm{\sigma}$ verschwindet.

Die Divergenz $\nabla\cdot\bm{\sigma}$ von $\bm{\sigma}$ kann damit explizit
durch die Geschwindigkeit ausgedrückt werden.
Wir berechnen die Divergenz der einzelnen Terme:
\begin{align}
(\nabla\cdot\bm{\varepsilon})_x
&=
\sum_i\frac{\partial \varepsilon_{ix}}{\partial i}
=
\frac12
\sum_i\frac{\partial}{\partial i}\biggl(
\frac{\partial v_i}{\partial x}+\frac{\partial v_x}{\partial i}
\biggr)
=
\frac{\partial}{\partial x}
\sum_i\frac{\partial v_i}{\partial i}
+
\frac12
\sum_i \frac{\partial^2 v_x}{\partial i^2}
=
\frac12\frac{\partial}{\partial x}
(\nabla\cdot\vec{v})
+
\frac12\Delta v_x
\notag
\\
\nabla\cdot\bm{\varepsilon}
&=
\frac12\nabla(\nabla\cdot\vec{v})
+
\frac12\Delta\vec{v}
\notag
\\
(\nabla\cdot(\nabla\cdot\vec{v})E)_x
&=
\sum_i \frac{\partial}{\partial i} (\nabla\cdot\vec{v}E)_{xi}
=
\sum_i \frac{\partial}{\partial i} (\nabla\cdot\vec{v}\delta_{xi})
=
\frac{\partial}{\partial x}(\nabla\cdot\vec{v})
\notag
\\
\nabla\cdot(\nabla\cdot\vec{v})E)
&=
\nabla(\nabla\cdot\vec v)
\notag
\\
\intertext{und erhalten so für die Divergenz von $\bm{\sigma}$:}
\nabla\cdot\bm{\sigma}
&=
2\nu\biggl(
\nabla\cdot\bm{\varepsilon}
-\frac13\nabla\cdot((\nabla\cdot\vec{v})E)
\biggr)
=
2\nu\biggl(
\frac12
\nabla(\nabla\cdot\vec{v})
+
\frac12\Delta\vec{v}
-\frac13
\nabla(\nabla\cdot\vec{v})
\biggr)
\\
&=
\nu\Delta\vec{v}
+\frac{\nu}3\nabla(\nabla\cdot\vec{v}).
\label{skript:sigmadiv}
\end{align}

\subsubsection{Inkompressible Strömung}
In einem inkompressiblen Fluid ist $\nabla\cdot\vec{v}=0$, dann fällt
der zweite Term in \eqref{skript:sigmadiv} weg.
Die Strömungsgleichung eines inkompressiblen Fluids erhält damit die
einfache Form
\begin{equation}
\frac{\partial\vec{v}}{\partial t}
=
-\nabla\cdot(\vec{v}\vec{v})
+\vec{b}
-\frac1{\varrho}(\nabla p
-\nu\Delta\vec{v}),
\label{skript:inkompressibel newtonsch}
\end{equation}
die klassische Navier-Stokes Gleichung.

\subsection{Zustandsgleichungen}
Die Dichte hängt vor allem auch von der Temperatur ab.
In den Ozeanen ändert die Dichte des Wassers mit dem Salzgehalt.
Eine vollständige Beschreibung der Strömung in Ozeanen oder der
Atmosphäre muss daher auch noch weitere Variablen modellieren.
In Kapitel~\ref{chapter:wetter und klima} haben wir bereits auf die
Wärmeleitungsgleichungen hingewiesen.

Die Felder $T$, $p$ und $\varrho$ sind bei einem idealen Gas miteinander
durch die Zustandsgleichung
\[
p=\varrho T R_s
\]
mit der spezifischen Gaskonstante $R_s$ verbunden.
Für den Zusammenhang von Dichte, Temperatur und Salzgehalt gibt
es jedoch kein derart einfaches Modell.
Eine weitere Kopplung zwischen der Temperatur und der
Strömung entsteht durch die Viskosität $\nu$, die sehr stark
von der Temperatur abhängt.
Auch dafür gibt es keine einfachen Modell.

In vielen Fällen schwanken die physikalischen Grössen nur geringfügig
um einen Mittelwert.
Zum Beispiel hängt die Dichte $\varrho$ von Meerwasser sowohl von
der Temperatur $T$ als auch vom Salzgehalt $h$ ab, die Dichte ist
also eine Funktion $\varrho(T,h)$.
Wir können $\varrho$ als Taylorreihe um die mittlere Temperatur $T_0$
und den mittleren Salzgehalt $h_0$ entwickeln:
\[
\varrho(T,h)
=
\varrho_0 -\alpha(T-T_0) + \beta(h-h_0).
\]
In Klimamodellen betrachten wir typischerweise nur kleine Abweichungen
von Mittelwerten, so dass ein solches Modell sehr erfolgreich sein kann.

\subsection{Boussinesq-Approximation}
Die Strömung in der Erdatmosphäre kann offensichtlich nicht als
inkompressibel betrachtet werden, die Dichte ist offenbar nicht
konstant.
Der Zustand der Atmosphäre weicht jedoch nur wenig einem mittleren
Dichteprofil $\varrho_0$ ab, welches im wesentlich durch das Temperaturprofil
festgelegt ist.
Im Normalzustand nimmt die Temperatur der Atmosphäre ziemlich genau
linear ab bis zur Höhe der Thermopause.
Auf die horizontale Komponente der Strömung hat eine Abweichung des
Temperaturprofils kaum einen Einfluss, denn andere Terme der
Navier-Stokes-Gleichung
\eqref{skript:navier-stokes2}
sind bedeutender.
Für die vertikale Bewegung ist der Term der äusseren Kräfte,
nämlich die Schwerkraft, dominant.
Wir können dies berücksichtigen, indem wir die Erdbeschleunigung
$g$ durch
\begin{equation}
g\frac{\varrho}{\varrho_0}
\label{skript:boussinesq}
\end{equation}
ersetzen.
Diese Approximation ist bekannt als die Boussinesq-Approximation.
Für unsere Zwecke hier brauchen wir nicht mehr als \eqref{skript:boussinesq}.
Dies wird bei der Herleitung der Lorenz-Gleichung in Abschnitt
\ref{section:lorenz-modell} benötigt.
Für die vollständigen Boussinesq-Gleichungen siehe \cite{skript:kaperengler}.



%
% lorenz.tex -- Lorenz-Modell
%
% - Herleitung aus den Gleichungen der Fluiddynamik
% - Dynamik aus numerischen Simulationen
%
% (c) 2018 Prof Dr Andreas Müller, Hochschule Rapperswil
%
\section{Lorenz-Modell}
Sowohl die Atmosphäre als auch die Ozeane werden durch die hydrodynamischen
Gleichungen beschrieben.
Es stellt sich damit die Frage, in welchem Masse sich daraus eine praktikable
Vorhersage sowohl von Wetter also auch des Klimas ableiten lässt.
In den Sechzigerjahren hat Edward Lorenz versucht, diese Frage mit einem
vereinfachten Modell zu beantworten.
Ziel dieses Abschnittes ist, das Lorenz-Modell aus den Gleichungen der
Fluiddynamik herzuleiten.

\subsection{Modellbeschreibung}
\begin{figure}
\centering
\includegraphics{chapters/2/lorenz-definition.pdf}
\caption{Definitionsgebiet für das Lorenz-Modell der Atmosphäre.
Gesucht sind Temperatur $T(x,y,t)$, Dichte $\varrho(x,y,t)$ und
Geschwindigkeit $\vec{v}(x,y,t)$ in einem Rechteckgebiet
$\mathbb R\times [0,\pi]$.
Die Temperatur ist an den Rändern vorgegeben, es gilt
$T(x,0,t)=T_0$ und $T(x,\pi,t)=0$.
Im Inneren Gebiet wird die Schwerkraft $g$ auf die Luft.
\label{skript:lorenzmodell definitionsgebiet}}
\end{figure}
Es soll ein dünner Schnitt durch die Atmosphäre modelliert werden.
Da Atmosphäre im Vergleich zur Krümmung der Erdoberfläche sehr dünn ist,
können wir sie als eben annehmen.
Wir verwenden die Koordinate $x$ parallel zur Erdoberfläche und $y$ als Höhe
(Abbildung~\ref{skript:lorenzmodell definitionsgebiet}).
Gesucht ist also die Temperatur $T(x,y,t)$ und die Dichte $\varrho(x,y,t)$
in Abhängigkeit von Position und Zeit sowie der Geschwindigkeitsvektor
\[
\vec v
=
\begin{pmatrix}v_x\\v_y\end{pmatrix}
=
\begin{pmatrix}v_x(x,y,t)\\v_y(x,y,t)\end{pmatrix}.
\]
Die Funktionen $T$, $\varrho$, $v_x$ und $v_y$ sind definiert in einem
Streifen.
Der Einfachheit halber wählen wir die Höhe des Streifens als $\pi$.
Wir können dies erreichen, indem wir die Längeneinheit geeignet wählen:
ist $h$ die ``Dicke'' der Atmosphäre\footnote{Die Konvektion in der Atmosphäre,
welche vom Lorenz-Modell vor allem beschrieben wird, findet im Wesentlichen
nur im untersten Teil der Atmosphäre, der sogenannten Troposphäre statt.
Die Troposphäre zeichnet sich aus durch mehr oder weniger lineare
Temperaturabnahme bis zur Höhe der sogenannten Tropopause in etwa
10km Höhe.
Wir können also die Höhe der Tropopause als $h$ verwenden.}, wählen wir
$h/\pi$ als Längeneinheit.
Das Definitionsgebiet für die Funktionen ist daher $R=\mathbb R\times [0,\pi]$.

Die Temperatur der Atmosphäre an der Erdoberfläche wird im wesentlichen von
der Temperatur des Bodens bestimmt, der von der einfallenden Strahlung
ergewärmt wird, es soll also $T(x,0,t)=T_0$ gelten.
Am oberen Rand des Schnittes schliesst die sehr dünne Hochatmosphäre an,
die im Wesentlichen in einem Strahlungsgleichgewicht mit der Umgebung steht.
Da wir die Dichte im wesentlichen als konstant ansehen wollen und damit
den Einfluss der Temperatur auf die Dichte nicht exakt modellieren wollen,
sind wir nicht gezwungen, eine bestimmte Temperaturskala zu verwenden.
Wir können daher willkürlich die Temperatur am oberen Rand als
$T(x,\pi,t)=0$ festlegen.

Auf das Medium im Streifen wirkt natürlich die Erdbeschleunigung,
die wir ebenfalls als konstant annehmen dürfen, da die Dicke der 
Atmosphäre im Vergleich zum Erdradius sehr klein ist.

\subsubsection{Stabile Atmosphäre}
\begin{figure}
\centering
\includegraphics{chapters/2/lorenz-stabil.pdf}
\caption{Stabilität der Atmosphäre: bewegt sich ein Luftpaket in der
Atmosphäre nach oben oder unten, expandiert oder kontrahiert es und
verändert seine Temperatur adiabatisch (grün).
Ist diese Temperaturänderung grösser als der aktuelle Temperaturgradient
(links),
ändert sich die Dichte der Luft weniger stark als die der Umgebungsluft,
die Bewegung wird gestoppt, die Atmosphäre ist im Gleichgewicht.
Andernfalls wird die Bewegung beschleunigt, die Atmosphäre ist instabil
(rechts).
\label{skript:stabilitaet der atmosphaere}}
\end{figure}
Die Temperatur muss im Gebiet von unten nach oben abnehmen.
Aber auch der Druck muss mit zunehmender Höhe abnehmen. 
Wenn ein Luftpaket aufsteigt, wird es wegen des geringer werdenden
Druckes expandieren und damit adiabatisch abkühlen.
Wenn die Temperatur der umgebenden Luft schneller abnimmt als die
adiabatische Abkühlung, dann ist das Luftpaket in seiner neuen Höhe
wärmer und damit leichter als die Umgebung, es wird weiter ansteigen
(Abbildung~\ref{skript:stabilitaet der atmosphaere}).
Wenn die Temperatur der umgebenden Luft langsamer abnimmt als die
adiabatische Abkühlung, dann ist das Luftpaket in der neuen Höhe 
kälter und damit Dichter als die Umgebung, es wird wieder absinken.
Solange der Temperaturunterschied nicht zu gross ist, wird sich also ein
Zustand einstellen, in dem die Luft in Ruhe bleibt, der Wärmetransport
erfolgt ausschliesslich durch Wärmeleitung.

\subsubsection{Instabilität}
Bei genügend grosser Temperaturdifferenz wird die Atmosphäre jedoch
instabil, der Wärmetransport wird zusätzlich von Konvektion übernommen.
Die entstehenden Konvektionszellen können wegen der Translationssymmetrie
entlang der $x$-Achse an einer beliebigen Stelle entstehen, es gibt also
unendlich viele Lösungen, von denen eine gewählt werden muss.
In der Realität würden kleine Temperaturfluktionen dies unterstützen,
kleine Unterschiede in den Anfangsbedingungen führen also zu völlig
verschiedenen Strömungen.
% Abbildung?
Diese sensitive Abhängigkeit der Lösung von Anfangsbedingungen wird
oft als ein Kennzeichen von Chaos angesehen.

Im folgenden sollen zunächst die Gleichungen der Fluiddynamik auf die
vorliegende Situation spezialisiert werden.
Mit Hilfe eines geeigneten Ansatzes soll dann die partielle
Differentialgleichung weiter auf ein System von gewöhnlichen 
Differentialgleichungen reduziert werden.
In numerischen Simulationen soll schliesslich gezeigt werden, dass die 
Lorenz-Gleichungen tatsächlich chaotische Lösungen haben.

\subsubsection{Kontinuitätsgleichung}
\subsubsection{Bewegungsgleichung}

\subsection{Grundgleichungen}

\subsection{Umwandlung in ein gewöhnliches Differentialgleichungssystem}






