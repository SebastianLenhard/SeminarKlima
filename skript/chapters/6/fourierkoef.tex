%
% fourierkoef.tex -- Bestimmung der Fourier-Koeffizienten
%
% (c) 2018 Prof Dr Andreas Müller, Hochschule Rapperswil
%
\section{Fourier-Koeffizienten}
Bestimmen wir die Fourierkoeffizienten für ein trigonometrisches Polynom
der Form
\begin{equation}
f(t)
=
a_0 + \sum_{k=1}^{n-1} a_k\cos kt + \sum_{k=1}^{n-1} b_k\sin kt + a_n\cos nt,
\label{skript:fourier:ansatz}
\end{equation}
so dass die Werte
\begin{equation}
y_j \quad\text{zu den Zeitpunkten}\quad t_j=2\pi\frac{j}{n},\qquad 1\le j\le N
\label{skript:fourier:gleichungen}
\end{equation}
möglichst genau die Funktionswerte $f(t_j)$ reproduzieren.
Im Ansatz~\eqref{skript:fourier:ansatz} finden wir $2n$ zu bestimmende 
Koeffizienten, in \eqref{skript:fourier:gleichungen} finden wir dagegen
genau $N$ Gleichungen, mit denen wir die Koeffizienten bestimmen könnten.
Für $N>2n$ haben wir also zu viele Daten, für $N<2n$ reichen die Datenpunkte
nicht, die Koeffizienten zu bestimmen.
Im optimalen Fall, für $N=2n$ sollte es möglich sein, die Koeffizienten so
zu bestimmen, dass die Funktionswerte $y_j$ exakt reproduziert werden.

\subsection{Least Squares}
Für $N>2n$ können wir nicht erwarten, dass der Ansatz die Daten
exakt reproduzieren kann, wir müssen uns also mit einer
Näherungslösung begnügen.
Wir verlangen stattdessen, dass der Fehler der Lösung möglichst gering
wird, dass also
\[
L=L(a_0,a_1,\dots,a_n,b_1,\dots,b_{n-1})= \sum_{j=1}^N (y_j - f(t_j))^2
\]
möglichst klein wird.

Die Grösse $L$ wird minimal, wenn alle Ableitungen nach den Koeffizienten
verschwinden:
\[
\frac{\partial L}{\partial a_0}=0,
\qquad
\frac{\partial L}{\partial a_l}=0, \;{1\le l\le n}
\qquad
\frac{\partial L}{\partial b_l}=0, \;{1\le l\le n-1}
\]
Wir berechnen die Ableitungen:
\begin{align*}
\frac{\partial L}{\partial a_0}
&=
-2 \sum_{j=1}^N (y_j-f(t_j))\cdot \frac{\partial f}{\partial a_0}(t_j)=0
&&
\\
\frac{\partial L}{\partial a_l}
&=
-2 \sum_{j=1}^N (y_j-f(t_j))\cdot \frac{\partial f}{\partial a_l}(t_j)=0
&k&=1,\dots,n
\\
\frac{\partial L}{\partial b_l}
&=
-2 \sum_{j=1}^N (y_j-f(t_j))\cdot \frac{\partial f}{\partial b_l}(t_j)=0
&k&=1,\dots,n-1
\end{align*}
Man beachte, dass der erste Klammerausdruck in der Summe die Koeffizienten
$a_0$, $a_k$ und $b_k$ nur linear enthält.
Die Funktion $f$ enthält die Koeffizienten ebenfalls nur linear, so
dass die Ableitungsterme die Koeffizienten nicht mehr enthalten werden.
Tatsächlich ergibt die Berechnung der Ableitungen
\begin{align*}
\frac{\partial f}{\partial a_0}(t_j)
&=
1
&&
\\
\frac{\partial f}{\partial a_l}(t_j)
&=
\cos lt_j
&l&=1,\dots,n
\\
\frac{\partial f}{\partial b_l}(t_j)
&=
\sin lt_j
&l&=1,\dots,n-1
\end{align*}
Die Gleichungen, die wir lösen müssen, sind also
\begin{equation}
\begin{aligned}
0&=
\sum_{j=1}^N
\biggl(
y_j - a_0 - \sum_{k=1}^n a_k\cos kt_j \sum_{k=1}^{n-1} b_k\sin kt_j
\biggl)
\\
&=
\sum_{j=1}^N y_j
-Na_0
-\sum_{k=1}^n a_k \sum_{j=1}^N\cos kt_j
-\sum_{k=1}^{n-1} b_k \sum_{j=1}^N\sin kt_j,
\\
0&=
\sum_{j=1}^N
\biggl(
y_j - a_0 - \sum_{k=1}^n a_k\cos kt_j \sum_{k=1}^{n-1} b_k\sin kt_j
\biggl) \cos lt_j
\\
&=
\sum_{j=1}^N y_j \cos lt_j
-a_0\sum_{j=1}^N \cos lt_j
-\sum_{k=1}^n a_k \sum_{j=1}^N\cos kt_j\cos lt_j
-\sum_{k=1}^{n-1} b_k \sum_{j=1}^N\sin kt_j\cos lt_j,
\\
0
&=
\sum_{j=1}^N
\biggl(
y_j - a_0 - \sum_{k=1}^n a_k\cos kt_j \sum_{k=1}^{n-1} b_k\sin kt_j
\biggl) \sin lt_j
\\
&=
\sum_{j=1}^N y_j \sin lt_j
-a_0\sum_{j=1}^N \sin lt_j
-\sum_{k=1}^n a_k \sum_{j=1}^N\cos kt_j\sin lt_j
-\sum_{k=1}^{n-1} b_k \sum_{j=1}^N\sin kt_j\sin lt_j.
\end{aligned}
\label{skript:fourier:gl2}
\end{equation}
Hier haben wir die Faktoren $-2$ ebenfalls weggelassen.

\subsection{Trigonometrische Summen}
In den Gleichungen~\eqref{skript:fourier:gl2} treten trigonometrische Summen
der Form
\begin{equation}
\sum_{j=1}^N \cos kt_j
\qquad
\text{oder}
\qquad
\sum_{j=1}^N \sin kt_j,
\label{skript:fourier:trigosum}
\end{equation}
der Summen von Produkten wie
\begin{equation}
\sum_{j=1}^N \cos kt_j\cos lt_j,\quad
\sum_{j=1}^N \cos kt_j\sin lt_j
\quad
\text{oder}
\quad
\sum_{j=1}^N \sin kt_j\sin lt_j
\end{equation}
auf.
In diesem Abschnitt sollen diese Summen mit Hilfe einer geometrischen
Überlegung berechnet werden.

Wir befassen uns zunächst mit Summen der Form
\label{skript:fourier:trigosum} und beweisen den folgenden Satz.

\begin{figure}
\centering
\begin{tikzpicture}[>=latex,thick]
\end{tikzpicture}
\begin{caption}
Verteilung der Punkte $(\cos t_j, \sin t_j)$  auf dem Einheitskreis.
\end{caption}
\end{figure}

\begin{satz}
Für beliebige ganze Zahlen $l$, $0\le l\le n$, gilt
\begin{equation*}
\begin{aligned}
\sum_{j=1}^N \cos lt_j
&=
\begin{cases}
N&\qquad l=0\\
0&\qquad\text{sonst}
\end{cases}
\\
\sum_{j=1}^N \sin lt_j
&=0
\end{aligned}
\end{equation*}
\end{satz}

\subsection{Bestimmung von $a_0$}


