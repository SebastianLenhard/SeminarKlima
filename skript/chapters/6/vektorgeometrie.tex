%
% vektorgeometrie.tex
%
% (c) 2018 Prof Dr Andreas Müller, Hochschule Rapperswil
%
\section{Vektorgeometrische Interpretation}
\rhead{Vektorschreibweise}
\subsection{Vektoren}
Die Operationen zur Bestimmung der Fourier-Koeffizienten können in 
vektorieller Schreibweise etwas übersichtlicher dargestellt werden.
Zunächst fassen wir die Funktionswerte $y_j$ in einem Vektor zusamen.
\begin{equation}
y = \begin{pmatrix}y_1\\\vdots\\y_N\end{pmatrix}
\end{equation}
Zur Berechnung der Fourier-Koeffizienten brauchen wir auch noch die
Werte der trigonometrischen Funktionen zu den Zeiten $t_j$, die wir
ebenfalls als Vektoren
\begin{align*}
c_0&=\begin{pmatrix}0\\\vdots\\0\end{pmatrix},
&
c_k&=\begin{pmatrix}\cos kt_1\\\vdots\\\cos kt_N\end{pmatrix},\;(k=1,\dots,n)
&&\text{und}
&
s_k&=\begin{pmatrix}\sin kt_1\\\vdots\\\sin kt_N\end{pmatrix},\;(k=1,\dots,n-1)
\end{align*}
schreiben.
Die Fourier-Koeffizienten können jetzt als Skalarprodukte geschrieben werden:
\begin{align*}
a_0 &=\frac1N c_0\cdot y,&
a_k &=\frac2N c_k\cdot y,\;(k=1,\dots,n),&
b_k &=\frac2N s_k\cdot y,\;(k=1,\dots,n-1).
\end{align*}

\subsubsection{Rekonstruktion der Funktion}
Auch die Darstellung der Funktion kann man wieder als Skalarprodukt schreiben.
Dazu schreiben wir die Fourier-Koeffizienten und die Werte der
trigonometrischen Funtionen also Vektoren
\begin{align*}
a
&=
\begin{pmatrix}
a_0\mathstrut\\
a_1\mathstrut\\
b_1\mathstrut\\
a_2\mathstrut\\
b_2\mathstrut\\
\vdots\\
b_{n-1}\mathstrut\\
a_n\mathstrut
\end{pmatrix}
&&\text{und}
&
e(t)
&=
\begin{pmatrix}
1\\
\cos t\\
\sin t\\
\cos2t\\
\sin2t\\
\vdots\\
\sin(n-1)t\\
\cos nt
\end{pmatrix}.
\end{align*}
Damit wird 
\[
p(t) = a\cdot e(t)
\]

\subsubsection{Fourier-Transformation}
Die Berechnung der Fourier-Koeffizienten ist eine lineare Operation
\[
a
=
\begin{pmatrix}
1           &1           &\dots &1            \\
\cos t_1    &\cos t_2    &\dots &\cos t_N     \\
\sin t_1    &\sin t_2    &\dots &\sin t_N     \\
\cos 2t_1   &\cos 2t_2   &\dots &\cos 2t_N    \\
\sin 2t_1   &\sin 2t_2   &\dots &\sin 2t_N    \\
\vdots      &\vdots      &\vdots&\vdots       \\
\sin(n-1)t_1&\sin(n-1)t_2&\dots &\sin(n-1)t_N \\
\cos nt_1   &\cos nt_2   &\dots &\cos nt_N    
\end{pmatrix}
\]

\subsection{Fast Fourier Transform}

