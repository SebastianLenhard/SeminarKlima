%
% crit.tex
%
% (c) 2018 Prof Dr Andreas Müller, Hochschule Rapperswil
%
\section{Gleichgewichtslösungen und kritische Punkte}
\rhead{Gleichgewichtslösungen}
Wir gehen in diesem Abschnitt von einer autonomen Differentialgleichung
der Form
\begin{equation}
\frac{dx}{dt} = f(x)
\label{skript:dgl:gleichung}
\end{equation}
mit einer Funktion $f\colon \mathbb R^n \to \mathbb R^n$.
Die Funktion $f$ wird im Allgemeinen von weiteren Parametern abhängen
wie zum Beispiel dem $\text{CO}_2$-Gehalt der Atmosphäre oder der Salinität
der Meere.
Sofern nötig machen wir diese mit der Schreibweise $f(x,p)$ mit
$p\in\mathbb R^m$ explizit sichtbar.

\begin{definition}
Ein Punkt $x_0\in\mathbb R^n$ heisst {\em Gleichgewichtslösung}
wenn die konstante Funktion $x(t)=x_0$ eine Lösung der
Differentialgleichung~\ref{skript:dgl:gleichung} ist.
\index{Gleichgewichtslösung}
\end{definition}

Eine Gleichgewichtslösung ist daher eine Nullstelle der Funktion $f$,
$f(x_0)=0$.
\begin{definition}
Eine Nullstelle von $f$ heisst {\em kritischer Punkt} der
Differentialgleichung~\ref{skript:dgl:gleichung}.
\end{definition}

Da $f$ ausserdem von den Parametern $p\in\mathbb R^m$ abhängt,
wird die Menge der Nullstellen von $f$ von $p$ abhängen.
Wir schreiben
\[
N(p) = \{x\in\mathbb R^n\;|\; f(x,p)\}
\]
für die Menge der Nullstellen. 
Im Allgemeinen werden sich die Mengen $N(p)$ für verschiedene $p$ 
unterscheiden.

Sei also $p_0\in\mathbb R^m$ ein Parametervektor und $x_0$ eine
Gleichgewichtslösung von $f$, also $f(x_0,p_0)=0$.
Unter zusätzlichen Annahmen über die Funktion $f$ kann man zeigen,
dass $x_0$ in einer Umgebung von $p_0$ zu einer Funktion
$x_0(p)$ erweitert werden kann, derart dass $x_0(p)$ jeweils ein
kritischer Punkt von $f$ ist für die Parameterwerte $p$, also
$f(x_0(p),p)=0$.
Diese Theorie ist für allerdings nicht besonders nützlich, denn
sie sagt uns nur, dass sich kritische stetig in Abhängigkeit vom
Parametervektor bewegen.
Besonders interessant für die Diskussion des Klimawandels sind
Fälle, wo Gleichgewichtslösungen sich sprunghaft ändern.

\section{Bifurkationen eindimensionaler Systeme}
\rhead{Bifurkationen}
In diesem Abschnitt betrachten wir eindimensionale Differentialgleichung
\begin{equation}
\frac{dx}{dt} = f(x,\lambda),
\end{equation}
die ausserdem von einem Parameter $\lambda$ abhängt.
Wir fragen nach den Gleichgewichtslösungen in Abhängigkeit vom
Parameter $\lambda$.

Die nachfolgenden prototypischen Bifurkationen können in vielen
weiteren Differentialgleichungen beobachtet werden. 
Für alle Werte des Parameters ist $0$ ein kritischer Wert,
es gilt also $f(0,\lambda)=0$.
Wir können die $f$ in eine Taylor-Reihe 
\begin{equation}
f(x,\lambda)
=
\sum_{k,l} \frac{1}{(k+l)!}\frac{\partial^{k+l} f(0,0)}{\partial t^k\partial\lambda^l} x^k\lambda^l
=
a_{10}x + a_{01}\lambda
+
a_{20}x^2 + a_{11}x\lambda + a_{02}\lambda^2 + \dots
\end{equation}
entwickeln.
Die verschiedenen Bifurkationen lassen sich charakterisierung durch
die führenden Terme in dieser Entwicklung.
Insbesondere können wir verlangen, dass der führende Term für $\lambda$
immer linear in $\lambda$ sein soll.
Ist dies nämlich nicht der Fall, ist also der führende Term in 
$\lambda$ von der Form $\lambda^\alpha$, ersetzen wir den
Parameter einfach durch $\tau=\lambda^\alpha$.

\begin{beispiel}
\begin{figure}
\centering
\includegraphics{chapters/3/lin1.pdf}
\caption{
Phasendiagramm der Differentialgleichung $\dot x = x-\lambda$ links und
$\dot x = -x-\lambda$ rechts.
Die Gleichgewichtslösung $x=\lambda$ ist im linken Fall instabil,
während $x=-\lambda$ im rechten Fall eine stabile Gleichgewichtslösung ist.
\label{skript:dgl:phasen1}
}
\end{figure}%
Der einfachste Fall ist bis auf eine Skalierung
\begin{equation}
f(x,\lambda)=x-\lambda.
\label{skript:dgl:linear1}
\end{equation}
$f$ hat nur einen einzigen kritischen Punkt, nämlich $x_0=-\lambda$.
Das Phasendiagramm dafür ist in Abbildung~\ref{skript:dgl:phasen1}
Man erkennt, dass Läsungen, die bei $x$-Werten $x>\lambda$ beginnen,
anwachsen und sich von der Gleichgewichtslösung entfernen.
Umgekehrt nehmen Lösungen ab, die bei $x$-Werten $x<\lambda$ beginnen,
und entfernen sich damit ebenfalls von der Gleichgewichtslösung.
Die Differentialgleichung mit rechter Seite~\ref{skript:dgl:linear1}
hat kein stabile Lösungen.

Die analoge Analyse für die Differentialgleichung
\[
\frac{dx}{dt} = -x-\lambda
\]
ist in Abbildung~\ref{skript:dgl:linear1} dargestellt.
Die Gleichgewichtslösung $x_0=-\lambda$ ist stabil.
\end{beispiel}

Dieses Beispiel zeigt, dass interessante Bifurkationsereignisse
erst dann auftreten, wenn der führende Term in $x$ der Taylor-Entwicklung
von $f$ von höherer als linearer Ordnung ist.

\subsection{Sattel-Knoten-Bifurkation}
\begin{figure}
\centering
\includegraphics{chapters/3/saddle-node.pdf}
\caption{Phasendiagramm der Sattel-Knoten-Bifurkation zur
Differentialgleichung~\ref{skript:dgl:sattel-knoten-dgl}.
Für $\lambda>0$ gibt es zwei Gleichgewichtslösungen $\pm\sqrt{\lambda}$,
eine ist stabil, die andere instabil.
\label{skript:dgl:saddle-node}}
\end{figure}%
Die {\em Sattel-Knoten-Bifurkation}
\index{Sattel-Knoten-Bifurkation}%
tritt auf in der Differentialgleichung
\begin{equation}
\frac{dx}{dt}=x^2 - \lambda.
\label{skript:dgl:sattel-knoten-dgl}
\end{equation}
Für $\lambda >0$ hat die Gleichung~\ref{skript:dgl:sattel-knoten-dgl}
zwei Gleichgewichtslösungen $\pm\sqrt{\lambda}$.
In Abbildung~\ref{skript:dgl:saddle-node} ist das Phasendiagramm
dargestellt.
Daraus geht hervor, dass die Gleichgewichtslösung $q\sqrt{\lambda}$
instabil ist, während $-\sqrt{\lambda}$ stabil ist.

\subsection{Heugabel-Bifurkation}
\begin{figure}
\centering
\includegraphics{chapters/3/pitchfork.pdf}
\caption{Phasendiagramm der Heugabel-Bifurkation.
\label{skript:dgl:pitchfork}}
\end{figure}
Die {\em Heugabel-Bifurkation} tritt bei der Differentialgleichung
\begin{equation}
\frac{dx}{dt} = x^3 - \lambda x
\label{skript:dgl:heugabel-dgl}
\end{equation}
auf, bei der der führende Term
dritter Ordnung in $x$ ist.
Die Differentialgleichung
kann auch in der Form
\[
\frac{dx}{dt}
=
x(x^2-\lambda)
\]
geschrieben werden.
Sie hat $0$ als Gleichgewichtslösung für alle $\lambda$.
Für $\lambda>0$ hat sie zusätzlich die Gleichgewichtslösungen
$\pm\sqrt{\lambda}$.

Das Phasendiagramm~\ref{skript:dgl:heugabel-dgl} zeigt, dass
die einzige Gleichgewichtslösung bei $x_0=0$ instabil ist.
Beim Übergang zu $\lambda>0$ wird die Gleichgewichtslösung $x_0=0$
stabil.
Die beiden neuen Gleichgewichtslösungen $\pm\sqrt{\lambda}$ sind
beide instalbil.

Die Differentialgleichung
\[
\frac{dx}{dt} = -x^3+\lambda x
\]
hat die gleichen Gleichgewichtslösungen, jedoch ist $0$ für
$\lambda<0$ eine stabile Lösung, die beim Übergang zu $\lambda>0$
instabil wird.
Die Gleichgewichtslösungen $\pm\sqrt{\lambda}$ für $\lambda >0$
sind stabil.

\subsection{Transkritische Bifurkation}
\begin{figure}
\centering
\includegraphics{chapters/3/trans.pdf}
\caption{Phasendiagramm der transkritischen Bifurkation in der
Differentialgleichung~\eqref{skript:dgl:transkritisch-dgl}.
\label{skript:dgl:transfig}}
\end{figure}
Die Differentialgleichung
\begin{equation}
\frac{dx}{dt}
=
x^2 + \lambda x
=
x(x+\lambda)
\label{skript:dgl:transkritisch-dgl}
\end{equation}
hat Gleichgewichtslösung $0$ und $-\lambda$.
Das Phasendiagramm in Abbildung~\ref{skript:dgl:transfig} zeigt, dass 
für $\lambda<0$ die Gleichgewichtslösung $0$ stabil ist, die
Gleichgewichtslösung $-\lambda$ hingegen instabil.
Beim Übergang zu $\lambda>0$ wird die Gleichgewichtslösung $0$ instabil
und die Gleichgewichtslösung $-\lambda$ wird instabil.

\subsection{Ein Beispiel zur globalen Mitteltemperatur}
\begin{figure}
\includegraphics{chapters/3/kubisch.pdf}
\caption{Phasendiagramm der Differentialgleichung~\eqref{skript:dgl:kubisch}.
\label{skript:dgl:kubischfig}}
\end{figure}
Das Budyko-Modell versucht, die globale Mittelemperatur in Abhängigkeit
von der Einstrahlung und der Albedo zu modellieren.
Mehr zu diesen Ansätzen wird in Kapitel~5 dargestellt.

Als Beispiel für die Diskussion eines solchen Modells betrachten wir
die Differentialgleichung
\begin{equation}
\frac{dx}{dt}
=
x^3 - 4x - \lambda.
\label{skript:dgl:kubisch}
\end{equation}

\section{Stabilität}
\rhead{Stabilität}

