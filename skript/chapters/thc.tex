%
% thc.tex -- Termohaline Zirkulation
%
% (c) 2018 Prof Dr Andreas Müller
%
\chapter{Termohaline Zirkulation}
\lhead{Termohaline Zirkulation}
\begin{figure}
\centering
\includegraphics[width=\hsize]{chapters/4/1280px-Thermohaline_circulation.png}
\caption{
Das globale Förderband der thermohalinen Zirkulation.
\label{skript:thc:foerderband}}
\end{figure}%
Der Salzgehalt des Meerwassers ist nicht konstant, beinflusst aber
wie die Temperatur die Dichte.
Dies führt zu einer grossräumigen Zirkulationsströmung in den Weltmeeren,
genannt die thermohaline Zirkulation,
und damit zu einem weiteren bedeutenden Energietransportmechanismus.
Abbildung~\ref{skript:thc:foerderband} zeigt den Umfang der Zirkulation.
Auf einer Zeitskala von Jahrzehnten bis Jahrhunderten wird Meerwasser 
und damit auch Wärmeenergie über Distanzen umgewälzt, welche mehrfach die
Erde umspannen.
Die Organismen, die in den oberen Wasserschichten absterben, sinken langsam
auf den Meeresgrund.
Ohne eine umfassende Umwälzung der Weltmeere würden die oberen Wasserschichten
mit der Zeit an Nährstoffen verarmen.
Die thermohaline Zirkulation stellt also auch die Versorung der
Weltmeere mit Nährstoffen sicher.

Der Golfstrom ist ein kleiner Ausschnitt des globalen Förderbandes.
Die gut bekannte Bedeutung des Golfstroms für das europäische Klima 
deutet an, wie wichtig die thermohaline Zirkulation für das globale
Klima ist.
Es ist daher unerlässlich zu verstehen, was die Zirkulation antreibt und
wie sich der Klimawandel darauf auswirken könnte.

In diesem Kapitel soll die thermohaline Zirkulation modelliert werden.
Besonderes Augenmerk liegt dabei auf der Tatsache, dass dieses System
kippen kann.
Bei einer genügend grossen Änderung der Klimaparameter kann die Zirkulation
sich auf irreversible Art ändern.
Ein solches Ereignis hätte katastrophale Auswirkungen für das Klima.

%
% salinitaet.tex -- Salinität
%
% (c) 2018 Prof Dr Andreas Müller
%
\section{Salinität und Dichte}
\rhead{Salinität}
Der Salzgehalt des Meerwassers ist nicht konstant.
Er steigt an, wenn Wasser verdampf oder sich Eis bildet.
Er sinkt, wenn das Salz durch Niederschläge verdünnt wird.
Mit der Veränderung des Salzgehaltes geht auch eine Änderung
der Dichte einher.

Den genauen Zusammenhang zwischen Salinität, Temperatur und Dichte
kann nicht aus Naturgesetzen abgeleitet werden.
Verschiedene Untersuchungen haben empirische Formeln für die
Dichte in Abhängigkeit von Temperatur und Salinität zu Tage
gefördert.
Zum Beispiel  in der Form
\[
\varrho
=
\varrho_0(T)
+
A(T)\cdot S + B(T)\cdot S^{\frac32}+C\cdot S^2,
\]
wobei die Koeffizienten $A(T)$ und $B(T)$ Polynome der Temperatur $T$ in
$\mathstrut^\circ\text{C}$ sind:
\begin{align*}
A(T)
&=
 0.824493 - 0.0040899\,T + 0.000076438\,T^2 - 0.00000082467\,T^3 + 0.0000000053875\,T^4
\\
B(T)
&=
 -0.00572466 + 0.00010227\,T - 0.0000016546\,T^2
\\
C
&=
0.00048314
\\
\varrho_0(T)
&=
 0.824493 - 0.0040899\,T + 0.000076438\,T^2
 - 0.00000082467\,T^3 + 0.0000000053875\,T^4,
\end{align*}
die man etwa in
\cite{skript:millero}
findet.
Diese Formeln geben die Dichte über einen weiten Parameterbereich
mit einem relativen Fehler $<0.001$ wieder.
Für unsere qualitativen Überlegungen ist diese Genauigkeit
nicht nötig.

Im Folgenden betrachten wir die Dichte-Anomalie $\varrho-\varrho_0$
in Abhängigkeit von der Temperatur-Anomalie $T-T_0$ und der
Salinitäts-Anomalie $S-S_0$.
Man kann die Dichte-Anomalie immer als Taylor-Reihe
\[
\varrho 
=
\varrho_0
+
\frac{\partial \varrho}{\partial T}(T-T_0)
+
\frac{\partial \varrho}{\partial S}(S-S_0)
+
\text{Terme höherer Ordnung}.
\]
Solange $T-T_0$ und $S-S_0$ nicht allzu gross sind, kann man sich auf
die linearen Terme beschränken und
\begin{equation}
\varrho
=
\varrho_0(1-\alpha(T-T_0)+\beta(S-S_0))
\label{skript:salinity-linear}
\end{equation}
schreiben.
Darin sind $\alpha$ und $\beta$ positive Zahlen, die Vorzeichen in
\eqref{skript:salinity-linear} sind so gewählt, dass die Dichte
mit höherer Temperatur abnimmt und mit höherer Salinität zunimmt.
Typische Werte sind
\[
\alpha = 1.5\cdot 10^{-4}\frac{1}{\text{K}}
\qquad
\text{und}
\qquad
\beta = 8\cdot 10^{-4}\frac{1}{\text{psu}}.
\]

Mit diesem Modell für die Dichte könnten wir jetzt versuchen, die
Bewegungsgleichungen der Fluiddynamik und die Wärmeleitungsgleichung
zu lösen.
Wegen der komplizierten Form der Weltmeere ist das eine Aufgabe,
die ausschliesslich numerisch gelöst werden kann.
Ausserem benötigen wir detaillierte Informationen über den Wärmeaustsusch
mit der Atmosphäere oder dem Meeresboden.
Eine derart detaillierte Modellierung scheint daher aussichtslos.
Für eine qualitative Aussage über die Zirkulation benötigen wir daher
ein dramatisch vereinfachtes Modell.






%
% box.tex
%
% (c) 2018 Prof Dr Andreas Müller, Hochschule Rapperswil
%
\section{Box-Modelle}
\rhead{Box-Modelle}



%
% dimensionslos.tex
%
% (c) 2018 Prof Dr Andreas Müller, Hochschule Rapperswil
%
\section{Dynamik der thermohalinen Zirkulation}
\rhead{Dynamik der thermohalinen Zirkulation}

\subsection{Elimination von Prozessen mit kurzer Zeitkonstante}
Die Diskussion in Abschnitt~\ref{skript:thc:zeitkonstanten}
ist es zulässig, Variablen mit sehr kleiner Zeitkonstanten
durch Konstanten zu ersetzen.
Tatsächlich erfolgt der Temperaturausgleich im Wasser sehr viel
schneller als der Salinitätsausgleich.
Wir können daher davon ausgehen, dass die Temperaturgleichungen
die Temperaturanomalien sehr schnell gegen eine Gleichgewichtstemperatur
streben lassen, und dass wir für die Lösung der Salinitätsgleichungen
mit dieser konstanten Temperatur arbeiten können.

Wir gehen also davon aus, dass $\Delta\bar T=2T^*$ konstant ist und
reduzieren damit das Gleichungssystem
\eqref{skript:thc:differenzgleichungen}
auf die eine Gleichung
\begin{equation}
\frac{d}{dt}\Delta\bar S
=
2H + d(2S^* -\Delta\bar S) - 2|q|\Delta\bar S
\qquad\text{mit}\qquad
q=k(2\alpha T^* -\beta \Delta\bar S).
\label{skript:thc:salinitaetallein}
\end{equation}
Diese Gleichungen beschreiben also die Salinitätsentwicklung unter
der Annahme, dass der Temperaturausgleich sehr schnell erfolgt.
Dieser Ausgleich kann nicht primär durch Durchmischung erfolgen,
denn dieser Mechanimus würde auch die Salinität mit gleicher
vergleichbarer Geschwindigkeit ausgleichen.
Dies bedeutet, dass der dominante Term in der Temperaturgleichung
der Term mit $c$ ist, nicht der Term mit $q$.

Die Gleichung \eqref{skript:thc:salinitaetallein} kann noch nicht auf
einfache Weise gelöst werden.
Wir vereinfachen wir sie daher weiter indem wir ausnutzen, dass 
der Salinitätsausgleich so viel langsamer ist als der Temperaturausgleich,
dass der Term mit $d$ im Vergleich zum Term mit $q$ vernachlässigbar ist.
Wir setzen also $d=0$ und
erhalten damit 
\begin{equation}
\frac{d}{dt}\Delta\bar S
=
2H - 2|q|\Delta\bar S
\qquad\text{mit}\qquad
q=k(2\alpha T^* -\beta \Delta\bar S)
\label{skript:thc:qgleichung}
\end{equation}
als vereinfachte Differentialgleichung zur Modellierung der 
thermohalinen Zirkulation.
Dies ist eine nichtlineare Differentialgleichung erster Ordnung,
die nicht in geschlossener Form gelöst werden kann.

\subsection{Eine dimensionslose Beschreibung}
Die Gleichung \eqref{skript:thc:qgleichung} ist wegen der vielen
Konstanten unübersichtlich.
Ausgeschrieben lautet sie
\begin{equation}
\frac{d}{dt}\Delta\bar S
=
2H
-2k|\alpha\Delta\bar T- \beta \Delta\bar S|\,\Delta\bar S
\label{skript:thc:smitdim}
\end{equation}
Die meisten der Konstanten können wir aber los werden, indem wir 
die unabhängigen Variablen und die Zeit neu skalieren.
Dies ist gleichbedeutend mit einem Wechsel der Masseinheiten.
Wir verwenden:
\begin{equation}
x=\frac{\beta\Delta\bar S}{\alpha\Delta\bar T},
\qquad
\tau = 2\alpha k\,|\Delta\bar T|\, t
\qquad\text{und}\qquad
\lambda = \frac{\beta H}{\alpha^2 k\Delta\bar T|\Delta\bar T|}.
\label{skript:thc:masseinheiten}
\end{equation}
Die Ableitung nach $t$ kann durch die Ableitung nach $\tau$ ausgedrücket
werden vermöge der Ersetzung
\[
\frac{d}{d\tau}
=
\frac{1}{2\alpha k|\Delta\bar T|}
\frac{d}{dt}
\qquad\Rightarrow\qquad
\frac{d}{dt}
=
2\alpha k|\Delta\bar T|\frac{d}{d\tau}.
\]
Setzen wir dies in die Gleichung~\eqref{skript:thc:masseinheiten}
ein, erhalten wir
\begin{equation}
2\alpha k|\Delta\bar T|
\frac{d}{d\tau} \Delta\bar S
=
2H-2k|\alpha\Delta\bar T-\beta\Delta\bar S|\,\Delta\bar S.
\end{equation}
Wir erweitern mit $\beta/\alpha\Delta\bar T$, damit wird die
Differentialgleichung zu
\begin{align}
2\alpha k|\Delta\bar T|
\frac{d}{d\tau}\frac{\beta\Delta\bar S}{\alpha\Delta\bar T}
&=
\frac{2\beta H}{\alpha\Delta\bar T} - 2k|\alpha\Delta\bar T-\beta\Delta\bar S|\,
\frac{\beta\Delta\bar S}{\alpha\Delta\bar T}
\notag
\\
\alpha k|\Delta\bar T|
\frac{d}{d\tau}x
&=
\frac{\beta H}{\alpha\Delta\bar T}
-k|\alpha\Delta\bar T-\beta\Delta\bar S|\, x
\notag
\\
k
\frac{d}{d\tau}x
&=
\frac{\beta H}{\alpha^2\Delta\bar T\,|\Delta\bar T|}
-k\biggl|1-\frac{\beta\Delta\bar S}{\alpha\Delta\bar T}\biggr|\,x
\notag
\\
\frac{d}{d\tau}x
&=
\frac{\beta H}{k\alpha^2\Delta\bar T\,|\Delta\bar T|}
-\biggl|1-\frac{\beta\Delta\bar S}{\alpha\Delta\bar T}\biggr|\,x
\notag
\\
\frac{dx}{d\tau}
&=
\lambda - |1-x|x.
\label{skript:thc:dimensionslos}
\end{align}
Damit haben wir die ursprüngliche Gleichung
\eqref{skript:thc:smitdim}
in eine dimensionslose Gleichung mit dem einen Parameter $\lambda$
umgewandelt.
Das Verhalten der Lösung hängt vom Parameter $\lambda$ ab.

\subsection{Gleichgewicht}
Um das Verhalten der Lösungen von 
\eqref{skript:thc:dimensionslos}
besser zu verstehen, suchen wir zunächst nach Gleichgewichtslösungen.
Diese hängen nicht von der Zeit ab, es gilt also
\begin{equation}
\frac{dx}{d\tau}=0
\qquad\Rightarrow\qquad
\lambda-|1-x|x=0
\qquad\Rightarrow\qquad
|1-x|x
=
\lambda.
\label{skript:thc:lambdagl}
\end{equation}
\begin{figure}
\centering
\includegraphics{chapters/4/rhs.pdf}
\caption{Graph der Funktion $|1-x|\,x$ und
Gleichgewichtslösungen der dimensionslosen Differentialgleichung
\eqref{skript:thc:dimensionslos}
\label{skript:thc:1-xxgraph}}
\end{figure}
In Abbildung~\ref{skript:thc:1-xxgraph} ist der Graph der Funktion 
$|1-x|\,x$ dargestellt.
Je nach dem Wert von $\lambda$ hat die dimensionslose Differentialgleichung
\eqref{skript:thc:dimensionslos} bis zu drei Gleichgewichtslösungen.

Für Werte von $\lambda$ zwischen $0$ und $0.25$ gibt es drei verschiedene
Werte $x$, die die Gleichung \eqref{skript:thc:lambdagl} erfüllen.
Für $x\le 1$  ist $1-x\ge 0$ und damit muss $x$ die 
Gleichung $(1-x)x=\lambda$ erfüllen, für $x\ge 1$ ist es die Gleichung
$(x-1)x=\lambda$.
Diese beiden Gleichungen haben die folgenden Lösungen
\begin{align*}
&\text{Fall $x \le 1$:}
&
(1-x)x&=\lambda
&
&\text{Fall $x\ge 1$:}
&
(x-1)x&=\lambda
\\
&&
x^2-x+\lambda&=0
&&&
x^2-x-\lambda&=0
\\
&&
x&=\frac12\pm\sqrt{\frac14-\lambda}
&&&
x&=\frac12+\sqrt{\frac14+\lambda}.
\end{align*}
Die rechte Gleichung hat für alle Werte $\lambda > 0$ zwar auch noch
eine Lösung $<1$, diese ist aber ausgeschlossen, daher nur das
positive Zeichen vor der Wurzel in diesem Fall.
Für $\lambda > 0.25$ hat die linke Gleichung keine Lösung.
Für $\lambda < 0$ ist die Lösung mit dem positiven Zeichen der linken
Gleichung ausgeschlossen.

Wir berechnen die Lösung der Differentialgleichung für einen gegebenen
Wert von $\lambda$.
Für $\lambda >\frac14$ gibt es nur einen Gleichgewichtspunkt, wir 
nennen ihn $x_3$.
Falls $x(\tau)<x_3$, dann ist
\[
\frac{dx}{d\tau}
=
\lambda - |1-x|x > 0,
\]
die Lösung $x(\tau)$ ist also monoton wachsend.
Für $x(\tau) > x_3$ ist hingegen
\[
\frac{dx}{d\tau}
=
\lambda - |1-x|x < 0,
\]
die Lösung ist also monoton wachsen.
Lösungskurven, die bei $x$-Werten $>x_3$ beginnen nehmen monoton ab
und konvergieren gegen $x_3$, solche, die bei $x$-Werten $<x_3$ beginnen,
nehmen monoton zu und konvergieren von unten gegen $x_3$.
Die Gleichgewichtslösung $x(\tau)=x_3$ ist also eine stabile Lösung,
alle anderen Lösungen konvergieren gegen diese Lösung.


Für $0<\lambda<\frac14$ seien
$x_1$, $x_2$ und $x_3$ die drei Gleichgewichtspunkte.
Wir untersuchen wieder die Vorzeichen von $dx/d\tau$. 
Für $x$-Werten zwischen $x_1$ und $x_2$ und für $x$-Werte grösser
als $x_3$ ist die Ableitung positiv, die Lösungen konvergieren monoton
wachsend gegen die Gleichgewichtslösungen $x_1$ bzw.~$x_3$.
Lösungen, die bei $x<x_1$ oder $x_2<x<x_3$ beginnen, konvergieren
dagegen monoton wachsend gegen $x_1$ bzw.~$x_3$.
Die Gleichgewichtslösung $x_2$ ist daher nicht stabil,
die Gleichgewichtslösungen $x_1$ und $x_3$ sind dagegen stabil.

\begin{figure}
\centering
\includegraphics{chapters/4/drei.pdf}
\caption{Lösungen im Fall $0\le \lambda\le\frac14$: die beiden
Gleichgewichtspunkte $x_1$ und $x_3$ sind stabil, $x_2$ ist instabil.
\label{skript:thc:drei}}
\end{figure}

\begin{figure}
\centering
\includegraphics{chapters/4/ein.pdf}
\caption{Lösungen im Fall $\lambda > \frac14$: der einzige
Gleichgewichtspunkt $x_3$ ist stabil.
\label{skript:thc:ein}}
\end{figure}

\subsection{Bifurkation}
\begin{figure}
\centering
\includegraphics{chapters/4/bi.pdf}
\caption{Bifurkationsdiagramm für die Differentialgleichung
\eqref{skript:thc:lambdagl}
\label{skript:thc:bifurkation}}
\end{figure}








