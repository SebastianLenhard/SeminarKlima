%
% hydrodynamik.texo
%
% (c) 2018 Prof Dr Andreas Müller, Hochschule Rapperswil
%
\section{Fluiddynamik}
\rhead{Fluiddynamik}
%Die Atmosphäre und die Ozeane unterschieden sich in ihren
%für das Studium von Wetter und Klima wesentlichen Eigenschaften
%ganz beträchtlich.
%Das Wasser der Ozeane ist fast inkompressibel, seine Dichte hängt aber
%von der Temperatur und dem Salzgehalt ab.
%Wasser hat eine sehr grosse Wärmekapazität, ausserdem kann Wärme durch
%Verdunstung aus den Ozeanen in die Atmosphäre übergehen, wobei gleichzeigit
%die Salzkonzentration steigt.
%
%Die Atmosphäre auf der anderen Seite hat eine wesentlich geringere
%Dichte und Wärmekapazität, ihre Temperatur kann sich daher sehr viel
%schneller ändern.
%Sie ist stark kompressibel.
%Wegen der geringeren Dichte kann die Atmosphäre sehr viel höhere
%Strömungsgeschwindigkeiten erreichen.
%
%Trotz dieser grossen Unterschiede lassen sich Atmosphäre und Ozeane
%beide als Fluide mit den gleichen partiellen Differentialgleichungen
%beschreiben, die im folgenden hergeleitet werden sollen.
%Die Unterschiede äussern sich vor allem in den Zustandsgleichungen,
%die die Zustandsgrössen Druck, Temepratur, Dichte und Saltzgehalt
%miteinander in Beziehung setzen.
%
In diesem Abschnitt gehen wir davon aus, dass das Fluid beschrieben wird
durch Funktionen der Raumkoordinaten $(x,y,z)$ und der Zeit $t$,
wobei wir meistens darauf verzichten, die unabhängigen Variablen
auszuschreiben.
Die Temperatur $T$ ist also zu lesen als die Funktion $T(x,y,z,t)$.
Die Newtonschen Bewegungsgleichungen stellen eine Verbindung zwischen
Masse, Beschleunigung und Kraft her, wir können daher davon ausgehen,
dass die Bewegungsgleichungen eines Fluides nur die Dichte $\varrho$ und 
den Geschwindigkeitsvektor $\vec{v}$ involvieren.
Den Zusammenhang zwischen Druck, Temperatur, Dichte und möglicherweise
weiteren Eigenschaften wird durch Zustandsgleichungen vermittelt.

\subsection{Kontinuitätsgleichung}
Die Kontinuitätsgleichung drückt aus, dass
Materie nicht einfach neu entstehen oder verschwinden kann.
Um sie herzuleiten, betrachten wir ein Volumen $V$ des Fluids.
Die Masse im Inneren des Volumens wird bestimmt durch das Volumenintegral
\[
m
=
\iiint_V \varrho \,dx\,dy\,dz.
\]
Ein kleiner Quader mit den Abmessungen $\Delta x$, $\Delta y$
und $\Delta z$ enthält die Masse 
\[
m = \varrho \Delta x \,\Delta y \, \Delta z.
\]
Wenn sich die Masse in dem Quader ändert, dann muss Materie durch die
Wände zu- oder abfliessen.
Wir berechnen daher für jede Wand des Quaders, wie gross der Massefluss
durch die Wand in einer Zeiteinheit $\Delta t$ ist.

Durch eine Rechteck mit Abmessungen $\Delta y \times \Delta z$ senkrecht
zur $x$-Achse fliesst in der Zeit $\Delta t$ die das Volumen
$v_x\Delta x\,\Delta y\,\Delta z$ und damit die Masse
\begin{equation}
\varrho v_x\,\Delta y\,\Delta z.
\label{skript:massenausdruck}
\end{equation}
Die Dichte $\varrho$ und die Geschwindigkeit $v_x$ sind dabei an der
Koordinate $x$ zu nehmen.
Durch die Wand des Quaders bei $x+\Delta x$ fliesst eine Masse, die
ebenfalls durch den Ausdruck \eqref{skript:massenausdruck}
beschrieben werden kann, jedoch für die $x$-Koordinaten $x+\Delta x$.
Um die Massenänderung im Quader zu bestimmen, sind diese beiden Ausdrücke
als mit entgegengesetzten Vorzeichen zu berücksichtigen.

Die Massenänderung ist daher
\begin{align}
\Delta m
&=
\varrho(x,y,z,t) v_x(x,y,z,t)\,\Delta y\,\Delta z\,\Delta t
-
\varrho(x+\Delta x,y,z,t) v_x(x+\Delta x,y,z,t)\,\Delta y\,\Delta z\,\Delta t
\notag
\\
&\quad
+
\varrho(x,y,z,t) v_y(x,y,z,t)\,\Delta x\,\Delta z\,\Delta t
-
\varrho(x,y+\Delta y,z,t) v_y(x,y+\Delta y,z,t)\,\Delta x\,\Delta z\,\Delta t
\notag
\\
&\quad
+
\varrho(x,y,z,t) v_z(x,y,z,t)\,\Delta x\,\Delta y\,\Delta t
-
\varrho(x,y,z+\Delta z,t) v_z(x,y,z+\Delta z,t)\,\Delta x\,\Delta y\,\Delta t.
\notag
\intertext{Wir fassen die Terme zu gegenüberliegenden Wänden zusammen wobei
wird das Produkt $\Delta x\,\Delta y\,\Delta z$ ausklammern können.
Wir teilen ausserdem durch $\Delta t$, um die zeitliche Massenänderungsrate
zu erhalten.}
\frac{\Delta m}{\Delta t}
&=
-
\bigg(
\frac{\varrho(x+\Delta x,y,z,t)v_x(x+\Delta x,y,z,t)-\varrho(x,y,z,t)v_x(x,y,z,t)}{\Delta x}
\notag
\\
&\qquad
+\frac{\varrho(x,y+\Delta y,z,t)v_y(x,y+\Delta y,z,t)-\varrho(x,y,z,t)v_y(x,y,z,t)}{\Delta y}
\notag
\\
&\qquad
+\frac{\varrho(x,y,z+\Delta z,t)v_y(x,y,z+\Delta z,t)-\varrho(x,y,z,t)v_y(x,y,z,t)}{\Delta z}
\bigg)
\Delta x\,\Delta y\,\Delta z\,\Delta t.
\notag
\intertext{Da $\Delta m=\varrho\Delta x\,\Delta y\,\Delta z$ können wir
auf beiden Seiten durch $\Delta x\,\Delta y\,\Delta z$ dividieren.
Um die zeitliche Änderung zu bestimmen, müssen wir ausserdem durch
$\Delta t$ dividieren.
Lassen wir die Inkremente $\Delta x$, $\Delta y$, $\Delta z$ und
$\Delta t$ gegen $0$ gehen, werden aus den Differenzenquotienten
Ableitungen.
Wir erhalten daher die {\em Kontinuitätsgleichung}
\index{Kontinuitätsgleichung} }
\frac{\partial \varrho}{\partial t}
&=
-
\biggl(
\frac{\partial \varrho v_x}{\partial x}
+
\frac{\partial \varrho v_y}{\partial y}
+
\frac{\partial \varrho v_z}{\partial z}
\biggr).
\label{skript:kontinuitaetsgleichung}
\end{align}
\index{Kontinuitätsgleichung}%
Die rechte Seite kann mit Hilfe des {\em Nabla-Operators}
\index{Nabla-Operator}
\[
\nabla
=
\begin{pmatrix}
\frac{\partial}{\partial x}\\
\frac{\partial}{\partial y}\\
\frac{\partial}{\partial z}
\end{pmatrix}
\]
kürzer geschrieben werden.
Der Nabla-Operator wird wie ein Vektor behandelt.
Für eine (skalare) Funktion $f$ ist $\nabla f$ ein Vektor,
der {\em Gradient}
\index{Gradient} der Funktion $f$.
Das Skalarprodukt $\nabla\cdot\vec{v}$ ist ein Skalar, die
{\em Divergenz}
\index{Divergenz}
eines Vektorfeldes $\vec{v}$, sie wird manchmal auch 
$\operatorname{div}\vec{v}$ geschrieben.
Aus 
\eqref{skript:kontinuitaetsgleichung}
wird dann
\[
\frac{\partial \varrho}{\partial t}
=
-\nabla\cdot (\varrho\vec{v})
\]
geschrieben werden.
So erhält die Kontinuitätsgleichung die kompakte Form
\[
\frac{\partial}{\partial t}\varrho = -\nabla\cdot (\varrho\vec{v}).
\]

\subsection{Inkompressible Strömung}
Bei einem inkompressiblen Fluid ist die Dichte eine Konstante, alle
\index{Fluid!inkompressibel}
\index{inkompressibel}
Ableitungen von $\varrho$ verschwinden.
Die Kontinuitätsgleichung wir damit zu
\[
\frac{\partial\varrho}{\partial t}
=
-\nabla\cdot(\varrho\vec{v})
=
-\nabla\varrho\cdot\vec{v}
-\varrho\nabla\vec{v}
=
-\varrho\nabla\vec{v}
=
0.
\]
In einer inkompressiblen Strömung verschwindet daher die Divergenz
des Geschwindigkeitsfeldes.

\subsubsection{Verallgemeinerung}
Die Herleitung der Kontinuitätsgleichung für die Massedichte funktioniert
auch für jede andere Erhaltungsgrösse, die im Fluid mit einer Dichte
$a(x,y,z,t)$ vorhanden ist und mit der Strömung mittransportiert wird.
Die {\em verallgemeinerte Kontinuitätsgleichung} für die Erhaltungsgrösse $a$
\index{Kontinuitätsgleichung!verallgemeinerte}
ist daher
\begin{equation}
\frac{\partial a}{\partial t}
=
-
\nabla(a\vec{v}).
\label{skript:verallgemeinerte kontinuitaetsgleichung}
\end{equation}

\subsection{Bewegungsgleichung}
Das zweite Newtonsche Gesetz $F=ma$ besagt, dass Kraft und Beschleunigung
proportional sind.
Dies gilt jedoch nur, wenn die Masse unveränderlich ist.
Da ein Volumen des Fluides wegen veränderlicher Dichte jedoch seine
Masse verändern kann, müssen wir verwenden, dass die Kraft die zeitliche
Änderung des Impulses ist.

\subsubsection{Impulsdichte}
Die Impulsdichte des Fluids wird an jeder Stelle durch die Grösse
$\vec{p}=\varrho\vec{v}$ gegeben.
Das zweite Newtonsche Gesetz besagt dann, dass die Änderung von $\vec p$
durch die äusseren Kräfte $\vec{b}$ bestimmt wird, die auf das Fluid wirkt.
Der Impuls ein einem Volumen kann aber auch ändern, dass das Fluid Impuls
in das Volumen hinein- oder aus dem Volumen heraustransportiert.
Jede Komponente des Impulses ist eine Erhaltungsgrösse, für die ohne
Wirkung äusserer Kräfte die verallgemeinerte Kontinuitätsgleichung
\eqref{skript:verallgemeinerte kontinuitaetsgleichung}
gilt.
Für die $x$-Komponente des Impulses gilt daher die Gleichung
\[
\frac{\partial \varrho v_x}{\partial t}
=
-\nabla \cdot(\varrho v_x\,\vec{v})
+\varrho b_x,
\]
und analog für die anderen Komponenten $\varrho v_y$ und $\varrho v_z$ 
der Impulsdichte.

\subsubsection{Innere Kräfte}
Damit sind aber innere Kräfte im Fluid noch nicht berücksichtigt.
Das Fluid widersetzt sich zum Beispiel der Kompression, dies äussert
sich im Druck, der jeweils senkrecht auf den Wänden des Volumens wirkt.
In einem zähen Medium sind aber auch Kräfte parallel zu den Wänden
\index{Zähigkeit}
möglich, sogenannte {\em Scherkräfte}.
\index{Scherkraft}
Im Allgemeinen wirkt auf ein $\Delta y\times\Delta z$-Rechteck senkrecht
zur $x$-Achse die Kraft
\[
\vec{\tau}_x
\,\Delta y\,\Delta z
=
\begin{pmatrix}
\tau_{xx}\\
\tau_{xy}\\
\tau_{xz}
\end{pmatrix}
\,\Delta y\,\Delta z
\]
und analog für die Wände senkrecht auf der $y$- bzw.~$z$-Achse.
Die diagonalen Komponente $\tau_{ii}$ beschreiben die Druckkraft
\index{Druck}
auf die jeweilige Seitenfläche, während die ausserdiagonalen Elemente
Scherkräfte beschreiben.

Die Matrix $\bm{\tau}$ mit Komponenten $\tau_{ij}$ heisst auch der
{\em Cauchy-Spannungstensor}.
\index{Cauchy-Spannungstensor}
\index{Spannungstensor}
Wir werden weiter unten (Seite~\pageref{skript:spannungstensor symmetrisch})
zeigen, dass $\tau_{ij}$ symmetrisch sein muss,
Dass $\tau_{ij}$ ein Tensor ist, ist für die weiteren Erörterungen nicht
von Bedeutung, wir werden daher diesen Begriff verwenden, ohne ihn wirklich
zu definieren.

Die resultierende Kraft $\vec{F}$ auf einen Quader mit den Kantenlängen
$\Delta x$, $\Delta y$ und $\Delta z$  hat daher die $i$-Komponente
\begin{align*}
F_x
&=
(
\tau_{xx}(x+\Delta x,y,z,t)
-
\tau_{xx}(x,y,z,t)
) \Delta y\,\Delta z
\\
&\qquad
+
(
\tau_{yx}(x,y+\Delta y,z,t)
-
\tau_{yx}(x,y,z,t)
) \Delta x\,\Delta z
\\
&\qquad
+
(
\tau_{zx}(x,y,z+\Delta z,t)
-
\tau_{zx}(x,y,z,t)
)\Delta x\,\Delta z
\\
&=
\bigg(
\frac{
\tau_{xx}(x+\Delta x,y,z,t)
-
\tau_{xx}(x,y,z,t)
}{\Delta x}
+
\frac{
\tau_{yx}(x,y+\Delta y,z,t)
-
\tau_{yx}(x,y,z,t)
}{\Delta y}
\\
&\qquad
+
\frac{
\tau_{zx}(x,y,z+\Delta z,t)
-
\tau_{zx}(x,y,z,t)
}{\Delta z}
\bigg)
\Delta x\,\Delta y\,\Delta z.
\end{align*}
Die Kraftdichte $f_i$ erhalten wir nach Division durch
$\Delta x\,\Delta y\,\Delta z$ und Grenzübergang, sie ist
\begin{equation}
f_x
=
\frac{\partial \tau_{xx}}{\partial x}
+
\frac{\partial \tau_{yx}}{\partial y}
+
\frac{\partial \tau_{zx}}{\partial z}.
\label{skript:spannungskraftdichte}
\end{equation}
Wir können damit die vollständige Bewegungsgleichung für das Fluid
hinschreiben, sie lautet
\begin{equation}
\frac{\partial \varrho v_x}{\partial t}
=
-\nabla\cdot (\varrho v_x\vec{v})
+
\varrho b_x
+
f_x.
\label{skript:navier-stokes komponente}
\end{equation}
\subsubsection{Vektorschreibweise}
Die Schreibweise
\eqref{skript:navier-stokes komponente}
für die Bewegungsgleichungen ist sehr schwerfällig und passt nicht
zu der deutlich elegantere vektoriellen Schreibweise zum Beispiel
der Kontinuitätsgleichung.
Die linke Seite von
\eqref{skript:navier-stokes komponente}
und der mittlere Term auf der rechten Seite können natürlich sofort
in eine vektorielle Schreibweise überführt werden, nicht jedoch die
anderen zwei Terme.

Der Term $\nabla \cdot(\varrho v_x\vec{v})$ ist ausgeschrieben
\[
\nabla \cdot(\varrho v_x\vec{v})
=
\frac{\partial}{\partial x}
(\varrho v_xv_x)
+
\frac{\partial}{\partial y}
(\varrho v_xv_y)
+
\frac{\partial}{\partial z}
(\varrho v_xv_z).
\]
Dieser Ausdruck sieht ganz ähnlich aus wie der Ausdruck
\eqref{skript:spannungskraftdichte}
für die $x$-Kompontente der Kraftdichte der inneren Kräfte.
Wir können die Ähnlichkeit formal noch etwas klarer machen.
Schreiben wir $A_{xy} = \varrho v_xv_y$, dann ist
\[
\nabla \cdot(\varrho v_x\vec{v})
=
\frac{\partial A_{xx}}{\partial x}
+
\frac{\partial A_{xy}}{\partial y}
+
\frac{\partial A_{xz}}{\partial z}
=
\sum_{i}\frac{\partial A_{xi}}{\partial i}.
\]
Da es offenbar auf die Reihenfolge der Indizes von $A$ nicht ankommt,
ist dies auch das gleiche wie
\[
\nabla \cdot(\varrho v_x\vec{v})
=
\frac{\partial A_{xx}}{\partial x}
+
\frac{\partial A_{yx}}{\partial y}
+
\frac{\partial A_{zx}}{\partial z}
=
\sum_{i}\frac{\partial A_{ix}}{\partial i}.
\]
Wir können daher die Wirkung des Nabla-Operators $\nabla$ auf einer
symmetrischen Matrix $A$ wie folgt definieren:

\begin{definition}
\label{skript:definition divergenz}
\index{Divergenz!einer Matrix}
Ist $A_{ij}$ eine symmetrische Matrix, dann ist die {\em Divergenz}
$\nabla\cdot A$
von
$A$ der Vektor mit den Komponenten
\[
(\nabla\cdot A)_x
=
\sum_{i}\frac{\partial A_{ix}}{\partial i}.
\]
\end{definition}
Falls die Matrix $\tau_{ij}$ symmetrisch ist, kann diese Definition
auch auf $\bm{\tau}$ angewendet werden.
Die $x$-Komponente der Divergenz von $\bm{\tau}$ ist dann
\[
(\nabla\cdot \bm{\tau})_x
=
\frac{\partial \tau_{xx}}{\partial x}
+
\frac{\partial \tau_{yx}}{\partial y}
+
\frac{\partial \tau_{zx}}{\partial z}
=
f_x.
\]
Dies ist genau der letzte Term in der Gleichung
\eqref{skript:navier-stokes komponente}.

Wir brauchen jetzt nur noch eine kompaktere Notation für die Matrix
$\varrho v_xv_y$.

\begin{definition}
\index{Kronecker-Produkt}
Das {\em Kronecker-Produkt} zweier Vektoren $\vec{a}$ und $\vec{b}$ ist die
Matrix $\vec{a}\otimes\vec{b}=\vec{a}\vec{b}^t$ mit den Komponenten
\[
(\vec{a}\otimes\vec{b})_{ij}
=
a_ib_j
=
(
\vec{a}
\vec{b}^t
)_{ij}
\]
Abgekürzt erlauben wir die Schreibweise $\vec{a}\otimes\vec{b}=\vec{a}\vec{b}$.
\end{definition}

Mit diesen Notationen bekommen wir jetzt die Bewegungsgleichungen in
Vektorform.
Sie lauten
\begin{equation}
\frac{\partial \varrho\vec v}{\partial t}
=
-\nabla\cdot(\varrho\vec{v}\vec{v})
+ \varrho\vec{b}
+ \nabla\cdot \bm{\tau}.
\label{skript:navier-stokes1}
\end{equation}
Dies ist die {\em Navier-Stokes Gleichung}.
\index{Navier-Stokes Gleichung}
Die drei Terme beschreiben die Impulsänderung durch den Zu- oder Abtransport
von Impuls durch die Strömung, durch die äusseren Kräfte bzw.~die inneren
Spannungen.

\subsubsection{Symmetrie des Spannungstensors}
\label{skript:spannungstensor symmetrisch}
\begin{figure}
\centering
\includegraphics{chapters/2/drehmoment.pdf}
\caption{Drehmoment um die $z$-Achse der Schwerkräfte auf einen Würfel
mit Kantenlänge $2l$. Gezeigt sind nur die Komponenten von $\bm{\tau}$,
die zu einem Drehmoment führen.
\label{skript:drehmoment}}
\end{figure}
In diesem Abschnitt wollen wir nachweisen, dass der Spannungstensor
symmetrisch ist.
Dazu betrachten wir das Drehmoment, welches die Scherkräfte auf einen
kleinen Würfel mit Kantenlänge $2l$ ausüben (Abbildung \ref{skript:drehmoment}).

Der Würfel hat die Masse $m=\varrho(2l)^3$.
Das Trägheitsmoment eines Würfels mit Masse $m$ und Kantenlänge $2l$
ist
\[
I_z
=
\frac1{12}m((2l)^2+(2l)^2)
=
\frac1{12}\varrho 8l^3\cdot8l^2
=
\frac{16}{3}\varrho l^5.
\]
Der Drehimpuls um die $z$-Achse ist $L_z=I_z\omega$.

Aus Abbildung~\ref{skript:drehmoment} kann man die Scherkräfte auf den
Seitenflächen ablesen, sie sind $\tau_{xy}4l^2$ bzw.~$\tau_{yx}4l^2$,
ihr Hebelarm ist $l$.
Das resultierende Drehmoment um die $z$-Achse ist daher
\[
M_z = 8l^3\tau_{xy} - 8l^3\tau_{yx}.
\]
Die Bewegungsgleichungen eines starren Körpers besagen jetzt, dass
für die Winkelgeschwindigkeit der Drehung des Würfels um die $z$-Achse
die Gleichung
\[
\frac{dL_z}{dt}
=
M_z
\qquad\Rightarrow\qquad
I_z\dot\omega
=
M_z
\qquad\Rightarrow\qquad
\dot\omega
=
\frac{M_z}{I_z}
=
\frac{8l^3(\tau_{xy}-\tau_{yx})}{\frac{16}{3}l^5}
=
\frac{3}{2l^2}(\tau_{xy}-\tau_{yx}).
\]
Wir nehmen an, es sei $\tau_{xy}\ne\tau_{yx}$.
Lässt man $l$ gegen $0$ gehen, folgt die Aussage, dass die
Winkelgeschwindigkeit eines sehr kleinen Würfels im Fluid sich mit beliebig
schnell anwachsender Winkelgeschwindigkeit drehen müsste.
Dieses unphysikalische Resultat erlaubt zu schliessen, dass
$\tau_{xy}=\tau_{yx}$ sein muss und und dass nur ein
symmetrischer Spannungstensor ein physikalisches Fluid beschreibt.

\subsubsection{Druck und Spannungen}
Die Diagonalelemente des Spannungstensors $\bm{\tau}$ beschreiben
Normalkräfte auf ein Volumenelement des Fluids.
Im Gleichgewicht sind sie alle gleich gross und stimmen mit dem
negativen {\em (hydrostatischen) Druck} überein, wir setzen daher
\index{Druck}
\index{hydrostatischer Druck}
\[
p=-\frac13\operatorname{Spur}\bm{\tau}.
\]
Wir können daher $\bm{\tau}$ zerlegen in eine Diagonalmatrix
mit Elementen $-p$ auf der Diagonalen und eine spurlose Matrix
\[
\bm{\tau} = -pE + \bm{\sigma},
\]
$E$ ist die Einheitsmatrix.
Die spurlose symmetrische Matrix $\bm{\sigma}$ heisst auch
{\em Spannungsdeviator}.
\index{Spannungsdeviator}

Für die Bewegungsgleichung brauchen wir die Divergenz beider Terme.
Die Druckterme sind alle gleich, nach
Definition~\ref{skript:definition divergenz} ist
\[
(\nabla\cdot(pE))_x
=
\sum_i
\frac{\partial p\delta_{xi}}{\partial i}
=
\frac{\partial p}{\partial x}
\qquad\Rightarrow\qquad
\nabla\cdot(pE)
=
\nabla p.
\]
Damit wird die Bewegungsgleichung 
\begin{equation}
\frac{\partial \varrho\vec{v}}{\partial t}
=
-\nabla\cdot(\varrho\vec{v}\vec{v})
+\varrho\vec b
-\nabla p
+\nabla\cdot\bm{\sigma}
\label{skript:navier-stokes2}
\end{equation}
Die Scherkräfte sind in einem newtonschen Fluid proportional zu
den Schergeschwindigkeiten.
Man kann zeigen (siehe \cite[p.~172]{skript:kaperengler}), dass $\bm{\sigma}$
geschrieben werden kann als
\[
\bm{\sigma}
=
2\nu\biggl(\bm{\varepsilon} - \frac13(\nabla\cdot\vec{v})E\biggr)
\qquad\text{mit}\qquad
\bm{\varepsilon}=\frac12\bigl(\nabla\vec{v}+(\nabla\vec{v})^t\bigr).
\]
Die spezielle Form von $\bm{\varepsilon}$ ist notwendig, damit die Matrix
$\bm{\varepsilon}$ symmetrisch wird.
Der zweite Term im Ausdruck von $\bm{\sigma}$ ist nötig, damit die Spur
\[
\operatorname{Spur}{\bm{\sigma}}
=
2\nu(\operatorname{\varepsilon} - \nabla\cdot\vec{v})
=
2\nu
\frac12
\biggl(
\sum_i \frac{\partial v_i}{\partial i}
+
\frac12
\sum_i \frac{\partial v_i}{\partial i}
-
\nabla\cdot\vec{v}
\biggr)
=
0
\]
von $\bm{\sigma}$ verschwindet.

Die Divergenz $\nabla\cdot\bm{\sigma}$ von $\bm{\sigma}$ kann damit explizit
durch die Geschwindigkeit ausgedrückt werden.
Wir berechnen die Divergenz der einzelnen Terme:
\begin{align}
(\nabla\cdot\bm{\varepsilon})_x
&=
\sum_i\frac{\partial \varepsilon_{ix}}{\partial i}
=
\frac12
\sum_i\frac{\partial}{\partial i}\biggl(
\frac{\partial v_i}{\partial x}+\frac{\partial v_x}{\partial i}
\biggr)
=
\frac{\partial}{\partial x}
\sum_i\frac{\partial v_i}{\partial i}
+
\frac12
\sum_i \frac{\partial^2 v_x}{\partial i^2}
=
\frac12\frac{\partial}{\partial x}
(\nabla\cdot\vec{v})
+
\frac12\Delta v_x
\notag
\\
\nabla\cdot\bm{\varepsilon}
&=
\frac12\nabla(\nabla\cdot\vec{v})
+
\frac12\Delta\vec{v}
\notag
\\
(\nabla\cdot(\nabla\cdot\vec{v})E)_x
&=
\sum_i \frac{\partial}{\partial i} (\nabla\cdot\vec{v}E)_{xi}
=
\sum_i \frac{\partial}{\partial i} (\nabla\cdot\vec{v}\delta_{xi})
=
\frac{\partial}{\partial x}(\nabla\cdot\vec{v})
\notag
\\
\nabla\cdot(\nabla\cdot\vec{v})E)
&=
\nabla(\nabla\cdot\vec v)
\notag
\\
\intertext{und erhalten so für die Divergenz von $\bm{\sigma}$:}
\nabla\cdot\bm{\sigma}
&=
2\nu\biggl(
\nabla\cdot\bm{\varepsilon}
-\frac13\nabla\cdot((\nabla\cdot\vec{v})E)
\biggr)
=
2\nu\biggl(
\frac12
\nabla(\nabla\cdot\vec{v})
+
\frac12\Delta\vec{v}
-\frac13
\nabla(\nabla\cdot\vec{v})
\biggr)
\\
&=
\nu\Delta\vec{v}
+\frac{\nu}3\nabla(\nabla\cdot\vec{v}).
\label{skript:sigmadiv}
\end{align}

\subsubsection{Inkompressible Strömung}
In einem inkompressiblen Fluid ist $\nabla\cdot\vec{v}=0$, dann fällt
der zweite Term in \eqref{skript:sigmadiv} weg.
Die Strömungsgleichung eines inkompressiblen Fluids erhält damit die
einfache Form
\begin{equation}
\frac{\partial\vec{v}}{\partial t}
=
-\nabla\cdot(\vec{v}\vec{v})
+\vec{b}
-\frac1{\varrho}(\nabla p
-\nu\Delta\vec{v}),
\label{skript:inkompressibel newtonsch}
\end{equation}
die klassische Navier-Stokes Gleichung.

\subsection{Zustandsgleichungen}
Die Dichte hängt vor allem auch von der Temperatur ab.
In den Ozeanen ändert die Dichte des Wassers mit dem Salzgehalt.
Eine vollständige Beschreibung der Strömung in Ozeanen oder der
Atmosphäre muss daher auch noch weitere Variablen modellieren.
In Kapitel~\ref{chapter:wetter und klima} haben wir bereits auf die
Wärmeleitungsgleichungen hingewiesen.

Die Felder $T$, $p$ und $\varrho$ sind bei einem idealen Gas miteinander
durch die Zustandsgleichung
\[
p=\varrho T R_s
\]
mit der spezifischen Gaskonstante $R_s$ verbunden.
Für den Zusammenhang von Dichte, Temperatur und Salzgehalt gibt
es jedoch kein derart einfaches Modell.
Eine weitere Kopplung zwischen der Temepratur und der
Strömung entsteht durch die Viskosität $\nu$, die sehr stark
von der Temperatur abhängt.
Auch dafür gibt es keine einfachen Modell.

In vielen Fällen schwanken die physikalischen Grössen nur geringfügig
um einen Mittelwert.
Zum Beispiel hängt die Dichte $\varrho$ von Meerwasser sowohl von
der Temperatur $T$ als auch vom Salzgehalt $h$ ab, die Dichte ist
also eine Funktion $\varrho(T,h)$.
Wir können $\varrho$ als Taylorreihe um die mittlere Temperatur $T_0$
und den mittleren Salzgehalt $h_0$ entwickeln:
\[
\varrho(T,h)
=
\varrho_0 -\alpha(T-T_0) + \beta(h-h_0).
\]
In Klimamodellen betrachten wir typischerweise nur kleine Abweichungen
von Mittelwerten, so dass ein solches Modell sehr erfolgreich sein kann.

\subsection{Boussinesq-Approximation}
Die Strömung in der Erdatmosphäre kann offensichtlich nicht als
inkompressibel betrachtet werden, die Dichte ist offenbar nicht
konstant.
Der Zustand der Atmosphäre weicht jedoch nur wenig einem mittleren
Dichteprofil $\varrho_0$ ab, welches im wesentlich durch das Temperaturprofil
festgelegt ist.
Im Normalzustand nimmt die Temperatur der Atmosphäre ziemlich genau
linear ab bis zur Höhe der Thermopause.
Auf die horizontale Komponente der Strömung hat eine Abweichung des
Temperaturprofils kaum einen Einfluss, denn andere Terme der
Navier-Stokes-Gleichung
\eqref{skript:navier-stokes2}
sind bedeutender.
Für die vertikale Bewegung ist der Term der äusseren Kräfte,
nämlich die Schwerkraft, dominant.
Wir können dies berücksichtigen, indem wir die Erdbeschleunigung
$g$ durch
\begin{equation}
g\frac{\varrho}{\varrho_0}
\label{skript:boussinesq}
\end{equation}
ersetzen.
Diese Approximation ist bekannt als die Boussinesq-Approximation.
Für unsere Zwecke hier brauchen wir nicht mehr als \eqref{skript:boussinesq}.
Dies wird bei der Herleitung der Lorenz-Gleichung in Abschnitt
\ref{section:lorenz-modell} benötigt.
Für die vollständigen Boussinesq-Gleichungen siehe \cite{skript:kaperengler}.

%In der Navier-Stokes-Gleichung können wir die horizontale Bewegung
%von der vertikalen vollständig trennen.
%Wir schreiben dazu
%\[
%\vec{u}
%=
%\begin{pmatrix}v_x\\v_y\end{pmatrix},
%\qquad
%w=v_z.
%\]
%Ausser in den für den Auftrieb wesentlichen Termen (Druck und äussere Kräfte)
%betrachten wir die Dichte als konstant.
%Die Kontinuitätsgleichung wird daher zu
%\[
%\nabla\cdot\vec{u}\frac{\partial w}{\partial z} = 0.
%\]
%Dies besagt, dass die Quellen des horizontalen Strömungsfeldes 
%durch die Vertikalströmung gespeist werden.
%Die horizontale Bewegungsgleichung bleibt bis auf den Druckterm
%erhalten.

\subsection{Inkompressible zweidimensionale Strömung}
Die Kontinuitätsgleichung
und die Navier-Stokes-Gleichung gelten auch für eine zweidimensionale
Strömung.
Im Allgmeinen ist die Strömung nicht wesentlich leichter zu berechnen.
Nur im Falle einer inkompressiblen Strömung oder der Boussinesq-Approximation
spielt die Dichte in den Gleichungen keine Rolle, was erlaubt,
sie weiter zu vereinfachen:
\begin{align}
0
&=
-\nabla\cdot\vec v
\label{skript:2dim kontinuitaet}
\\
\frac{\partial \vec{v}}{\partial t}
&=
-\nabla\cdot(\vec{v}\vec{v})+\vec b + \frac1{\varrho}\nabla\cdot\bm{\tau}.
\label{skript:2dim navier-stokes}
\end{align}
Im Falle der Boussinesq-Approximation kommt auf der rechten Seite noch
ein Term für die Auftriebskraft hinzu.

Die Gleichungen
\eqref{skript:2dim kontinuitaet}
und
\eqref{skript:2dim navier-stokes}
bilden ein System von partiellen Differentialgleichungen für die zwei
unbekannten Funktionen $v_x$ und $v_y$, die Komponenten der 
Strömungsgeschwindigkeit.
Wir werden im Folgenden zeigen, dass die 
\eqref{skript:2dim kontinuitaet}
ermöglicht, das System auf eine einzelne partielle Differentialgleichung
für nur eine einzige Funktion zu reduzieren.

\subsubsection{Satz von Green}
\begin{figure}
\centering
\includegraphics{chapters/2/green-curve.pdf}
\caption{Satz von Green: Das Wegintegral entlang der Randkurve $C$
stimmt mit dem zweifachen Integral über $D$ überein.
\label{skript:green-kurve}}
\end{figure}
Die Kontinuitätsgleichung 
\eqref{skript:2dim kontinuitaet}
ist ausgeschrieben
\[
0
=
\nabla\cdot\vec v
=
\frac{\partial v_x}{\partial x} + \frac{\partial v_y}{\partial y}.
\]
Die Divergenz auf der rechten Seite kommt auch im Satz von Green
vor:

\begin{satz}[Green]
\label{skript:2dim green}
Sei $D$ in kompaktes Gebiet in der $x$-$y$-Ebene mit Rand $\partial D=C$.
Weiter seine $f,g\colon D\to\mathbb R$ stetige Funktionen, die in $D$ 
stetig differenzierbar sind.
Dann gilt
\begin{equation}
\iint_{D}
\frac{\partial g(x,y)}{\partial x}
-
\frac{\partial f(x,y)}{\partial y}\,dx\,dy
=
\oint_C (f(x,y)\,dx + g(x,y)\,dy)
\label{skript:green formel}
\end{equation}
(Abbildung~\ref{skript:green-kurve}).
\end{satz}

Das Integral auf der rechten Seite wird mit Hilfe einer Parametrisierung
$\gamma\colon [a,b]\to\mathbb R^2$ der Randkurve $C$ definiert.
\begin{align*}
\oint_C f(x,y)\,dx
&=
\int_a^b f(\gamma_x(t),\gamma_y(t))\dot{\gamma}_x(t)\,dt,
\\
\oint_C g(x,y)\,dy
&=
\int_a^b g(\gamma_x(t),\gamma_y(t))\dot{\gamma}_y(t)\,dt.
\end{align*}
Es kann auch vektoriell mit dem Skalarprodukt als
\[
\oint_C
\underbrace{
\begin{pmatrix}f\\g\end{pmatrix}
}_{\displaystyle =\vec{w}}
\cdot
\begin{pmatrix}\dot\gamma_x\\\dot\gamma_y\end{pmatrix}
\,dt
=
\oint_C \vec{w}\cdot\dot\gamma(t)\,dt
=:
\oint_C \vec{w}\cdot d\vec{s}
\]
geschrieben werden kann.

\subsubsection{Stromfunktion}
Wir wenden den Satz \ref{skript:2dim green} von Green auf die
Funktionen $f(x,y)=-v_y(x,y)$ und $g(x,y)=v_x(x,y)$ an.
Die Formel \eqref{skript:green formel} ergibt
\begin{equation}
\oint_C (-v_y(x,y)\,dx + v_x(x,y)\,dy)
=
\iint_D
\underbrace{
\frac{\partial v_x(x,y)}{\partial x}
+
\frac{\partial v_y(x,y)}{\partial y}
}_{\displaystyle=0}
\,dx\,dy
=
0.
\label{skript:wegunabh}
\end{equation}
Man kann dieses Resultat auch wie folgt interpretieren.
Wenn $C_1$ und $C_2$ zwei Kurven sind, die den Punkt $A$ mit dem Punkt $B$
verbinden, dann lässt sich eine geschlossene Kurve $C$ konstruieren, indem
zuerst die Kurve $C_1$ von $A$ nach $B$ durchlaufen wird und dann die
Kurve $C_2$ in umgekehrter Richtung von $B$ nach $A$.
Die Formel \eqref{skript:wegunabh} besagt dann, dass 
\begin{align*}
0
&=
\oint_{C} (-v_y(x,y)\,dx + v_x(x,y)\,dy)
\\
&=
\int_{C_1} (-v_y(x,y)\,dx + v_x(x,y)\,dy)
-
\int_{C_2} (-v_y(x,y)\,dx + v_x(x,y)\,dy)
\end{align*}
oder
\begin{align*}
\Rightarrow
\qquad
\int_{C_1} (-v_y(x,y)\,dx + v_x(x,y)\,dy)
&=
\int_{C_2} (-v_y(x,y)\,dx + v_x(x,y)\,dy).
\end{align*}
Das Wegintegral hängt also nicht von der Wahl des Weges ab, jeder Weg
von $A$ nach $B$ führt auf den gleichen Wert des Integrals.

\begin{figure}
\centering
\includegraphics{chapters/2/green-curves.pdf}
\caption{Verschiedene Pfade zur Berechnung der Funktion $\psi(B)$
führen auf den gleichen Wert von $\psi(B)$ und ermöglichen, die partiellen
Ableitungen zu berechnent.
\label{skript:psi-pfade}}
\end{figure}

Wir halten den Punkt $A$ fest und definieren die Funktion
\[
\psi(B)
=
\int_C (-v_y(x,y)\,dx + v_x(x,y)\,dy)
\]
für einen beliebigen Weg von $A$ nach $B$. 
Zum Beispiel können für die Berechnung die Kurven
$C_1$ oder $C_2$ in Abbildung~\ref{skript:psi-pfade}
verwendet werden.
Damit lassen sich die Integrale ausschreiben:
\begin{align*}
\psi(x,y)
&=
\int_{C_1} (-v_y(x,y)\,dx + v_x(x,y)\,dy)
=
-\int_{x_0}^x v_y(\xi, y_0)\,d\xi
+
\int_{y_0}^y v_x(x,\eta)\,d\eta
\\
&=
\int_{C_2} (-v_y(x,y)\,dx + v_x(x,y)\,dy)
=
\int_{y_0}^y v_x(x_0,\eta)\,d\eta
-
\int_{x_0}^x v_y(\xi,y)\,d\xi.
\end{align*}
Diese Ausdrücke erlauben uns, die partiellen Ableitungen von $\psi(x,y)$
zu berechnen.
Für die Ableitung nach $x$ verwenden wir den zweiten Ausdruck, für die
Ableitung nach $y$ den ersten.
Wir erhalten
\begin{align*}
\frac{\partial\psi(x,y)}{\partial x}
&=
-\frac{\partial}{\partial x} \int_{x_0}^x v_y(\xi,y)\,d\xi
=
-v_y(x,y),
\\
\frac{\partial\psi(x,y)}{\partial y}
&=
\frac{\partial}{\partial y} \int_{y_0}^y v_x(x,\eta)\,d\eta
=
v_x(x,y).
\end{align*}
In vektorieller Form kann man dies auch als
\begin{equation}
\vec{v}
=
\begin{pmatrix}v_x\\v_y \end{pmatrix}
=
\underbrace{
\begin{pmatrix}0&-1\\1&0\end{pmatrix}
}_{\displaystyle=J}
\begin{pmatrix}
\frac{\partial\psi}{\partial x}\\
\frac{\partial\psi}{\partial y}
\end{pmatrix}
=
J\nabla\psi
\label{skript:Jnablapsi}
\end{equation}
schreiben.
Aus der Funktion $\psi$ lässt sich das Vektorfeld $\vec{v}$ also wieder
rekonstruieren.
Sie heisst die {\em Stromfunktion} des Vektorfeldes $\vec{v}$.
\index{Stromfunktion}
Natürlich ist $\psi(x,y)$ nur bis auf eine Konstante bestimmt.

Umgekehrt ist für jede beliebige Funktion $\varphi(x,y)$ das Vektorfeld
$\vec{u}=J\nabla\varphi$ divergenzfrei:
\begin{align*}
\nabla\cdot\vec{u}
=
\frac{\partial}{\partial x}
\biggl(-\frac{\partial\varphi}{\partial y}\biggr)
+
\frac{\partial}{\partial y}
\biggl(\frac{\partial\varphi}{\partial x}\biggr)
=
-\frac{\partial^2\varphi}{\partial x\,\partial y}
+\frac{\partial^2\varphi}{\partial y\,\partial x}
=
0.
\end{align*}
Die Darstellung \eqref{skript:Jnablapsi} des Geschwindigkeitsfeldes 
erlaubt eine geometrische Interpretation.
Der Gradient $\nabla\psi$ ist ein Vektorfeld, welches auf den
Niveaulinien der Funktion $\psi$ senkrecht steht.
Je schneller die Zunahme von $\psi$, desto grösser ist der
Vektor $\nabla\psi$.

Die Matrix $J$ ist eine Drehmatrix, sie dreht Vektoren um $90^\circ$ 
im Gegenuhrzeigersinn.
Die Vektoren $J\nabla\psi$ sind also tangential an die Niveaulinien,
die Niveaulinien sind also gleichzeitig die Stromlinien der Strömung.
Ist die Strömung auf ein kompaktes Gebiet beschränkt, dann ist der
Rand des Gebietes eine Stromlinie, also eine Niveaulinie von $\psi$.
Da $\psi$ nur bis auf eine Konstante festgelegt ist, kann man $\psi$
so wählen, dass der Rand des Gebietes durch die Gleichung $\psi(x,y)=0$
beschrieben wird.

\index{Stromlinien}
\begin{figure}
\centering
\includegraphics{chapters/2/rotation.pdf}
\caption{Strömungsfunktion $\psi(x,y)=a(x^2+y^2)$ und das zugehörige
Vektorfeld.
Die Strömungsgeschwindigkeit ist proportional zum Radius, es handelt
sich also um eine starre Drehung um den Nullpunkt.
\label{skript:rotation}}
\end{figure}
Die Funktion $\psi(x,y)=a(x^2+y^2)$ führt auf das Vektorfeld
\[
\vec{v}
=
J\nabla\psi
=
2a
\begin{pmatrix}
-y\\x
\end{pmatrix}
\]
(Abbildung~\ref{skript:rotation}).
Die Strömungsgeschwindigkeit ist $2a\sqrt{x^2+y^2}=2ar$, es handelt sich
also um eine starre Drehung um den Nullpunkt des Koordinatensystems mit
Winkelgeschwindigkeit $\omega=2a$.

\subsubsection{Vorticity}
Wir suchen eine Grösse, mit der wir das Ausmass messen können, wie schnell
sich das Fluid dreht.
Die Winkelgeschwindigkeit bei der Drehung um den Punkt $(x,y)$ können wir
durch Vergleich der Geschwindigkeit an den Punkten $(x\pm h,y)$ und
$(x,y\pm h)$ finden.
Es ist
\[
\omega
=
\frac{v_y(x+h),y) - v_y(x-h,y)}{2h}
=
\frac{-v_x(x,y+h) + v_x(x,y-h)}{2h}.
\]
Beim Grenzübergang $h\to 0$ erhalten wir
\[
\omega
=
\frac{\partial v_y(x,y)}{\partial x}
=
-\frac{\partial v_x{x,y}}{\partial y}
\qquad\text{oder}\qquad
2\omega
=
\frac{\partial v_y}{\partial x}-\frac{\partial v_x}{\partial y}.
\]
Damit haben wir eine Grösse gefunden, die als Mass für die Drehgeschwindigkeit
dienen kann.
\index{Vorticity}
\begin{definition}
Ist $\vec{v}$ das Geschwindigkeitsfeld der Strömung, dann schreiben wir
\[
\nabla\times\vec{v}
=
\frac{\partial v_y}{\partial x}-\frac{\partial v_x}{\partial y}
=
\zeta.
\]
Die Funktion $\zeta$ heisst die {\em Vorticity} des Strömungsfeldes.
\end{definition}

Beschreibt man die Strömung mit Hilfe der Strömungsfunktion, dann gilt
für die Vorticity
\begin{equation}
\zeta
=
\nabla\times\vec{v}
=
\nabla\times J\nabla\psi
=
\frac{\partial}{\partial x}
\biggl(-\frac{\partial\psi}{\partial y}\biggr)
-
\frac{\partial}{\partial x}
\frac{\partial\psi}{\partial x}
=
-\biggl(
\frac{\partial^2\psi}{\partial x^2}
+
\frac{\partial^2\psi}{\partial y^2}
\biggr)
=
-\Delta \psi.
\label{skript:stroemung-vorticity}
\end{equation}
Der Laplace-Operator verbindet also die Strömungsfunktion direkt mit der
Vorticity.
Für die Strömung in einem kompakten Gebiet $\Omega$ ist der Rand eine
Niveaulinie von $\psi$.
Wie früher dargelegt können wir $\psi$ so wählen, dass $\psi=0$ gilt auf
dem Rand.
Bei gegebener Vorticity $\zeta$ ist daher $\psi$ die Lösung der
partiellen Differentialgleichung
\begin{equation}
\Delta \psi = -\zeta\quad\text{in $\Omega$}
\qquad
\psi = 0\quad\text{auf $\partial\Omega$}.
\label{skript:zetapsielliptisch}
\end{equation}
Die Theorie der elliptischen partiellen Differentialgleichungen 
sagt, dass $\psi$ eindeutig bestimmt ist.
Statt die Strömungsgleichungen für $\psi$ zu lösen, können wir
also auch versuchen, eine Gleichung für die Vorticity $\zeta$
aufzustellen und dann mit Hilfe des elliptischen partiellen
Randwertproblems \eqref{skript:zetapsielliptisch} die Strömungsfunktion
und schliesslich $\vec{v}$ bestimmen.

Um die Differentialgleichung für $\zeta$ zu finden, wenden wir den
Operator $\nabla\times\mathstrut$ auf die Bewegungsgleichung
\eqref{skript:2dim navier-stokes}
an:
\[
\nabla\times\frac{\partial \vec{v}}{\partial t}
=
-\nabla\times(\nabla\cdot(\vec{v}\vec{v}))+\nabla\times\vec b + \nabla\times\biggl(\frac1{\varrho}\nabla\cdot\bm{\tau}\biggr).
\]
Die linke Seite ist die Zeitableitung der Vorticity.
Die Divergenz von $\vec{v}\vec{v}$ ist 
\[
(\nabla\cdot(\vec{v}\vec{v}))_j
=
\sum_i \frac{\partial}{\partial i}v_iv_j
=
\biggl(\sum_i \frac{\partial v_i}{\partial i}\biggr)v_j
+
\sum_i v_i\frac{\partial v_j}{\partial i}
=
(\underbrace{\nabla\cdot\vec{v}}_{\displaystyle=0})v_j
+
\vec{v}\cdot\nabla v_j.
\]
Es folgt
\[
\nabla\cdot(\vec{v}) \vec{v})
=
\vec{v}\cdot\nabla \vec{v}.
\]
Daraus kann man jetzt auch die Vorticity berechnen:
\begin{align*}
\nabla\times(\nabla\cdot(\vec{v}\vec{v}))
&=
\nabla\times(\vec{v}\cdot\nabla\vec{v})
=
\frac{\partial}{\partial x}(\vec{v}\cdot\nabla v_y)
-
\frac{\partial}{\partial y}(\vec{v}\cdot\nabla v_x)
\\
&=
\vec{v}\cdot\nabla
\biggl(\frac{\partial v_y}{\partial x}-\frac{\partial v_x}{\partial y}\biggr)
+
\frac{\partial\vec{v}}{\partial x} \cdot\nabla v_y
-
\frac{\partial\vec{v}}{\partial y} \cdot\nabla v_x
\\
&=
\vec{v}\cdot\nabla\zeta
+
\frac{\partial v_x}{\partial x} \frac{\partial v_y}{\partial x}
+
\frac{\partial v_y}{\partial x} \frac{\partial v_y}{\partial y}
-
\frac{\partial v_x}{\partial y} \frac{\partial v_x}{\partial x}
-
\frac{\partial v_y}{\partial y} \frac{\partial v_x}{\partial y}
\\
&=
\vec{v}\cdot\nabla\zeta
+
\frac{\partial v_x}{\partial x}
\biggl(\frac{\partial v_y}{\partial x}-\frac{\partial v_x}{\partial y}\biggr)
+
\frac{\partial v_y}{\partial y}
\biggl(\frac{\partial v_y}{\partial x}-\frac{\partial v_x}{\partial y}\biggr)
\\
&=
\vec{v}\cdot\nabla\zeta
+
(\nabla\cdot\vec{v})\zeta
=
\vec{v}\cdot\nabla\zeta.
\end{align*}
Wir waren also nicht ganz erfolgreich, die Geschwindigkeit aus der
Bewegungsgleichung zu eliminieren.
Wir haben nur die Form
\[
\frac{\partial \zeta}{\partial t}
=
-\vec{v}\cdot\nabla\zeta
+\nabla\times\vec b
+ \nabla\times\biggl(\frac1{\varrho}\nabla\cdot\bm{\tau}\biggr)
\]
erreicht.
Ausserdem ist es möglich, dass die Spannungen $\bm{\tau}$ ebenfalls von
den Geschwindigkeiten abhängig sind.

Wir können aber die Vorticity auch noch durch die Strömungsfunktion
ausdrücken.
Ersetzen wir $\zeta=-\Delta\psi$ in der Bewegungsgleichung, erhalten
wir
\begin{equation}
\frac{\partial \Delta \psi}{\partial t}
=
-J\nabla\psi\cdot\nabla\Delta\psi
+\nabla\times\vec b
+ \nabla\times\biggl(\frac1{\varrho}\nabla\cdot\bm{\tau}\biggr).
\label{skript:voritictyequation}
\end{equation}
Jetzt ist die Strömung vollständig durch die einzige unbekannte Funktion 
$\psi$.

Der erste Term auf der rechten Seite von \eqref{skript:voritictyequation}
kann noch etwas kompakter geschrieben werden.
Es ist
\begin{align*}
(J\nabla f)\cdot(\nabla g)
&=
\begin{pmatrix}
-\frac{\partial f}{\partial y}
\\
\frac{\partial f}{\partial x}
\end{pmatrix}
\cdot
\begin{pmatrix}
\frac{\partial g}{\partial x}\\
\frac{\partial g}{\partial y}
\end{pmatrix}
=
\frac{\partial f}{\partial x}
\frac{\partial g}{\partial y}
-\frac{\partial f}{\partial y}
\frac{\partial g}{\partial x}.
\end{align*}
Der Ausdruck auf der rechten Seite kommt auch in anderem Zusammenhang vor.

\begin{definition}
Seien $f$ und $g$ Funktionen der Variablen $x$ und $y$.
Dann heisst
\[
\frac{\partial(f,g)}{\partial(x,y)}
=
\frac{\partial f}{\partial x}
\frac{\partial g}{\partial y}
-
\frac{\partial f}{\partial y}
\frac{\partial g}{\partial x}
\]
die
{\em Funktionaldeterminante} oder {\em Jacobische Determinante} von
$f$ und $g$.
\end{definition}
\index{Funktionaldeterminante}
\index{Jacobi-Determinante}
Mit dieser Definition wird die Bewegungsgleichung
\begin{equation}
\frac{\partial \Delta \psi}{\partial t}
=
\nabla\times\vec b
+ \nabla\times\biggl(\frac1{\varrho}\nabla\cdot\bm{\tau}\biggr)
-\frac{\partial(\psi,\Delta\psi)}{\partial(x,y)}.
\label{skript:psidgl}
\end{equation}
Im Falle der Boussinesq-Näherung kommt noch ein Term für den
Auftrieb hinzu.

\subsubsection{Spannungen und Stromfunktion}
Für eine newtonsche Flüssigkeit haben wir in 
\eqref{skript:inkompressibel newtonsch}
bereits den Spannungstensor durch den Druck und die Spannungen
ausgedrückt.
Aus der Bewegungsgleichung für die Geschwindigkeit haben wir die
Differentialgleichung \eqref{skript:psidgl} erhalten, indem
wir den Operator $\nabla\times\mathstrut$ angewendet haben.
Wir müssen jetzt also 
\[
\nabla\times\biggl(\frac1{\varrho}(\nabla p-\nu\Delta\vec{v})\biggr)
=
\frac{1}{\varrho}
\biggl(
\nabla\times\nabla p
-
\nu \nabla\times\Delta\vec{v}
\biggr)
=
\frac1{\varrho}
\biggl(
\frac{\partial}{\partial x}\frac{\partial p}{\partial y}
-
\frac{\partial}{\partial y}\frac{\partial p}{\partial x}
-\nu\Delta\nabla\times\vec{v}
\biggr)
\]
berechnen.
Der erste Term fällt weg, weil es auf die Reihenfolge der zweiten
Ableitungen nicht ankommt.
Im zweiten Term haben wir angekommen, dass $\nu$ nicht vom Ort
abhängigt.
Dies ist genau genommen nicht richtig, da $\nu$ zum Beispiel
stark von der Temperatur abhängt, die ebenfalls nicht konstant sein
muss.
Der zweite Term in der Klammer ist natürlich einfach
$\nu\Delta\zeta=-\nu\Delta^2\psi$.
Damit bekommen wir die Bewegungsgleichung für $\psi$ in der Form
\begin{equation}
\frac{\partial \Delta \psi}{\partial t}
=
\nabla\times\vec b
+\frac{\nu}{\varrho}\Delta^2\psi
-\frac{\partial(\psi,\Delta\psi)}{\partial(x,y)}.
\label{skript:psidgl2}
\end{equation}

