%
% pdeloesung.tex -- Loesungsprinzip für partielle Differentialgleichungen
%
% (c) 2018 Prof Dr Andreas Müller, Hochschule Rapperswil
%
\section{Lösungen von partiellen Differentialgleichungen}
In diesem Kapitel haben wir Strömungen mit Hilfe partieller
Differentialgleichungen beschrieben.
Funktionenräume sind unendlich dimensional, was sehr erschwert,
dafür überhaupt eine Lösung zu finden.

\subsection{Diskretisation}
Der einfachste Ansatz, partielle Differntialgleichungen zu lösen,
folgt den Verfahren, die auch für gewöhnliche Differentialgleichungen
\cite{skript:mathsem-dgl} 

\subsection{Basisfunktionen}
Lineare Differentialgleichungen haben die Eigenschaft, dass 
mit zwei Lösungen auch deren Linearkombinationen wieder Lösungen sind.
In der Theorie der gewöhnlichen linearen Differentialgleichungen endlicher
Ordnung geben die Koeffizienten der Linearkombination die nötige Flexibilität,
die Anfangsbedingungen zu erfüllen.
Dasselbe gilt auch für partielle Differentialgleichungen, mit dem Unterschied
allerdings, dass es meinstens unendlich viele linear unabhängige Lösungen
gibt.
Ist $L$ ein linearer Differentialoperator und $u_k$, $k\in\mathbb N$ eine
Familie von Lösungen der Gleichung $Lu=0$.
Im besten Fall lässt sich dann jede andere Lösung $u$ in einem noch
zu definierenden Sinn beliebig genau durch Linearkombinationen
\[
u=\sum_{k\in\mathbb N} a_ku_k
\]
approximiert werden.
Wir verzichten darauf, auf die analytischen Details einzugehen.

Bei einer nichtlinearen Differentialgleichung ist es sicher nicht
mehr möglich, Lösungen durch Linearkombination anderer Lösungen
zu bilden.
Aber es spricht nichts dagegen, dass es eine Familie $u_k$, $k=1,\dots,N$,
von Funktionen gibt, mit der man Lösungen genügend genau approximieren
kann.

\subsubsection{Die Potenzreihenmethode}
Diese Idee liegt zum Beispiel der Potenzreihenmethode zu Grunde.
Dabei nimmt man an, dass die Lösung einer Differentialgleichung als
Linearkombination der Potenzfunktionen $u_k(x)=x^k, k=0\dots,N$
approximiert werden kann.
Als Beispiel betrachten wir die Differentialgleichung
\[
y''=-\lambda y.
\]
Eine Potenzfunktion $u_k(x)=x^k$ ist offensichtlich keine Lösung.
Man kann aber die Lösung als eine Linearkombination
der Potenzfunktionen
\[
y(x)
=
a_0+a_1x+a_2x^2+a_3x^3+\dots
\]
ansetzen und in die Differentialgleichung einsetzen.
Man erhält
\[
2\cdot 1 \cdot a_2
+
3\cdot 2 \cdot a_3x
+
4\cdot 3 \cdot a_4x^2
+
\dots
=
-\lambda(
a_0+a_1x+a_2x^2+a_3x^3+\dots)
\]
Man liest daraus die Gleichungen für die Koeffizienten $a_k$ ab:
\begin{equation*}
\left.
\begin{aligned}
2\cdot 1\cdot a_2&=-\lambda a_0\\
3\cdot 2\cdot a_3&=-\lambda a_1\\
4\cdot 3\cdot a_4&=-\lambda a_2\\
5\cdot 4\cdot a_5&=-\lambda a_3\\
&\;\vdots
\end{aligned}
\right\}
\qquad\Rightarrow\qquad
\left\{
\begin{aligned}
a_{2k}  &=(-1)^k\frac{\lambda^k}{ 2k!   }a_0\\
a_{2k+1}&=(-1)^k\frac{\lambda^k}{(2k+1)!}a_1
\end{aligned}
\right.
\end{equation*}
Daraus kann man erkennen, dass jede Lösung die Form
\begin{align*}
u(x)
&=
a_0
\biggl(
1-\frac{(\sqrt{\lambda}x)^2}{2!} + \frac{(\sqrt{\lambda}x)^4}{4!}-\dots
\biggr)
+
\frac{a_1}{\sqrt{\lambda}}
\biggl(
\sqrt{\lambda}x
-
\frac{(\sqrt{\lambda}x)^3}{3!} + \frac{(\sqrt{\lambda}x)^5}{5!}-\dots
\biggr)
\\
&= a_0 \cos \sqrt{\lambda}x
+ \frac{a_1}{\sqrt{\lambda}} \sin\sqrt{\lambda}x
\end{align*}
hat.

\subsubsection{Erfolgsfaktoren}
Aus dem eben entwickelten Beispiel kann man einige heuristische Regeln ableiten,
wie die Funktionenfamilie $u_k$ beschaffen sein muss, damit es
durchführbar ist.
\begin{enumerate}
\item Die Ableitungen $u_k'(x)$ der Funktionen können durch
Linearkombinationen derselben ausgedrückt werden.
Im Beispiel ist $u'_k(x)=kx^{k-1}=ku_{k-1}(x)$.
\item Produkte von Funktionen $u_k(x)$ lassen sich durch Linearkombinationen
approximieren.
Im Beispiel ist $u_k(x)u_l(x)=x^kx^k = x^{k+l}=u_{k+l}(x)$.
\item Beim Einsetzen des Ansatzes in die Differentialgleichungen
entstehen algebraische Gleichungen, aus denen die Koeffizienten
bestimmt werden können.
Im Beispiel 
\end{enumerate}

\subsubsection{Alternative Lösung für das Beispiel}

\subsubsection{Ein etwas komplexeres Beispiel}
Wir versuchen, diese Idee auf die Lösung der Gleichung von
Burgers
\[
\frac{\partial u}{\partial t} + u\frac{\partial u}{\partial x}=0
\]
anzuwenden.
Wir suchen $2\pi$-periodische Lösungen für ebensolche Anfangsbedingung
$u(0,x)=g(x)$.
Wie im Kapitel~\ref{chapter:fourier} dargelegt wird, sind die Funktionen
$\cos kx$ und $\sin kx$ geeignete Basisfunktionen für das skizzierte Verfahren.
Zur Vereinfachung der Rechnung verwenden wir statt der reellen Funktion
$\cos kx$ und $\sin kx$ die komplexen Exponentialfunktionen $e^{ikx}$.
Als Ansatz verwenden wir daher
\[
u(x) = \sum_{k\in\mathbb Z} c_k(t) e^{ikx}
\]
und setzen dies in die Differentialgleichung ein, die dadurch zu
\[
\sum_{k\in\mathbb Z} \dot c_k(t) e^{ikx}
=
\sum_{j,l\in\mathbb Z} c_j(t)e^{ijx} c_l(t) le^{ilx}
=
\sum_{j,l\in\mathbb Z} c_j(t)c_l(t) le^{i(j+l)x}
\]
Daraus lesen wir die Gleichungen
\begin{equation}
\dot c_k(t) = \sum_{l} lc_l(t)c_{k-l}(t)
\label{burgers:gewoedgl}
\end{equation}
für die Koeffizienten ab.
Wir haben also ein System von gewöhnlichen linearen Differentialgleichungen
erster Ordnung gefunden, und damit das partielle Differentialgleichungssystem
auf ein einfachers System reduziert.
Allerdings ist von der rechten Seite der Differentialgleichung
\eqref{burgers:gewoedgl}
nicht einmal garantiert, dass diese Summen konvergieren.

Für eine approximative Lösung vernachlässigen wir die Koeffizienten $c_k$
mit $|k|>N$.


\subsection{Separation}



