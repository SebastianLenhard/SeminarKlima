%
% anforderungen.tex -- Anforderungen an Klima-Modelle
%
% (c) 2018 Prof Dr Andreas Müller, Hochschule Rapperswil
%

\section{Anforderungen an Klima-Modelle\label{section:anforderungen}}
Aus der vorangegangenen Diskussion können wir einige Anforderungen
ableiten, was Klimamodelle können müssen, was sie berücksichten
müssen und welche Aspekte sie vernachlässigen können.

Das Ziel ist die Modellierung der Klima-Entwicklung über wenige
hundert Jahre.
Es ist jedoch nicht erforderlich, den vollständigen Zustand der
Atmosphäre von Tag zu Tag zu modellieren.
Es genügt diejenigen Eigenschaften zu modellieren, die 
für den Energiehaushalt der Erde wesentlich sind.
Dazu gehören die folgenden Eigenschaften.
\begin{enumerate}
\item
Der Strahlungshaushalt der Erde muss korrekt modelliert werden,
da dies die global Mitteltemperatur bestimmt.
Dies bedeutet insbesondere auch, dass die Albedo sowie der Gehalt an
Treibhausgasen korrekt wiedergeben werden.
\item
Die Albedo der Erde muss modelliert sein. 
D.~h.~der durchschnittliche Vereisungsgrad und die Häufigkeit und
Dichte von Bewölkung muss korrekt wiedergeben sein.
\item
Strahlungs- und Wasserhaushalt der Atmosphäre unterscheiden sich
über Kontinenten und über den Ozeanen.
Das Modell muss daher räumlich genügend aufgelöst sein, dass die
für den Energiehaushalt wesentlichen Unterschiede abgebildet
werden können.
\item
Die Energietransportmechanismen müssen für Zeitskalen in der Grössenordnung
von Jahren und Jahrzehnten korrekt modelliert sein, weil dies die
Verteilung der Energie über die Erdoberfläche festlegt.
\item
Wasser in der Atmosphäre hat einerseits einen grossen Einfluss auf
den Treibhauseffekt, übernimmt aber auch für einen wesentlichen Teil
des Energietransports in der Atmosphäre.
Daher muss der Wassergehalt 
\item
Der Salzgehalt der Meere treibt die thermohaline Zirkulation an, welche
auf einer Zeitskala von Jahrzehnten einen wesentlichen Beitrag zum
Energietransport in den Ozeanen leistet.
Salzgehalt und Verdunstung an der Meeresoberfläche muss so genau
modelliert sein, dass diese Energieströme korrekt modelliert werden.
\end{enumerate}

Um die Auswirkungen des Klimawandels zu verstehen muss man vorhersagen,
wie sich kurzfristige Wetterphänomene verändern.
Dazu kann man gewöhnliche Wettermodelle verwenden, die sowohl zeitlich wie
auch räumlich eine bessere Auflösung haben.
Man kann aber gewisse qualitative Aussagen auch ohne solche detaillierten
Modelle machen.
Ein höherer Wassergehalt der Atmosphäre wird zum Beispiel zunächst
zu stärkeren Niederschlägen führen.
Da aber auch mehr Energie in Form von latenter Wärme zur Verfügung
steht, muss man auch mit stärkeren Winden rechnen.
Zum Beispiel muss man also damit rechnen, dass Hurrikane intensiver
werden.

