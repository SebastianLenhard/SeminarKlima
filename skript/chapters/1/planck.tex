%
% planck.tex
%
% (c) 2018 Prof Dr Andreas Müller, Hochschule Rapperswil
%
\documentclass[tikz]{standalone}
\usepackage{times}
\usepackage{amsmath}
\usepackage{txfonts}
\usepackage[utf8]{inputenc}
\usepackage{graphics}
\usetikzlibrary{arrows,intersections}
\begin{document}
\begin{tikzpicture}[thick, >= latex, xscale=5]

\draw[->] (-0.,0)--(2.1,0) coordinate[label={above:$\lambda$}];
\draw[->] (0,-0.1)--(0,9.3)
	coordinate[label={right:$E\;[\mu\text{W}/\text{nm}\cdot\text{m}^2]$}];

\foreach \x in {0,1,...,2}{
	\draw ({\x},-0.05)--({\x},0.05);
}
\foreach \y in {0,1,...,9}{
	\draw (-0.02,{\y})--(0.02,{\y});
}

\node at (0,-0.1) [below] {$100$nm};
\node at (1,-0.1) [below] {$1\mu$m};
\node at (2,-0.1) [below] {$10\mu$m};

\foreach \y in {1,2,...,9}{
	\node at (-0.03,{\y}) [left] {$\y 0$};
}

\draw[line width=0.1] (0.699,0)--(0.699,9);
\draw (0.699,-0.05)--(0.699,0.05);
\node at (0.699,-0.1) [below] {$500$nm};

\draw[color=red] (0.000, 0.000)
--(0.002, 0.000)
--(0.004, 0.000)
--(0.006, 0.000)
--(0.008, 0.000)
--(0.010, 0.000)
--(0.012, 0.000)
--(0.014, 0.000)
--(0.016, 0.000)
--(0.018, 0.000)
--(0.020, 0.000)
--(0.022, 0.000)
--(0.024, 0.000)
--(0.026, 0.000)
--(0.028, 0.000)
--(0.030, 0.000)
--(0.032, 0.000)
--(0.034, 0.000)
--(0.036, 0.000)
--(0.038, 0.000)
--(0.040, 0.000)
--(0.042, 0.000)
--(0.044, 0.000)
--(0.046, 0.000)
--(0.048, 0.000)
--(0.050, 0.000)
--(0.052, 0.001)
--(0.054, 0.001)
--(0.056, 0.001)
--(0.058, 0.001)
--(0.060, 0.001)
--(0.062, 0.001)
--(0.064, 0.001)
--(0.066, 0.001)
--(0.068, 0.001)
--(0.070, 0.001)
--(0.072, 0.001)
--(0.074, 0.001)
--(0.076, 0.001)
--(0.078, 0.001)
--(0.080, 0.002)
--(0.082, 0.002)
--(0.084, 0.002)
--(0.086, 0.002)
--(0.088, 0.002)
--(0.090, 0.002)
--(0.092, 0.002)
--(0.094, 0.002)
--(0.096, 0.003)
--(0.098, 0.003)
--(0.100, 0.003)
--(0.102, 0.003)
--(0.104, 0.003)
--(0.106, 0.004)
--(0.108, 0.004)
--(0.110, 0.004)
--(0.112, 0.005)
--(0.114, 0.005)
--(0.116, 0.005)
--(0.118, 0.006)
--(0.120, 0.006)
--(0.122, 0.006)
--(0.124, 0.007)
--(0.126, 0.007)
--(0.128, 0.008)
--(0.130, 0.008)
--(0.132, 0.009)
--(0.134, 0.009)
--(0.136, 0.010)
--(0.138, 0.010)
--(0.140, 0.011)
--(0.142, 0.012)
--(0.144, 0.012)
--(0.146, 0.013)
--(0.148, 0.014)
--(0.150, 0.015)
--(0.152, 0.016)
--(0.154, 0.016)
--(0.156, 0.017)
--(0.158, 0.018)
--(0.160, 0.020)
--(0.162, 0.021)
--(0.164, 0.022)
--(0.166, 0.023)
--(0.168, 0.024)
--(0.170, 0.026)
--(0.172, 0.027)
--(0.174, 0.029)
--(0.176, 0.030)
--(0.178, 0.032)
--(0.180, 0.034)
--(0.182, 0.036)
--(0.184, 0.037)
--(0.186, 0.039)
--(0.188, 0.042)
--(0.190, 0.044)
--(0.192, 0.046)
--(0.194, 0.048)
--(0.196, 0.051)
--(0.198, 0.053)
--(0.200, 0.056)
--(0.202, 0.059)
--(0.204, 0.062)
--(0.206, 0.065)
--(0.208, 0.068)
--(0.210, 0.072)
--(0.212, 0.075)
--(0.214, 0.079)
--(0.216, 0.083)
--(0.218, 0.086)
--(0.220, 0.091)
--(0.222, 0.095)
--(0.224, 0.099)
--(0.226, 0.104)
--(0.228, 0.109)
--(0.230, 0.114)
--(0.232, 0.119)
--(0.234, 0.124)
--(0.236, 0.130)
--(0.238, 0.135)
--(0.240, 0.141)
--(0.242, 0.147)
--(0.244, 0.154)
--(0.246, 0.160)
--(0.248, 0.167)
--(0.250, 0.174)
--(0.252, 0.182)
--(0.254, 0.189)
--(0.256, 0.197)
--(0.258, 0.205)
--(0.260, 0.214)
--(0.262, 0.223)
--(0.264, 0.232)
--(0.266, 0.241)
--(0.268, 0.250)
--(0.270, 0.260)
--(0.272, 0.270)
--(0.274, 0.281)
--(0.276, 0.292)
--(0.278, 0.303)
--(0.280, 0.315)
--(0.282, 0.326)
--(0.284, 0.339)
--(0.286, 0.351)
--(0.288, 0.364)
--(0.290, 0.377)
--(0.292, 0.391)
--(0.294, 0.405)
--(0.296, 0.420)
--(0.298, 0.435)
--(0.300, 0.450)
--(0.302, 0.466)
--(0.304, 0.482)
--(0.306, 0.498)
--(0.308, 0.515)
--(0.310, 0.533)
--(0.312, 0.551)
--(0.314, 0.569)
--(0.316, 0.588)
--(0.318, 0.607)
--(0.320, 0.627)
--(0.322, 0.647)
--(0.324, 0.668)
--(0.326, 0.689)
--(0.328, 0.710)
--(0.330, 0.733)
--(0.332, 0.755)
--(0.334, 0.778)
--(0.336, 0.802)
--(0.338, 0.826)
--(0.340, 0.851)
--(0.342, 0.876)
--(0.344, 0.902)
--(0.346, 0.929)
--(0.348, 0.955)
--(0.350, 0.983)
--(0.352, 1.011)
--(0.354, 1.039)
--(0.356, 1.068)
--(0.358, 1.098)
--(0.360, 1.128)
--(0.362, 1.159)
--(0.364, 1.191)
--(0.366, 1.222)
--(0.368, 1.255)
--(0.370, 1.288)
--(0.372, 1.322)
--(0.374, 1.356)
--(0.376, 1.391)
--(0.378, 1.426)
--(0.380, 1.462)
--(0.382, 1.498)
--(0.384, 1.536)
--(0.386, 1.573)
--(0.388, 1.611)
--(0.390, 1.650)
--(0.392, 1.690)
--(0.394, 1.730)
--(0.396, 1.770)
--(0.398, 1.811)
--(0.400, 1.853)
--(0.402, 1.895)
--(0.404, 1.938)
--(0.406, 1.981)
--(0.408, 2.025)
--(0.410, 2.069)
--(0.412, 2.114)
--(0.414, 2.160)
--(0.416, 2.206)
--(0.418, 2.252)
--(0.420, 2.299)
--(0.422, 2.347)
--(0.424, 2.395)
--(0.426, 2.444)
--(0.428, 2.493)
--(0.430, 2.542)
--(0.432, 2.592)
--(0.434, 2.643)
--(0.436, 2.694)
--(0.438, 2.745)
--(0.440, 2.797)
--(0.442, 2.849)
--(0.444, 2.902)
--(0.446, 2.955)
--(0.448, 3.009)
--(0.450, 3.062)
--(0.452, 3.117)
--(0.454, 3.171)
--(0.456, 3.226)
--(0.458, 3.282)
--(0.460, 3.338)
--(0.462, 3.394)
--(0.464, 3.450)
--(0.466, 3.507)
--(0.468, 3.563)
--(0.470, 3.621)
--(0.472, 3.678)
--(0.474, 3.736)
--(0.476, 3.794)
--(0.478, 3.852)
--(0.480, 3.910)
--(0.482, 3.969)
--(0.484, 4.028)
--(0.486, 4.087)
--(0.488, 4.146)
--(0.490, 4.205)
--(0.492, 4.264)
--(0.494, 4.324)
--(0.496, 4.383)
--(0.498, 4.443)
--(0.500, 4.502)
--(0.502, 4.562)
--(0.504, 4.622)
--(0.506, 4.682)
--(0.508, 4.741)
--(0.510, 4.801)
--(0.512, 4.861)
--(0.514, 4.920)
--(0.516, 4.980)
--(0.518, 5.039)
--(0.520, 5.098)
--(0.522, 5.158)
--(0.524, 5.217)
--(0.526, 5.276)
--(0.528, 5.334)
--(0.530, 5.393)
--(0.532, 5.451)
--(0.534, 5.509)
--(0.536, 5.567)
--(0.538, 5.625)
--(0.540, 5.682)
--(0.542, 5.739)
--(0.544, 5.796)
--(0.546, 5.853)
--(0.548, 5.909)
--(0.550, 5.964)
--(0.552, 6.020)
--(0.554, 6.075)
--(0.556, 6.130)
--(0.558, 6.184)
--(0.560, 6.238)
--(0.562, 6.291)
--(0.564, 6.344)
--(0.566, 6.396)
--(0.568, 6.448)
--(0.570, 6.500)
--(0.572, 6.551)
--(0.574, 6.601)
--(0.576, 6.651)
--(0.578, 6.700)
--(0.580, 6.749)
--(0.582, 6.797)
--(0.584, 6.845)
--(0.586, 6.892)
--(0.588, 6.938)
--(0.590, 6.984)
--(0.592, 7.029)
--(0.594, 7.073)
--(0.596, 7.117)
--(0.598, 7.160)
--(0.600, 7.202)
--(0.602, 7.244)
--(0.604, 7.285)
--(0.606, 7.325)
--(0.608, 7.365)
--(0.610, 7.403)
--(0.612, 7.441)
--(0.614, 7.479)
--(0.616, 7.515)
--(0.618, 7.551)
--(0.620, 7.586)
--(0.622, 7.620)
--(0.624, 7.653)
--(0.626, 7.686)
--(0.628, 7.718)
--(0.630, 7.748)
--(0.632, 7.779)
--(0.634, 7.808)
--(0.636, 7.836)
--(0.638, 7.864)
--(0.640, 7.890)
--(0.642, 7.916)
--(0.644, 7.941)
--(0.646, 7.965)
--(0.648, 7.989)
--(0.650, 8.011)
--(0.652, 8.033)
--(0.654, 8.053)
--(0.656, 8.073)
--(0.658, 8.092)
--(0.660, 8.110)
--(0.662, 8.127)
--(0.664, 8.144)
--(0.666, 8.159)
--(0.668, 8.174)
--(0.670, 8.187)
--(0.672, 8.200)
--(0.674, 8.212)
--(0.676, 8.223)
--(0.678, 8.233)
--(0.680, 8.242)
--(0.682, 8.250)
--(0.684, 8.258)
--(0.686, 8.264)
--(0.688, 8.270)
--(0.690, 8.275)
--(0.692, 8.279)
--(0.694, 8.282)
--(0.696, 8.284)
--(0.698, 8.286)
--(0.700, 8.286)
--(0.702, 8.286)
--(0.704, 8.285)
--(0.706, 8.283)
--(0.708, 8.280)
--(0.710, 8.276)
--(0.712, 8.272)
--(0.714, 8.267)
--(0.716, 8.261)
--(0.718, 8.254)
--(0.720, 8.246)
--(0.722, 8.237)
--(0.724, 8.228)
--(0.726, 8.218)
--(0.728, 8.207)
--(0.730, 8.196)
--(0.732, 8.183)
--(0.734, 8.170)
--(0.736, 8.156)
--(0.738, 8.142)
--(0.740, 8.126)
--(0.742, 8.110)
--(0.744, 8.093)
--(0.746, 8.076)
--(0.748, 8.058)
--(0.750, 8.039)
--(0.752, 8.019)
--(0.754, 7.999)
--(0.756, 7.978)
--(0.758, 7.957)
--(0.760, 7.935)
--(0.762, 7.912)
--(0.764, 7.889)
--(0.766, 7.865)
--(0.768, 7.840)
--(0.770, 7.815)
--(0.772, 7.789)
--(0.774, 7.763)
--(0.776, 7.736)
--(0.778, 7.708)
--(0.780, 7.680)
--(0.782, 7.652)
--(0.784, 7.622)
--(0.786, 7.593)
--(0.788, 7.563)
--(0.790, 7.532)
--(0.792, 7.501)
--(0.794, 7.469)
--(0.796, 7.437)
--(0.798, 7.405)
--(0.800, 7.372)
--(0.802, 7.339)
--(0.804, 7.305)
--(0.806, 7.271)
--(0.808, 7.236)
--(0.810, 7.201)
--(0.812, 7.166)
--(0.814, 7.130)
--(0.816, 7.094)
--(0.818, 7.058)
--(0.820, 7.021)
--(0.822, 6.984)
--(0.824, 6.947)
--(0.826, 6.909)
--(0.828, 6.871)
--(0.830, 6.832)
--(0.832, 6.794)
--(0.834, 6.755)
--(0.836, 6.716)
--(0.838, 6.677)
--(0.840, 6.637)
--(0.842, 6.597)
--(0.844, 6.557)
--(0.846, 6.517)
--(0.848, 6.476)
--(0.850, 6.436)
--(0.852, 6.395)
--(0.854, 6.354)
--(0.856, 6.313)
--(0.858, 6.271)
--(0.860, 6.230)
--(0.862, 6.188)
--(0.864, 6.146)
--(0.866, 6.105)
--(0.868, 6.063)
--(0.870, 6.020)
--(0.872, 5.978)
--(0.874, 5.936)
--(0.876, 5.894)
--(0.878, 5.851)
--(0.880, 5.809)
--(0.882, 5.766)
--(0.884, 5.724)
--(0.886, 5.681)
--(0.888, 5.638)
--(0.890, 5.596)
--(0.892, 5.553)
--(0.894, 5.510)
--(0.896, 5.467)
--(0.898, 5.425)
--(0.900, 5.382)
--(0.902, 5.339)
--(0.904, 5.297)
--(0.906, 5.254)
--(0.908, 5.212)
--(0.910, 5.169)
--(0.912, 5.126)
--(0.914, 5.084)
--(0.916, 5.042)
--(0.918, 4.999)
--(0.920, 4.957)
--(0.922, 4.915)
--(0.924, 4.873)
--(0.926, 4.831)
--(0.928, 4.789)
--(0.930, 4.747)
--(0.932, 4.706)
--(0.934, 4.664)
--(0.936, 4.623)
--(0.938, 4.581)
--(0.940, 4.540)
--(0.942, 4.499)
--(0.944, 4.458)
--(0.946, 4.417)
--(0.948, 4.377)
--(0.950, 4.336)
--(0.952, 4.296)
--(0.954, 4.256)
--(0.956, 4.216)
--(0.958, 4.176)
--(0.960, 4.136)
--(0.962, 4.097)
--(0.964, 4.057)
--(0.966, 4.018)
--(0.968, 3.979)
--(0.970, 3.940)
--(0.972, 3.902)
--(0.974, 3.863)
--(0.976, 3.825)
--(0.978, 3.787)
--(0.980, 3.749)
--(0.982, 3.712)
--(0.984, 3.674)
--(0.986, 3.637)
--(0.988, 3.600)
--(0.990, 3.563)
--(0.992, 3.527)
--(0.994, 3.490)
--(0.996, 3.454)
--(0.998, 3.418)
--(1.000, 3.382)
--(1.002, 3.347)
--(1.004, 3.312)
--(1.006, 3.277)
--(1.008, 3.242)
--(1.010, 3.207)
--(1.012, 3.173)
--(1.014, 3.139)
--(1.016, 3.105)
--(1.018, 3.071)
--(1.020, 3.038)
--(1.022, 3.005)
--(1.024, 2.972)
--(1.026, 2.939)
--(1.028, 2.907)
--(1.030, 2.875)
--(1.032, 2.843)
--(1.034, 2.811)
--(1.036, 2.779)
--(1.038, 2.748)
--(1.040, 2.717)
--(1.042, 2.686)
--(1.044, 2.656)
--(1.046, 2.626)
--(1.048, 2.596)
--(1.050, 2.566)
--(1.052, 2.536)
--(1.054, 2.507)
--(1.056, 2.478)
--(1.058, 2.449)
--(1.060, 2.421)
--(1.062, 2.392)
--(1.064, 2.364)
--(1.066, 2.336)
--(1.068, 2.309)
--(1.070, 2.281)
--(1.072, 2.254)
--(1.074, 2.227)
--(1.076, 2.201)
--(1.078, 2.174)
--(1.080, 2.148)
--(1.082, 2.122)
--(1.084, 2.097)
--(1.086, 2.071)
--(1.088, 2.046)
--(1.090, 2.021)
--(1.092, 1.996)
--(1.094, 1.972)
--(1.096, 1.948)
--(1.098, 1.924)
--(1.100, 1.900)
--(1.102, 1.876)
--(1.104, 1.853)
--(1.106, 1.830)
--(1.108, 1.807)
--(1.110, 1.784)
--(1.112, 1.762)
--(1.114, 1.740)
--(1.116, 1.718)
--(1.118, 1.696)
--(1.120, 1.675)
--(1.122, 1.653)
--(1.124, 1.632)
--(1.126, 1.611)
--(1.128, 1.591)
--(1.130, 1.570)
--(1.132, 1.550)
--(1.134, 1.530)
--(1.136, 1.510)
--(1.138, 1.491)
--(1.140, 1.472)
--(1.142, 1.452)
--(1.144, 1.433)
--(1.146, 1.415)
--(1.148, 1.396)
--(1.150, 1.378)
--(1.152, 1.360)
--(1.154, 1.342)
--(1.156, 1.324)
--(1.158, 1.307)
--(1.160, 1.289)
--(1.162, 1.272)
--(1.164, 1.255)
--(1.166, 1.238)
--(1.168, 1.222)
--(1.170, 1.205)
--(1.172, 1.189)
--(1.174, 1.173)
--(1.176, 1.157)
--(1.178, 1.142)
--(1.180, 1.126)
--(1.182, 1.111)
--(1.184, 1.096)
--(1.186, 1.081)
--(1.188, 1.066)
--(1.190, 1.052)
--(1.192, 1.037)
--(1.194, 1.023)
--(1.196, 1.009)
--(1.198, 0.995)
--(1.200, 0.982)
--(1.202, 0.968)
--(1.204, 0.955)
--(1.206, 0.941)
--(1.208, 0.928)
--(1.210, 0.915)
--(1.212, 0.903)
--(1.214, 0.890)
--(1.216, 0.878)
--(1.218, 0.865)
--(1.220, 0.853)
--(1.222, 0.841)
--(1.224, 0.829)
--(1.226, 0.818)
--(1.228, 0.806)
--(1.230, 0.795)
--(1.232, 0.784)
--(1.234, 0.772)
--(1.236, 0.761)
--(1.238, 0.751)
--(1.240, 0.740)
--(1.242, 0.729)
--(1.244, 0.719)
--(1.246, 0.709)
--(1.248, 0.698)
--(1.250, 0.688)
--(1.252, 0.679)
--(1.254, 0.669)
--(1.256, 0.659)
--(1.258, 0.650)
--(1.260, 0.640)
--(1.262, 0.631)
--(1.264, 0.622)
--(1.266, 0.613)
--(1.268, 0.604)
--(1.270, 0.595)
--(1.272, 0.586)
--(1.274, 0.578)
--(1.276, 0.569)
--(1.278, 0.561)
--(1.280, 0.553)
--(1.282, 0.545)
--(1.284, 0.537)
--(1.286, 0.529)
--(1.288, 0.521)
--(1.290, 0.513)
--(1.292, 0.506)
--(1.294, 0.498)
--(1.296, 0.491)
--(1.298, 0.484)
--(1.300, 0.476)
--(1.302, 0.469)
--(1.304, 0.462)
--(1.306, 0.456)
--(1.308, 0.449)
--(1.310, 0.442)
--(1.312, 0.435)
--(1.314, 0.429)
--(1.316, 0.422)
--(1.318, 0.416)
--(1.320, 0.410)
--(1.322, 0.404)
--(1.324, 0.398)
--(1.326, 0.392)
--(1.328, 0.386)
--(1.330, 0.380)
--(1.332, 0.374)
--(1.334, 0.368)
--(1.336, 0.363)
--(1.338, 0.357)
--(1.340, 0.352)
--(1.342, 0.347)
--(1.344, 0.341)
--(1.346, 0.336)
--(1.348, 0.331)
--(1.350, 0.326)
--(1.352, 0.321)
--(1.354, 0.316)
--(1.356, 0.311)
--(1.358, 0.306)
--(1.360, 0.302)
--(1.362, 0.297)
--(1.364, 0.293)
--(1.366, 0.288)
--(1.368, 0.284)
--(1.370, 0.279)
--(1.372, 0.275)
--(1.374, 0.271)
--(1.376, 0.266)
--(1.378, 0.262)
--(1.380, 0.258)
--(1.382, 0.254)
--(1.384, 0.250)
--(1.386, 0.246)
--(1.388, 0.243)
--(1.390, 0.239)
--(1.392, 0.235)
--(1.394, 0.231)
--(1.396, 0.228)
--(1.398, 0.224)
--(1.400, 0.221)
--(1.402, 0.217)
--(1.404, 0.214)
--(1.406, 0.211)
--(1.408, 0.207)
--(1.410, 0.204)
--(1.412, 0.201)
--(1.414, 0.198)
--(1.416, 0.195)
--(1.418, 0.191)
--(1.420, 0.188)
--(1.422, 0.185)
--(1.424, 0.183)
--(1.426, 0.180)
--(1.428, 0.177)
--(1.430, 0.174)
--(1.432, 0.171)
--(1.434, 0.169)
--(1.436, 0.166)
--(1.438, 0.163)
--(1.440, 0.161)
--(1.442, 0.158)
--(1.444, 0.156)
--(1.446, 0.153)
--(1.448, 0.151)
--(1.450, 0.148)
--(1.452, 0.146)
--(1.454, 0.144)
--(1.456, 0.141)
--(1.458, 0.139)
--(1.460, 0.137)
--(1.462, 0.135)
--(1.464, 0.132)
--(1.466, 0.130)
--(1.468, 0.128)
--(1.470, 0.126)
--(1.472, 0.124)
--(1.474, 0.122)
--(1.476, 0.120)
--(1.478, 0.118)
--(1.480, 0.116)
--(1.482, 0.114)
--(1.484, 0.113)
--(1.486, 0.111)
--(1.488, 0.109)
--(1.490, 0.107)
--(1.492, 0.105)
--(1.494, 0.104)
--(1.496, 0.102)
--(1.498, 0.100)
--(1.500, 0.099)
--(1.502, 0.097)
--(1.504, 0.096)
--(1.506, 0.094)
--(1.508, 0.093)
--(1.510, 0.091)
--(1.512, 0.090)
--(1.514, 0.088)
--(1.516, 0.087)
--(1.518, 0.085)
--(1.520, 0.084)
--(1.522, 0.082)
--(1.524, 0.081)
--(1.526, 0.080)
--(1.528, 0.078)
--(1.530, 0.077)
--(1.532, 0.076)
--(1.534, 0.075)
--(1.536, 0.073)
--(1.538, 0.072)
--(1.540, 0.071)
--(1.542, 0.070)
--(1.544, 0.069)
--(1.546, 0.068)
--(1.548, 0.066)
--(1.550, 0.065)
--(1.552, 0.064)
--(1.554, 0.063)
--(1.556, 0.062)
--(1.558, 0.061)
--(1.560, 0.060)
--(1.562, 0.059)
--(1.564, 0.058)
--(1.566, 0.057)
--(1.568, 0.056)
--(1.570, 0.055)
--(1.572, 0.054)
--(1.574, 0.054)
--(1.576, 0.053)
--(1.578, 0.052)
--(1.580, 0.051)
--(1.582, 0.050)
--(1.584, 0.049)
--(1.586, 0.048)
--(1.588, 0.048)
--(1.590, 0.047)
--(1.592, 0.046)
--(1.594, 0.045)
--(1.596, 0.045)
--(1.598, 0.044)
--(1.600, 0.043)
--(1.602, 0.042)
--(1.604, 0.042)
--(1.606, 0.041)
--(1.608, 0.040)
--(1.610, 0.040)
--(1.612, 0.039)
--(1.614, 0.038)
--(1.616, 0.038)
--(1.618, 0.037)
--(1.620, 0.036)
--(1.622, 0.036)
--(1.624, 0.035)
--(1.626, 0.035)
--(1.628, 0.034)
--(1.630, 0.033)
--(1.632, 0.033)
--(1.634, 0.032)
--(1.636, 0.032)
--(1.638, 0.031)
--(1.640, 0.031)
--(1.642, 0.030)
--(1.644, 0.030)
--(1.646, 0.029)
--(1.648, 0.029)
--(1.650, 0.028)
--(1.652, 0.028)
--(1.654, 0.027)
--(1.656, 0.027)
--(1.658, 0.026)
--(1.660, 0.026)
--(1.662, 0.025)
--(1.664, 0.025)
--(1.666, 0.025)
--(1.668, 0.024)
--(1.670, 0.024)
--(1.672, 0.023)
--(1.674, 0.023)
--(1.676, 0.023)
--(1.678, 0.022)
--(1.680, 0.022)
--(1.682, 0.021)
--(1.684, 0.021)
--(1.686, 0.021)
--(1.688, 0.020)
--(1.690, 0.020)
--(1.692, 0.020)
--(1.694, 0.019)
--(1.696, 0.019)
--(1.698, 0.019)
--(1.700, 0.018)
--(1.702, 0.018)
--(1.704, 0.018)
--(1.706, 0.017)
--(1.708, 0.017)
--(1.710, 0.017)
--(1.712, 0.017)
--(1.714, 0.016)
--(1.716, 0.016)
--(1.718, 0.016)
--(1.720, 0.015)
--(1.722, 0.015)
--(1.724, 0.015)
--(1.726, 0.015)
--(1.728, 0.014)
--(1.730, 0.014)
--(1.732, 0.014)
--(1.734, 0.014)
--(1.736, 0.013)
--(1.738, 0.013)
--(1.740, 0.013)
--(1.742, 0.013)
--(1.744, 0.013)
--(1.746, 0.012)
--(1.748, 0.012)
--(1.750, 0.012)
--(1.752, 0.012)
--(1.754, 0.012)
--(1.756, 0.011)
--(1.758, 0.011)
--(1.760, 0.011)
--(1.762, 0.011)
--(1.764, 0.011)
--(1.766, 0.010)
--(1.768, 0.010)
--(1.770, 0.010)
--(1.772, 0.010)
--(1.774, 0.010)
--(1.776, 0.010)
--(1.778, 0.009)
--(1.780, 0.009)
--(1.782, 0.009)
--(1.784, 0.009)
--(1.786, 0.009)
--(1.788, 0.009)
--(1.790, 0.008)
--(1.792, 0.008)
--(1.794, 0.008)
--(1.796, 0.008)
--(1.798, 0.008)
--(1.800, 0.008)
--(1.802, 0.008)
--(1.804, 0.007)
--(1.806, 0.007)
--(1.808, 0.007)
--(1.810, 0.007)
--(1.812, 0.007)
--(1.814, 0.007)
--(1.816, 0.007)
--(1.818, 0.007)
--(1.820, 0.006)
--(1.822, 0.006)
--(1.824, 0.006)
--(1.826, 0.006)
--(1.828, 0.006)
--(1.830, 0.006)
--(1.832, 0.006)
--(1.834, 0.006)
--(1.836, 0.006)
--(1.838, 0.006)
--(1.840, 0.005)
--(1.842, 0.005)
--(1.844, 0.005)
--(1.846, 0.005)
--(1.848, 0.005)
--(1.850, 0.005)
--(1.852, 0.005)
--(1.854, 0.005)
--(1.856, 0.005)
--(1.858, 0.005)
--(1.860, 0.005)
--(1.862, 0.004)
--(1.864, 0.004)
--(1.866, 0.004)
--(1.868, 0.004)
--(1.870, 0.004)
--(1.872, 0.004)
--(1.874, 0.004)
--(1.876, 0.004)
--(1.878, 0.004)
--(1.880, 0.004)
--(1.882, 0.004)
--(1.884, 0.004)
--(1.886, 0.004)
--(1.888, 0.004)
--(1.890, 0.004)
--(1.892, 0.003)
--(1.894, 0.003)
--(1.896, 0.003)
--(1.898, 0.003)
--(1.900, 0.003)
--(1.902, 0.003)
--(1.904, 0.003)
--(1.906, 0.003)
--(1.908, 0.003)
--(1.910, 0.003)
--(1.912, 0.003)
--(1.914, 0.003)
--(1.916, 0.003)
--(1.918, 0.003)
--(1.920, 0.003)
--(1.922, 0.003)
--(1.924, 0.003)
--(1.926, 0.003)
--(1.928, 0.003)
--(1.930, 0.002)
--(1.932, 0.002)
--(1.934, 0.002)
--(1.936, 0.002)
--(1.938, 0.002)
--(1.940, 0.002)
--(1.942, 0.002)
--(1.944, 0.002)
--(1.946, 0.002)
--(1.948, 0.002)
--(1.950, 0.002)
--(1.952, 0.002)
--(1.954, 0.002)
--(1.956, 0.002)
--(1.958, 0.002)
--(1.960, 0.002)
--(1.962, 0.002)
--(1.964, 0.002)
--(1.966, 0.002)
--(1.968, 0.002)
--(1.970, 0.002)
--(1.972, 0.002)
--(1.974, 0.002)
--(1.976, 0.002)
--(1.978, 0.002)
--(1.980, 0.002)
--(1.982, 0.002)
--(1.984, 0.002)
--(1.986, 0.001)
--(1.988, 0.001)
--(1.990, 0.001)
--(1.992, 0.001)
--(1.994, 0.001)
--(1.996, 0.001)
--(1.998, 0.001)
--(2.000, 0.001)
;


\end{tikzpicture}
\end{document}

