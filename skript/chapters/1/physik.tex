%
% physik.tex -- Physikalische Eigenschaften des Klimasystems
%
% (c) 2018 Prof Dr Andreas Müller, Hochschule Rapperswil
%

\section{Physikalische Eigenschaften des Klimasystems\label{section:physik}}
In diesem Abschnitt stellen wir die physikalischen Eigenschaften
aller wesentlicher Komponenten des Klimasystems zusammen.
Dabei geht es zunächst nur darum, die grundlegende Physik in 
Erinnerung zu rufen und die Naturgesetze, die die Wechselwirkungen
zwischen den Komponenten beschreiben.
Auf die Details der mathematischen Modellierung der zukünftigen
Veränderung dieser Grössen werden wir erst später eingehen.

\subsection{Wärme, Konvektion, Kondensation}
Die wohl wichtigste Klima-Grösse ist die Temperatur.
Sie drückt aus, wieviel Energie in Form von Wärme ein Körper enthält.

\subsubsection{Wärmekapazität}
Die spezifische Wärme $C$ gibt an, wie die innere Energie sich bei
einer Temperaturänderung $\Delta T$ verändert:
\[
\Delta E = C\cdot\Delta T.
\]
Der Körper speichert Energie in Form der thermischen Bewegung der
einzelnen Atome.
Schwerere Atome können bei gleicher Bewegungsgeschwindigkeit 
mehr Energie speichern.
Stoffe mit grösserer Dichte können mehr Atome und damit auch mehr
Wärmeenergie in einem kleineren Volumen unterbringen.
Die spezifische Wärmekapazität $c$ gibt an, welche Wärmekapazität
ein Kilogramm eines Stoffes hat.
Ein Körper der Masse $m$ hat also die Wärmekapazität $C=cm$.

\subsubsection{Wärmeleitung}
Herrschen in einem Körper Temperaturunterschiede, ist $T$ nicht mehr
nur eine konstante, sondern eine Funktion der Koordinaten und auch der
Zeit.
Temperaturunterschiede werden sich ausgleichen, indem Energie von
wärmeren zu kälteren Teilen des Körpers fliegt.
Dies geschieht umso schneller, je grösser die Unterschiede sind.
Die Wärmeleitungsgleichung
\begin{equation}
\frac{\partial T}{\partial t}
=
\kappa
\biggl(
\frac{\partial^2}{\partial x^2}
+
\frac{\partial^2}{\partial y^2}
+
\frac{\partial^2}{\partial z^2}
\biggr)
T
\label{skript:waermeleitung}
\end{equation}
beschreibt die Entwicklung der Funktion $T(x,y,z,t)$ an jedem
Ort des Raumes \cite{skript:waermeleitung}.
Der Koeffizient $\kappa$ ist eine Materialkonstante, die beschreibt,
wie schnell sich die Temperaturunterschiede ausgleichen können.
Ist $\kappa=0$, folgt $\partial T/\partial t=0$, die Temperatur 
ändert sich nicht, es findet keine Wärmeleitung statt.

Die rechte Seite von \eqref{skript:waermeleitung} kann mit dem
sogenannten Laplace-Operator gemäss der folgenden Definition 
geschrieben werden.

\begin{definition}
Der Operator
\[
\Delta
=
\frac{\partial^2}{\partial x^2}
+
\frac{\partial^2}{\partial y^2}
+
\frac{\partial^2}{\partial z^2}
\]
heisst der
{\em Laplace-Operator}.
\end{definition}

Die Wärmeleitungsgleichung erhält damit die Form
\begin{equation}
\frac{\partial T}{\partial t}
=
\kappa\Delta T.
\label{skript:waermeleitung2}
\end{equation}

\subsubsection{Konvektion}
Wärmeleitung kann Wärmeenergie nur vergleichsweise langsam transportieren.
Das einleitende Beispiel des Kochtopfs zeigt auch, wie ein effizienterer
Energietransport funktionieren kann.
In der Atmosphäre dehnt sich warme Luft aus.
Dank der geringeren Dichte können warme Luftblasen aufsteigen und damit
Wärme viel effizienter in die obere Atmosphäre transportieren
als dies mit Wärmeleitung möglich wäre.
Dieser Prozess heisst {\em Konvektion} \cite{skript:konvektion}.
\index{Konvektion}%

Wir wollen den Fall eines strömenden Mediums mathematisch etwas genauer
ausarbeiten.
Bewegt sich das Medium mit der Geschwindigkeit $\vec v$, dann ändert sich
die Temperatur des Mediums, welches sich über dem Punkt $P=(x,y,z)$
befindet.
Nach der Zeit $\Delta t$ befindet sich derjenige Teil des Mediums
über dem Punkt $P$, der sich vorher über dem Punkt $P-\Delta t\cdot\vec v$
befand.
Die Temperatur zur Zeit $t+\Delta t$ ist daher
$T(P,t+\Delta t)=T(P-\Delta t,t)$.
Die Temperaturänderung
\begin{align*}
T(P,t+\Delta t)
&=
T(P,t) + (T(P,t+\Delta t)-T(P,t))
=
T(P,t) + T(P-\vec v\Delta t, t)-T(P,t)
\\
\frac{
T(P,t+\Delta t)
-
T(P,t)
}{\Delta t}
&=
\frac{
T(P-\vec v\Delta t, t)-T(P,t)
}{\Delta t}.
\end{align*}
Beim Grenzübergang $\Delta t\to 0$ wird aus der linken Seite die
partielle Ableitung nach $t$.
Die rechte Seite kann mit Hilfe der Kettenregel berechnet weren.
Es wird
\begin{equation}
\frac{\partial T}{\partial t}
=
-
\frac{\partial T}{\partial x} v_x
-
\frac{\partial T}{\partial y} v_y
-
\frac{\partial T}{\partial z} v_z.
\label{skript:advektion1}
\end{equation}
Der Ausdruck auf der rechten Seite kann vektoriell mit der folgenden
Definition etwas eleganter geschrieben werden.

\begin{definition}
Der vektorielle Operator 
\[
\nabla
=
\arraystretch{1.3}
\begin{pmatrix}
\frac{\partial}{\partial x}\\
\frac{\partial}{\partial y}\\
\frac{\partial}{\partial z}
\end{pmatrix}
\]
heisst der {\em Nabla-Operator}.
Der Vektor
\[
\nabla f
=
\arraystretch{1.3}
\begin{pmatrix}
\frac{\partial f}{\partial x}\\
\frac{\partial f}{\partial y}\\
\frac{\partial f}{\partial z}
\end{pmatrix}
=
\operatorname{grad} f
\]
heisst der {\em Gradient} von $f$.
\end{definition}
\index{Gradient}%
\index{Nabla-Operator}%

Die Temperaturänderung in Folge der Strämung 
\eqref{skript:advektion1}
wird 
\begin{equation}
\frac{\partial T}{\partial t}
=
-\vec{v}\cdot\nabla T.
\label{skript:advektion2}
\end{equation}
\index{Advektion}%
Man nennt diese Temperaturänderung durch die Strömung auch
{\em Advektion}.
Die Wärmeleitungsgleichung kann damit zu einem umfassenderen
Modell
\begin{equation}
\frac{\partial T}{\partial t}
=
-\vec{v}\cdot\nabla T +\kappa\Delta T
\label{skript:waermeleitungadvektion}
\end{equation}
zusammengefasst werden.
Es ist geeignet für die Beschreibung sowohl der Atmosphäre wie auch des
Wärmeaustausches in den Ozeanen.

\subsubsection{Phasenübergänge}
Um ein Kilogramm Wasser bei $20^\circ\text{C}$ zu verdunsten, ist eine
latente Wärme von $2480\,\text{kJ}$ nötig.
Um ein Kilogramm Luft um ein Grad zu erwärmen, sind dagegen nur
$1.005\,\text{kJ}$ notwendig.
Anders herum bedeutet dies, dass eine mit Wasserdampf angereicherte Atmosphäre
sehr viel mehr Energie in Form von latenter Wärme speichern kann, als
allein durch die Wärmekapazität trockener Luft möglich wäre.

Wir haben damit zwei Mechanismen identifiziert, wie eingestrahlte
und in der Erdkruste als Wärme gespeicherte Energie in die Atmosphäre 
transportiert werden kann.
Einerseits kann Luft über aufgewärmten Landmassen oder dem Meer erwärmt
werden und als Konvektionsströmung aufsteigen.
Andererseits kann Wasser an der Oberfläche verdampft werden damit die
latente Wärme in die Atmosphäre übergehen.
Man nennt diese Mechanismen auch turbulente Flüsse
(\cite[S.~70]{skript:wiefunktioniertdas}).

Der Wassergehalt der Luft kann höchstens einige wenige Prozente betragen.
Zwar ist die Wärmespeicherung durch Verdunstung über 2000 mal effizienter,
aber weil nur wenig Wasser dafür zur Verfügung steht, übernimmt die Verdunstung
doch nicht einen derart grossen Teil des Energietransports von der
Oberfläche in die Atmosphäre.
In der Tat finden etwa 30\% des Energietransports von der Erdkruste
in die Atmosphäre durch turbulente Flüsse statt, davon etwa
7\% durch Konvektion und 23\% durch latente Wärme.
(\cite[S.~70]{skript:wiefunktioniertdas}).
Höhere Temperaturen begünstigen die Verdunstung und verschieben diesen
Anteil zugunsten der latenten Wärme.
Man darf also davon ausgehen, dass höhere Oberflächentemperaturen
zu einem überproportional höheren Energietransport in die Atmosphäre
führen.

In der Atmosphäre kann die Energie über grosse Distanzen transportiert
und später wieder freigesetzt werden, wie Hurricanes und Tornadoes
eindrücklich demonstrieren können.
Damit ein Klimamodell Aussagen machen kann über das Auftreten von
extremen Wetterphänomenen muss es also den Wassergehalt der
Atmosphäre modellieren.

\subsection{Strahlung}

\subsection{Erdrotation und Zirkulation}

\subsection{Periodische Einflüsse}

