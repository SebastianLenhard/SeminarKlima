%
% main.tex -- Paper zum Thema <thema>
%
% (c) 2018 Hochschule Rapperswil
%
\chapter{Achsneigung und Eiszeiten\label{chapter:neigung}}
\lhead{Achseneigung und Eiszeiten}
\begin{refsection}
\chapterauthor{Sebastian Lenhard}

\section{Einführung}\label{sec:einf} \rhead{Einführung}
Das Energie Haushaltsmodell wird verwendet um die Erdtemperatur theoretisch zu erklären. Dabei wird die Erde als Kugel betrachtet, die konstant von der Sonne angestrahlt wird. Die eingestrahlte Energie führt zur Erwärmung der Erde. Die Erde ihrerseits strahlt ebenfalls Energie ab, wie jedes bestrahlte Objekt. Dabei wird die Erde als schwarze Kugel modelliert, wodurch die Abstrahlung mit der Stephan Bolzmann Gesetz modelliert werden kann. Die Änderung der Erdtemperatur kann somit mit der Gleichung 

\begin{eqnarray}
\label{eq1}
C \frac{d T}{d t} = (1-\alpha(T)) Q- \varepsilon \sigma T^4
\end{eqnarray}
modelliert werden. Die linke Seite der Gleichung entsprich der Veränderung der Erdtemperatur, wobei $C$ der Wärmekapazität entspricht, $T$ die Temperatur in Kelvin und $t$ die Zeit darstellt. Die Einstrahlung $Q$ wird dabei nicht vollständig von der Erde absorbiert, ein Teil der Energieeinstrahlung wird reflektiert und hängt von der Albedo $\alpha(T)$ ab, wobei die Albedo ihrerseits Temperaturabhängig ist. Je kälter die Erde ist, desto höher ist der Albedo der Erde, da hellere Flächen wie Eis weniger Energie aufnehmen als dunkle Flächen. Dieser temperaturabhängige Albedo kann durch dich Gleichung 

\begin{eqnarray} 
\alpha(T) = 0.5 - 0.2 \tanh \left( \frac{T-265}{10} \right) 
\end{eqnarray}
dargestellt werden. Dadurch ist der Albedo symmetrisch um 265 K und kann maximal zwischen 0.3 und 0.7 schwanken. Die Gleichgewichtstemperatur ist dabei erreicht, wenn sich die Temperatur über die Zeit nicht mehr ändert. Abbildung \ref{fig:abb1} die Einstrahlung und Ausstrahlung je nach Temperatur. Schnittpunkte sind Gleichgewichtstemperaturen. Der mittlere Schnittpunkt ist dabei ein instabiles Gleichgewicht, da mit einer minimal höheren Temperatur die Einstrahlung höher ist als die Ausstrahlung, wodurch sich die Erde erwärmt und in das stabile warme (rechte) Gleichgewicht bewegt. Die selbe Argumentation gilt mit einer minimal tieferen Temperatur, wodurch die Erde sich auf das kalte (linke) Gleichgewicht zubewegt. Die ausführliche Herleitung befindet sich in Kapitel 5.
%
%ABBILDUNG1
\begin{figure}
	\centering
	\includegraphics[width= 0.8\textwidth]{Strahlung_1.png}
	\caption[Gleichgewichtstemperatur]{Gleichgewichtstemperatur}
	\label{fig:abb1}
\end{figure}
%
Die Neigung der Erdachse bewirkt die Jahreszeiten. Der genaue Zusammenhang wird in Abschnitt \ref{sec:winkel} und \ref{sec:neigung} erläutert. Je grösser der Neigungswinkel, desto extremer sind die Jahreszeiten. Durch die Jahreszeiten kommt es zu Schwankungen der Gleichgewichtstemperatur. Dies wird in Kapitel \ref{sec:bi} anhand einer Simulation gezeigt. Starke Schwankungen können dazu führen, dass das Erdklima vom warmen ins kalte Gleichgewicht wechselt und eine Eiszeit entsteht. Da die Jahreszeiten auf der Nordhalbkugel und Südhalbkugel jeweils entgegengesetzt zueinander auftreten wird der Fokus auf eine Halbkugel gelegt (ohne Energieaustausch). In Abschnitt \ref{sec:math} wird eine Halbkugel mathematisch modelliert und das Klima wird unter verschiedenen Erdachsenneigungen prognostiziert. Abschnitt\ref{sec:schluss} fasst die wichtigsten Erkenntnisse nochmals zusammen und gibt einen Ausblick für weitere Erweiterungen des Modells.

\newpage
\subsection{Milankovi\'c Zyklen}\label{sec:mil} \rhead{Milankovi\'c Zyklen}
In diesem Abschnitt wird eine kurze empirisch bestätigter Ausblick über das Verhalten der Erde gegeben. Der serbische Mathematiker Milutin Milankovi\'c hat Mitte des 20. Jahrhunderts folgende Entdeckung gemacht: Die Erde bewegt sich in verschiedenen Zyklen auf ihrer Umlaufbahn. Zudem führen die Planeten wie Saturn dazu, dass es zu Schwankungen der Achsenneigung kommen kann. Milakovi\'c unterteilt die Effekte in drei Zyklen. Die Milankovi\'c Zyklen wurden empirisch bestätigt.

\subsubsection{Präzession}
Der erste Zyklus wird Präzession genannt und hat eine länge von 23'000 Jahren.Diser Zyklus wird durch eine Bewegung der Erdachse und eine Bewegung der Ellipse der Erdbahn hervorgerufen. Durch die Anziehungskraft von Sonne und Mond aus gehend, wird die Erdachse in ein Taumeln gebracht, wie ein aus dem Gleichgewicht geratener Kreisel. Dagegen wirkt eine zweite Kraft, die durch den Einfluß aller Planeten unseres Sonnensystems auf die Umlaufbahn der Erde einfluss nimmt. Beide Bewegungen zusammen bewirken, dass sich die Jahreszeiten verschieben. Dieser Effekt wird auch als Perihelwanderung bezeichnet.\footnote{Als Perihel wird der Punkt mit dem kleinsten Abstand auf der Erdumlaufbahn zur Sonne bezeichnet.}
%
%ABBILDUNG2
\begin{figure}
	\centering
	\includegraphics[width= 0.3\textwidth]{Precession.png}
	\caption[Präzession]{Präzession}
	\label{fig:abb}
\end{figure}

\subsubsection{Exzentrizität}
Der zweite Zyklus ist die Exzentrizität und hat eine Dauer von 100'000 Jahren. Wie Abbildung \ref{fig:abb2} zeigt, kreist die Erde um die Sonne. Die Kreisbahn wird allerdings durch den Einfluss der anderen Planeten verzerrt und zu einer Ellipse geformt. Je nach Planetenkonstelation ist daher die Umlaufbahn eher Kreisförmig oder nicht. Dieser Effekt ist nur zur vollständigkeitshalber hier aufgeführt und wird in dem weiteren Verlauf der Arbeit nicht weiter beachtet. Es wird vereinfacht angenommen, dass die Umlaufbahn der Erde einem Kreis entspricht. 
%
%ABBILDUNG2
\begin{figure}
	\centering
	\includegraphics[width= 0.8\textwidth]{Eccentricity.png}
	\caption[Exzentrizität]{Exzentrizität}
	\label{fig:abb2}
\end{figure}

\subsubsection{Ekliptikschiefe}
Der letzte Zyklus ist die Ekliptikschiefe und hat einen Rythmus von 41'000 Jahren. Dabei bewegt sich die Achsenneigung durch den Einfluss der Planeten. Abbildung \ref{fig:abb3} gibt einen Überblick über das geschehen. Dieser Zyklus motiviert die folgende Arbeit, die den Zusammenhang zwischen Klima und dem Winkel der Erdachse untersucht. Die Abbildungen stammen aus \cite{fm}.
%
%ABBILDUNG3
\begin{figure}
	\centering
	\includegraphics[width= 0.3\textwidth]{Obliquity.png}
	\caption[Ekliptikschiefe]{Ekliptikschiefe}
	\label{fig:abb3}
\end{figure}
%
\newpage

\section{Einstrahlungswinkel}\label{sec:winkel} \rhead{Einstrahlungswinkel}
Da die Erde einen sehr grossen Abstand und nur eine geringe relative Grösse zur Sonne hat, mit 1:109, wird vereinfacht angenommen, dass alle Sonnenstrahlen im rechten Winkel auf die Erdkugel treffen. Nehmen wir zudem vereinfacht an, dass der Neigungswinkel der Erde $0^\circ$ . Dies bedeutet, dass die Sonnenstrahlen am Äquator mit einem Winkel von $90^\circ$ auf die Erdoberfläche treffen. Durch die Kugelform der Erde nimmt der Einstrahlungswinkel mit der Distanz zum Äquator ab. Dies ist in Abbildung \ref{fig:abb5} schematisch dargestellt. 
%
%ABBILDUNG5
\begin{figure}
	\centering
	\includegraphics[width= 0.8\textwidth]{Schatten.png}
	\caption[Einstrahlungswinkel]{Einstrahlungswinkel}
	\label{fig:abb5}
\end{figure}
In der Abbildung kommen die Sonnenstrahlen von der linken Seite. Beispielsweise ist die angestrahlte Fläche 1$m^2$, zweidimensional dargestellt als 1$m$ in Abbildung \ref{fig:abb5}. Nahe den Polen wird die selbe Flächengrösse von weniger Energie bestrahlt, während am Äquator eine viel grössere Menge an Energie auf die selbe Fläche trifft. Die eingestrahlte Energiemenge kann als Schattenfläche aufgefasst werden. In Abbildung \ref{fig:abb5} ist ersichtlich, dass die Energiemenge pro Fläche am Äquator deshlab erheblich grösser ist als an den Polen. Der Einstrahlungswinkel beeinflusst somit wieviel Energie auftrifft und somit aufgenommen wird. Daraus folgt, dass es am Äquator wärmer ist als an den Polen. 
In diesem Gedankenexperiment wird somit der Äquator konstant in einem Winkel von $90^\circ$ bestrahlt, während die Pole direkte Strahlung empfangen. Im Verlaufe eines Jahres legt die Erde eine Umkreisung der Sonne zurück. Vereinfacht wird eine Kreisbahn angenommen. Mit einem Neigungswinkel von $0^\circ$ ist die Einstrahlung entlang der Umlaufbahn konstant für jeden Punkt auf der Erde. Im folgenden Kapitel wird die Annahme des Neigungswinkel mit $0^\circ$ gelockert und erläutert wie dadurch die Jahreszeiten entstehen.
\newpage

\section{Achsenneigung}\label{sec:neigung} \rhead{Achsenneigung}
Durch die Neigung der Erdachse kommt es zu der Entstehung der Jahreszeiten. Dies wird anhand der Abbildung \ref{fig:abb6} intuitiv erklärt. 
%
%ABBILDUNG6
\begin{figure}
	\centering
	\includegraphics[width= 0.8\textwidth]{tagundnacht.png}
	\caption[Achsenneigung]{Achsenneigung}
	\label{fig:abb6}
\end{figure}
%
Die linke Erdkugel in Abbildung \ref{fig:abb6} zeigt die Situation mit einem Neigungswinkel von $0^\circ$. Die Mittlere Linie entspricht der Breitengrade des Äquators. Wie in Kapitel \ref{sec:winkel} erläutert ist somit die Einstrahlung am grössten am Äquator, da die Sonnenstrahlen in einem Winkel von $90^\circ$ auf die Erdoberfläche auftreffen. Die rechte Erdkugel illustriert nun einen Neigungswinkel. Dadurch wird ersichtlich, dass der Äquator nicht länger der Breitengrad ist, in dem die Sonnenstrahlen in einem Winkel von $90^\circ$ eintreffen. In der Abbildung \ref{fig:abb6} entspricht dies nun der Linie des Wendekreises. In der Abbildung kommen die Sonnenstrahlen von der rechten Seite, somit entsprich die gelbe Fläche der angestrahlten Fläche der Erdkugel. Es ist ersichtlich, dass am Nordpol dadurch Sonnenstrahlen ankommen, während der Südpol nicht bestrahlt wird. Im Verlaufe eines Tages dreht sich die Erde einmal um die eigene Achse. In der rechten Abbildung bedeutet das, dass am Nordpol keine Nacht stattfindet, während am Südpol ewige Nacht herrscht. Am Äquator dauert ein Tag genau 12 Stunden, während auf der Nordhalbkugel die Tage länger sind als auf der Südhalbkugel. Dies entsprich der Situation Sommer auf der Nordhalbkugel, respektive Winter auf der Südhalbkugel. Die Abbildung stammt aus \cite{fa}.
Auf ihrer Umlaufbahn um die Sonne bleibt die Erdachse konstant geneigt. Dies bedeutet, dass im Nordwinter, die Situation entsteht, in der in Abbildung \ref{fig:abb6} die gelbe Fläche bestrahlt wird während die blaue Fläche nicht bestrahlt wird. Somit am Nordpol ewige Nacht herrscht und am Südpol die Sonne nie untergeht. Im Frühling und im Herbst ist die Strahlungssituation in der linken Seite abgebildet. Diese kontinuierlichen Übergänge von heissem Sommer zu kaltem Winter werden im nächsten Kapitel \ref{sec:math} mathematisch modelliert. Dabei wird der Fokus auf eine Halbkugel gelegt, da die Jahreszeiten halbkugelspezifisch auftreten. 
\newpage

\section{Mathematischer Zusammenhang}\label{sec:math} \rhead{Mathematischer Zusammenhang}
Motiviert durch die vorhergegangenen zwei Kapitel wird das Energiehaushaltsmodell aus Kapitel \ref{sec:einf} erweitert. Im folgenden wird zuerst in Kapitel \ref{sec:einst} erläutert, wie die zyklisch ändernde Einstrahlung mathematisch modelliert wird, abhängig von der Achsenneigung. In Kapitel \ref{sec:bi} werden die Gleichgewichte als Bifurkation zusammengefasst. Kapitel \ref{sec:sim1} wird eine Simulation gerechnet und präsentiert. Da die Simulation zeigt, dass dadurch keine Eiszeit entstehen kann, wird in Kapitel \ref{sec:co} die Coalbedo nicht linear transformiert. Die Simulation wird mit verändertet Coalbedo nochmals gerechnet in Kapitel \ref{sim2}. Dadurch wird gezeigt, dass es zu einer Eiszeit kommen kann. 


\subsection{Einstrahlung} \label{sec:einst} \rhead{Einstrahlung}
Die Einstrahlung auf eine Halbkugel schwankt zyklisch im Rhythmus eines Jahres. Anstelle der konstanten Einstrahlung $Q$ im Energie Haushaltsmodell in Kapitel \ref{sec:einf} wird nun eine Zeit abhängige Einstrahlung modelliert. Gegeben der dem Neigungswinkel $\omega$ Schwankt die Einstrahlung um die Solarkonstante $Q$. $t$ ist die Zeit, die anhand des $\sin(.)$ die Umlaufbahn als Kreisbahn der Erde um die Sonne modelliert. Je grösser der Neigungswinkel $\omega$, desto extremer sind die Schwankungen. Dies wird Anhand der Funktion 

\begin{eqnarray*} 
I(t, \omega) = Q+\frac{\sin(\omega)\sin(t)}{\sqrt{1+(\sin(\omega) \sin(t))^2}}Q
\end{eqnarray*}
ermöglicht.  Mit einem Neigungswinkel von $\omega=0$ entspricht das Modell wieder dem ursprünglichen Energie Haushaltsmodell. Somit wird die Gleichung \eqref{eq1} modifiziert zu:

\begin{eqnarray} \label{eq2}
C \frac{d T}{d t} = (1-\alpha(T)) I(t, \omega) - \varepsilon \sigma T^4
\end{eqnarray}
Anstelle der Solarkonstanten $Q$ stehet in Gleichung \eqref{eq2} nun die Zeit und winkelabhängige Einstrahlung $I(t,\omega)$. Dadurch entstehen verschieden Gleichgewichte, die vom jeweiligen Zeitpunkt und gegeben dem Winkel abhängen. Im Sommer hat die Funktion $I(t,\omega)$ ein maximum gegeben $\omega$, während im Winter jeweils ein Minimum erreicht ist. Die verschiedenen Einstrahlungen sind in Abbildung \ref{fig:abb7} dargestellt. 
%
%ABBILDUNG7
\begin{figure}
	\centering
	\includegraphics[width= 0.8\textwidth]{Strahlung_2.png}
	\caption[Gleichgewichtstemperatur]{Gleichgewichtstemperatur}
	\label{fig:abb7}
\end{figure}
Durch die Neigung und die mit ihr einherkommende zyklische Schwankung der Einstrahlung entstehen somit verschieden Gleichgewichte. Über ein Jahr bewegt sich die Einstrahlungsfunktion von der violetten (Winter) Linie kontinuierlich über die blaue (Frühling) zur gelben (Sommer) und wieder über die blaue (Herbst) zurück. Eine Möglichkeit dies in einem übersichtlicheren Format darzustellen bietet die Bifurkation. Diese wird im nächsten Abschnitt \ref{sec:bi} erläutert. 


\subsection{Mögliche Eiszeit} \label{sec:bi} \rhead{Mögliche Eiszeit}
Die Bifurkation zeigt wie sich die Lösungen einer Differentialgleichung verändern, gegeben der Änderung eines Parameters. Die betrachtete Differenzialgleichung \eqref{eq1} aus dem Ursprünglichen Energie Haushaltsmodell, indem nun die Einstrahlung $Q$ verändert wird. Dies ist äquivalent zur Differenzialgleichung \eqref{eq2}.
%
%ABBILDUNG8
\begin{figure}
	\centering
	\includegraphics[width= 0.8\textwidth]{Bifurkation.png}
	\caption[Bifurkation]{Bifurkation}
	\label{fig:abb8}
\end{figure}
%
Betrachtete man in Abbildung \ref{fig:abb7} das warme Gleichgewicht im Winter und erhöht die Strahlung, so erhöht sich die Gleichgewichtstemperatur. Dies entsprich in Abbildung \ref{fig:abb8} der ausgezogenen Linie im oberen drittel, die einen positiven Zusammenhang zwischen Einstrahlungsintensität und Gleichgewichtstemperatur aufweist. Senkt man die Einstrahlung, so ist in Abbildung \ref{fig:abb7} zu erkennen, dass das warme Gleichgewicht und das instabile Gleichgewicht auf den selben Punkt fallen indem sich Einstrahlung und Ausstrahlung tangieren. Dies entspricht in Abbildung \ref{fig:abb8} dem Zusammentreffen mit der gestrichelten Linie ($T_{fl}$). Eine weitere Minderung der Einstrahlung hat zur folge, dass das warme Gleichgewicht nicht mehr existiert, sondern nur noch das kalte Gleichgewicht.
Die gestrichelte Linie entspricht dem instabilen Gleichgewicht in Abbildung \ref{fig:abb7}, welches einen negativen Zusammenhang zwischen Einstrahlung und Gleichgewichtstemperatur hat. In dem Punkt $T_{lf}$, wo sich das instabile und kalte Gleichgewicht treffen, ist in Abbildung \ref{fig:abb7} der fall, wenn die Einstrahlung so hoch ist, dass das kalte Gleichgewicht mit dem instabilen Gleichgewicht zusammefällt im Punkt wo sich Einstrahlung und Ausstrahlung tangieren. eine weitere Erhöhung der Einstrahlung führt dazu, dass das kalte Gleichgewicht nicht mehr existiert.
Wie die Bifurkation zeigt, existieren für gewisse Einstrahlungen drei Gleichgewichte (warmes, instabiles und kaltes) wie in Abbildung \ref{fig:abb7} dargestelt. Für extrem hohe Einstrahlungen exitiert lediglich das warme Gleichgewicht während für tiefere Einstrahlungen lediglich das kalte Gleichgewicht existiert. Die Erde befindet sich momentan in ihrem warmen Gleichgewicht. Durch die zyklische Einstrahlung schwankt die Gleichgewichtstemperatur in der oberen ausgezogenen Linie in Abbildung \ref{fig:abb8}. Durch einen extrem kalten Winter beziehungsweise durch eine tiefe Einstrahlung kann es sein, dass die Erde in das kalte Gleichgewicht fällt. Dadurch würde sie in Zukunft auf der unteren ausgezognen Linie schwanken. Um eine solche Situation genauer zu untersuchen wird im nächsten Kapitel eine Simulation berechnet. 


\subsection{Simulation I} \label{sec:sim1} \rhead{Simulation I}
Durch die Änderung des Neigungswinkels gemäss den Milankovi\'c Zyklen kann es zu einer möglichen Eiszeit kommen. Dies wird im Folgenden anhand einer Simulation versucht darzustellen. Dabei wird die Differenzialgleichung \eqref{eq3} gelöst. Es werden verschiedene Neigungswinkel $\omega$ verwendet und die Gleichgewichtstemperatur über die Zeit projezieren. Abbildung \ref{fig:abb9} zeigt das Ergebnis der Simulationen. 
%
%ABBILDUNG9
\begin{figure}
	\centering
	\includegraphics[width= 0.8\textwidth]{Zeitachse_0.png}
	\caption[Simulation I]{Simulation I}
	\label{fig:abb9}
\end{figure}
%
Als blaue Linie ist dargestellt wie das ursprüngliche Energiehaushaltsmodell aus Kapitel \ref{sec:einf} mit einem Neigungswinkel $\omega=0$ aussieht. Da bei ist ersichtlich, dass das System unterhalb der Gleichgewichtstemperatur startet und nach dem ersten Sommer das konstante Gleichgewicht erreicht. In Orange und Gelb ist ein grösserer Neigungswinkel simuliert. Es ist ersichtlich, dass das System im warmen Gleichgewicht bleibt und um diese herum schwankt. Im violetten Fall ist der Neigungswinkel noch grösser, was dazu führt , dass das System im ersten Winter sofort in das kalte Gleichgewicht wechselt und dann in dem kalten Gleichgewicht schwankt, wie erklärt in Abschnitt \ref{sec:bi}. Die Erde wechselt von dem warmen in das kalte Gleichgewicht, es bildet sich eine Eiszeit in der es weiterhin Sommer und Winter gibt, jedoch um ca. $60 ^\circ$ Kelvin tiefer. 
In allen Situationen entscheidet sich das Gleichgewicht bereits nach der ersten Periode, beziehungsweise im ersten Jahr. In keinem der Fälle wird sich durch einen längeren Zeitraum ein weiterer Gleichgewichtswechsel stattfinden. Um eine Transformation zum kalten Gleichgewicht zu modellieren muss die Albedo beziehungsweise Coalbedo angepasst werden. 


\subsection{Coalbedo} \label{sec:co} \rhead{Coalbedo}
Die temperaturabhängige Albedo ist in Gleichung \eqref{eq2} gegeben und gibt an wieviel Energie reflektiert wird. Diese ist symmetrisch um $265 K$ und beschränkt zwischen $0.3$ und $0.7$ und reagiert gleichermassen auf Temperaturerhöhungen wie auf Senkungen. Tiefere Temperaturen führen zu einer erhöhten Energiereflexion, da sich grössere Eisflächen bilden und diese die Sonnenstrahlen stärker reflektieren als die braune oder grüne Erde. Die Coalbedo entspricht dem Wert $1-\alpha(T)$ und gibt an wieviel Energie von dem Planeten aufgenommen wird. Mit der Funktion in Gleichung \ref{eq4} wird die Coalbedo asymmetrisch transformiert. 

\begin{eqnarray} \label{eq4}
g(x)=x-a(x-\frac{1}{2})^2
\end{eqnarray}
Input $x$ in der Funktion ist die ursprüngliche symmetrische Coalbedo $1-\alpha(T)$. In Abbildung \ref{fig:abb10} ist die Transofmationsfunktion für verschiedene Parameter $a$ dargestellt. Mit $a=0$ bleibt der lineare Zusammenhang bestehen, während $a>0$ nun asymmetrische Transformationen erlaubt. Die Transformation führt zu einer tieferen Coalbedo für hohe beziehungweise tiefe Temperaturen. Bei einer Temperatur von $265 K$ ist die Coalbedo wiederum $0.5$, wie im ursprünglichen Modell. Dies bedeutet, dass die Coalbedo stärker reagiert für tiefere Temperaturen während die Reaktion bei höheren Temperaturen verlangsamt wird. Dies kann motiviert werden durch das selbe Argument wie zu Beginn des Kapitels. Das Eis reflektiert die Sonnenstrahlen stärker als die brauen oder grüne Erde. Ist das Eis jedoch geschmolzen, so ist der Effekt einer weiteren Temperaturerhöhung geringer als der Effekt vor der Eisschmelze. 
%
%ABBILDUNG10
\begin{figure}
	\centering
	\includegraphics[width= 0.8\textwidth]{Funktion.png}
	\caption[Asymmetrie in der Coalbedo]{Asymmetrie in der Coalbedo}
	\label{fig:abb10}
\end{figure}
%
Durch diese Transformation wird die Coalbedo zu:

\begin{eqnarray*}
\kappa(T)=(1-\alpha(T)) - a \left( 1-\alpha(T) - \frac{1}{2} \right)^2
\end{eqnarray*}
Diese hängt wiederum von der Temperatur $T$ ab, ist nun aber asymmetrisch durch den Parameter $a$ verzerrt. Das Energie Haushaltsmodell kann nun mit dieser Funktion der Coalbedo erweitert werden. Somit wird das Modell aus Gleichung \ref{eq2} durch die folgende Differentialgleichung beschrieben. 

\begin{eqnarray} \label{eq5}
C \frac{d T}{d t} =  \kappa(T) I(t, \omega) - \varepsilon \sigma T^4
\end{eqnarray}
Wiederum wird die Gleichgewichtstemperatur ermittelt indem sich die Temperatur über die Zeit für die gegebenen Parameter nicht ändert $\frac{d T}{d t}=0$, beziehungsweise die Energie Einstrahlung der Ausstrahlung der Erde entspricht. Abbildung \ref{fig:abb11} zeigt die Einstrahlung abhängig von der Temperatur, sowie die Ausstrahlung abhängig von der Temperatur. Diese Abbildung kann mit Abbildung \ref{fig:abb7} verglichen werden. Die Ausstrahlung bleibt die selbe, weiterhin als schwarze Kugel mit der Stephan Boltzmann Konstante modelliert. Die Einstrahlung hat sich nun verändert, und ist nun geringer für Temperaturen $T \neq 265 K$ verglichen zu Abbildung \ref{fig:abb7}. Dies als Resultat der transformation der Coalbedo, da diese für $a>0$ tiefer ist, wie im obigen Abschnitt erklärt. Die zugehörige Bifurkation sieht schematisch gleich aus wie in Abbildung \ref{fig:abb8} beschrieben. 
%
%ABBILDUNG11
\begin{figure}
	\centering
	\includegraphics[width= 0.8\textwidth]{Strahlung_3.png}
	\caption[Gleichgewichtstemperatur]{Gleichgewichtstemperatur}
	\label{fig:abb11}
\end{figure}
%

\subsection{Simulation II} \label{sim2} \rhead{Simulation II}
In diesem Kapitel wird erneut eine Simulation gerechnet, welche nun im Vergleich zu Kapitel \ref{sec:sim1} die asymmetrische Coalbedo verwendet. Wiederum werden vier verschiedene Neigungswinkel simuliert. Zudem werden drei verschiedene Werte für den Parameter $a$, welcher die Stärke der Asymmetrie bestimmt angenommen. Die Resultate sind in Abbildung \ref{fig:abb12}-\ref{fig:abb14} dargestellt und können mit Abbildung \ref{fig:abb9} verglichen werden. In Abbildung \ref{fig:abb9} ist der Parameter $a=0$, während er für die restlichen Abbildungen jeweils erhöht wird. In allen drei folgenden Abbildungen wird klar, dass nun ein Sprung in das kalte Gleichgewicht möglich ist über die Zeit. 
%
%ABBILDUNG12
\begin{figure}
	\centering
	\includegraphics[width= 0.8\textwidth]{Zeitachse_1.png}
	\caption[Simulation II a]{Simulation II a}
	\label{fig:abb12}
\end{figure}
%
Als blaue Linie ist wiederum der ursprüngliche Fall mit einem Neigungswinkel von $\omega=0$ dargestellt. Da zu Beginn der Simulation die Gleichgewichtstemperatur noch nicht erreicht ist, erwärmt sich die Erde und bleibt konstant im warmen Gleichgewicht. Durch die Achsenneigung kommt es in dem orangen Fall zu zyklischer Einstrahlung und somit zu einer Schwankenden Temperatur. Die Temperaturschwankung ist allerdings zu gering, als das ein Sprung in das kalte Gleichgewicht stattfinden könnte. 

Im gelben Fall ist der Neigungswinkel höher als im orangen Fall. Es kommt zu stärkeren Schwankungen wobei ersichtlich ist, dass diese asymmetrische nach untern ausreissen. Es wird kälter über die  Zeit, jeweils im Sommer wie auch im Winter. Bis zum moment, indem ein solch kalter Winter erreicht ist, indem das warme Gleichgewicht nicht mehr existiert, somit findet ein Sprung in das kalte Gleichgewicht statt. Es ist ebenfalls erkennbar, dass die Schwankungen im kalten Gleichgewicht geringe ausfallen als im warmen. Dies kommt durch die Asymmetrie in der Coalbedo.

Im violetten Fall ist die Achsenneigung so gross, dass bereits der erste Winter so kalt ist, dass die Erde ins kalte Gleichgewicht wechselt. Es ist auch gut zu erkennen, dass die Schwankungen stärker sind als in dem gelben Fall im kalten Gleichgewicht. Dies aufgrund der stärkeren Achsenneigung, die extremere Jahreszeiten zu folge hat. 

In Abbildung \ref{fig:abb13} und \ref{fig:abb14} ist ersichtlich, wie eine stärkere Asymmetrie in der Coalbedo den Effekt der Temperatursenkung verstärkt und somit schneller eine Eiszeit erreicht wird. Dabei gelten die gleichen Effekte wie im obigen Abschnitt beschrieben. 
%
%ABBILDUNG13
\begin{figure}
	\centering
	\includegraphics[width= 0.8\textwidth]{Zeitachse_2.png}
	\caption[Simulation II b]{Simulation II b}
	\label{fig:abb13}
\end{figure}
%
%
%ABBILDUNG14
\begin{figure}
	\centering
	\includegraphics[width= 0.8\textwidth]{Zeitachse_3.png}
	\caption[Simulation II c]{Simulation II c}
	\label{fig:abb14}
\end{figure}
%
\newpage

\section{Schlussfolgerung} \label{sec:schluss} \rhead{Schlussfolgerung}
Die Neigung der Erdachse bewirkt die Jahreszeiten. Je stärker die Erdachse geneigt ist, desto extremer Fallen diese aus. In dieser Arbeit wurde gezeigt, wie dies mathematisch für eine Halbkugel modelliert werden kann. Zudem wurde gezeigt, dass für die Erde zwei Gleichgewichtstemperaturen bestehen. Durch einen extrem kalten Winter kann es dazu kommen, dass die Erde von der warmen Gleichgewichtstemperatur in die kalte Gleichgewichtstemperatur wechselt. Dies würde zu einer Eiszeit führen. Um aus dieser Eiszeit zu entkommen, wäre eine extreme Erwärmung der Erde notwendig, wie sie möglicherweise durch Vulkanausbrüche stattfinden könnte. 
Das mathematische Modell welches in dieser Arbeit verwendet wurde fokussiert sich dabei auf eine Halbkugel. Es findet kein Energieaustausch statt, die Jahreszeiten sind jeweils entgegengesetzt zwischen den Erdhalbkugeln, womit es für die durchschnittliche Erdtemperatur nur geringe Auswirkungen hat. Zudem ist die Ausstrahlung der Erde weiterhin als schwarze Kugel modelliert. 
Trotz dieser Vereinfachungen zeigt die Simulation wie es zu einer Eiszeit kommen kann. Ein wichtiger Punkt dabei ist die Asymmetrie der Coalbedo, wobei die Energieaufnahme der Erde bei tieferen Temperaturen stärker reagiert als bei hohen Temperaturen. Ohne diese Asymmetrie entscheidet sich die Art des Gleichgewichtes bereits in der ersten Periode. 
Es sei zum Schluss angemerkt, dass ein Zusammenspiel der Milankovi\'c Zyklen dazu führen kann, dass ein kälterer Winter auftritt. Durch die Exzentrizität kann es zu einer weiteren Distanz zwischen Erde und Sonne kommen. Verbunden mit der Ekliptikschiefe, welche zu einem höheren Neigungswinkel führt, kann somit eine Eiszeit entstehen. 


\printbibliography[heading=subbibliography]
\end{refsection}
