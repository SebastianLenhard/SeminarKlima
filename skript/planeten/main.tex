%
% main.tex -- Paper zum Thema <thema>
%
% (c) 2018 Sebastain Lenhard und Nicolas Tobler, Hochschule Rapperswil
%
\chapter{Klima auf anderen Planeten\label{chapter:thema}}
\lhead{Klima auf anderen Planeten}
\begin{refsection}
\chapterauthor{Nicolas Tobler}

\section{Einführung}
\rhead{Abschnitt}

Verschiedene Organisationen, wie unter anderem Elon Musk's SpaceX, haben sich zum Ziel gemacht, in absehbarer Zukunft den Mars für den Menschen bewohnbar zu machen. Insbesondere sollte der Mars eine erdählnliche Atmosphere erhalte, also terraformed werden. Was auf Computer-generierten bildern ziemlich simpel aussieht, wird sich in realität wahrscheinlich ziemlich schwierig herausstellen. In diesem Kapitel wird die aktuelle lage des Klimas auf dem Mars analysiert und mögliche Wege den Mars zu teraformen auf die Machbarkeit untersucht.

\section{Das Klima auf dem Mars}

Verschiedene Mars proben

\subsection{Temperatur}

\subsection{Albedo}


\subsection{Energieerhaltungs Gleichungen}




\subsection{Atmosphereische Eigenschaften}

dünne Atmosphere

Eann und wiso velohr der Mars seine Athmosphere


Rückgang der Atmosphere durch Sonnenwind
	https://www.nasa.gov/press-release/nasa-mission-reveals-speed-of-solar-wind-stripping-martian-atmosphere
	Vor 4.2 Milliarden Jahren gefrohr der Kern


\section{Den Mars terraformen}



\subsection{Aussetzen von Treibhausgasen}

Aussetzen von hocheffektiven treibhausgasen wie FCKW's

verwendung von Methan 


\section{Einleitung}

venus co2 bindung
wie ist venus co geworden


was führt zu wasserlosen athmosphere
mars keine
venus ohne wasser
erde zwischendrin

boxmodell anteile element

\section{Modell}

Boxmodell

Parameter

atmosphere
wasser
co2
sauerstoff

land
wasser

\section{Energiehaushalt}

\begin{equation}
\frac{dT}{dt} = P_{in} - P_{out}
\end{equation}

%reference to skript needed


\begin{equation}
P_{in} = \sigma T^4 \frac{R_{\astrosun}}{a_{planet}}^2 \cdot (1-\alpha)
\end{equation}

$\alpha$ ist das albedo des Planeten.

\begin{equation}
P_{out} = P_{blackbody} \cdot \beta
\end{equation}

$\beta$ ist der treibhausfaktor

\begin{equation}
\beta = f(m_{CO_2}, m_{O_2}, m_{H_2O})
\end{equation}

\begin{equation}
\alpha = f(m_{CO_2}, m_{O_2}, m_{H_2O})
\end{equation}



\section{Entweichen von atmospherischen Gasen}

Ob ein Planet oder Mond eine Atmosphere besitzt ist von wenigen parametern abhängig. Planeten behalten ihr Atmosphere, wenn die Gravitation ausreichend stark ist, um die Moleküle gegen ihre thermische Geschwindigkeit zurückzuhalten.
Die Moleküle in der Atmosphere, für die

% https://www.tcd.ie/Physics/people/Peter.Gallagher/lectures/PY4A03/pdfs/PY4A03_lecture12n13_amospheres.ppt.pdf

\begin{equation}
v_{escape} > v_{therm}
\end{equation}

zutrifft, werden nach und nach in den Weltraum abgestossen.
Die Fluchtgeschwindigkeit eines Planeten $v_{escape}$ wird durch deren Masse $M$ und Radius $R$ berechnet: 

\begin{equation}
v_{escape} = \sqrt{\frac{2GM}{R}}
\end{equation}

wobei $G$ die Gravitationskonstante ist. Die Fluchtgeschwindigkeit ist somit bei grossen und schweeren Gestirnen grösser.

Die thermische Geschwindigkeit eines Moleküls ist Maxwell-Bolzmann-verteilt. Deshalb gilt der berechnete Wert in diesem Zusammenhang nur approximativ. Die höchst wahrscheinliche Geschwindigkeit ist: 

\begin{equation}
v_{therm} = \sqrt{\frac{3kT}{m}}
\end{equation}

Dabei ist $k$ die Bolzmann-Konstante und $m$ die Mol-Masse des Moleküls. Die kritische Temperatur $T_{escape}$, bei welcher Gase abgestossen werden ist somit:

\begin{equation}
T_{escape} = \frac{2GMm}{3kR}
\end{equation}








\section{Schlussfolgerung}
\rhead{Schlussfolgerung}

\printbibliography[heading=subbibliography]
\end{refsection}
