%
% main.tex -- Paper zum Thema <thema>
%
% (c) 2018 Hansruedi Patzen, Hochschule Rapperswil
%
\chapter{H"oherdimensionales Lorenzsystem\label{chapter:lorenz2}}
\lhead{H"oherdimensionales Lorenzsystem}
\begin{refsection}
\chapterauthor{Hansruedi Patzen}

Auf das dreidimensionale Lorenzsystem wurde bereits in 
\cref{section:lorenz-modell} sowie in \cref{chapter:lorenz} detailliert 
eingegangen. In diesem Kapitel wollen wir nun aber Lorenzsysteme h"oherer 
Ordnung untersuchen. Das kommt mit einiger Arbeit einher. So brauchen wir 
entweder komplett neue Basisfunktionen, oder aber wir erweitern unsere 
bestehenden, wie in \cref{section:lorenz2:basic_function} aufgezeigt. Mit den 
dann gefunden Basisfunktionen k"onnen wir dann unsere Gleichungen aufbauen und 
versuchen daf"ur eine L"osung zu finden, letzteres geschieht in 
\cref{section:lorenz2:ho-model}.


\section{Einf"uhrung}
\rhead{Einf"uhrung}

Bevor wir uns auf die Suche nach Basisfunktionen machen, folgen hier ein paar 
Angaben zu den in diesem Kapitel verwendeten Notation und einigen grundlegenden 
Funktionen und Umformungen, welche sp"ater verwendet werden, ein kleiner 
``Teaser'' zu dem was uns noch erwartet sozusagen.

\subsection{Notation}
Die verwendete Notation entspricht grunds"atzlich derjenigen, welche man 
bereits aus anderen mathematischen B"uchern kennt. Einzig die Verwendung der 
sogenannten Multi-Index-Notationen k"onnte f"ur einige Leser zu Unklarheiten 
f"uhren, daher hier eine kurze Einf"uhrung.

Unter der Multi-Index-Notation versteht man einen Index, der einem 
$n$-Tupel
\begin{equation*}
	\alpha = (\alpha_1, \alpha_2, \dotsc, \alpha_n) \qquad \alpha_k \in 
	\mathbb{N}_{0}
\end{equation*}
entspricht. Zur klaren Unterscheidung zu normalen Indexen, wird meist ein 
griechischer Buchstabe verwendet. Multi-Indexe haben zudem die Eigenschaft, 
dass ihr absoluter Wert wie folgt definiert ist
\begin{equation*}
	|\alpha| = \alpha_1 + \alpha_2 + \dots + \alpha_n.
\end{equation*}
 
N"utzlich ist diese Notation insbesondere, wenn mit Summen gearbeitet wird. So 
kann mittels einer Forderung wie beispielsweise $|\alpha| = k$, die Summe
\begin{equation}
	\sum_{k = 0}^{\infty}\sum_{l = 0}^{k}a_{l, k - l}
	\label{equation:lorenz2:doublesum}
\end{equation}
vereinfacht werden zu
\begin{equation}
	\sum_{k = 0}^{\infty}a_{\alpha} \qquad |\alpha| = k.
	\label{equation:lorenz2:mmsum}
\end{equation}
M"oglich ist dies, da $|\alpha| = k$ f"ur alle Kombinationen der verschiedenen 
$|\alpha_m|$ gilt. Zum Beispiel ist $|\alpha| = k = 2$ f"ur $2$-Tupel 
Multi-Indexe
\begin{align*}
	\alpha &= (0, 2), \\
	\alpha &= (1, 1), \\
	\alpha &= (2, 0)
\end{align*}
erf"ullt, was genau dem in \cref{equation:lorenz2:doublesum} f"ur $k = 2$ 
geforderten Index-Tupel $(l, k - l)$ entspricht.

\subsection{Formel Sammelserum}
Einige Umformungen, die wir in den n"achsten Abschnitten vornehmen werden, 
verwenden teilweise nicht ganz gel"aufige Funkionen. Damit nicht immer gleich 
Wikipedia bem"uht werden muss, werden diese hier in einem kleinen 
Formel Sammelserum zusammengestellt.

Die Signum-Funktion (kurz $\sgn(x)$):
\begin{equation}
\sgn(x) =
\begin{cases*}
1 & f"ur x > 0 \\
0 & f"ur x = 0 \\
-1 & f"ur x < 0
\end{cases*}
\label{equation:lorenz2:signum}
\end{equation}

Betragsfunktion (kurz $|x|$)
\begin{equation}
|x| =
\begin{cases*}
x & f"ur x $\geq$ 0 \\
-x & f"ur x < 0
\end{cases*}
\label{equation:lorenz2:absfunction}
\end{equation}

Sinus und Kosinus eines absoluten Wertes (\cref{equation:lorenz2:absfunction}):
\begin{equation}
\begin{split}
\sin(|x|) &= \sgn(x)\sin(x) \qquad \Leftrightarrow \qquad \sin(x) = 
\sgn(x)\sin(|x|)
\\
\cos(|x|) &= \cos(x)
\end{split}
\label{equation:lorenz2:sincosabs}
\end{equation}

Produkte von Sinus und Kosinus Kombinationen:
\begin{align*}
\cos(x)\sin(y) &= \frac{1}{2} \left(\sin(x + y) - \sin(x - y)\right)
\\
\sin(x)\cos(y) &= \frac{1}{2} \left(\sin(x - y) + \sin(x + y)\right)
\\
\sin(x)\sin(y) &= \frac{1}{2} \left(\cos(x - y) - \cos(x + y)\right)
\\
\cos(x)\cos(y) &= \frac{1}{2} \left(\cos(x - y) + \cos(x + y)\right)
\end{align*}

\section{Erweiterung der Basisfunktionen\label{section:lorenz2:basic_function}}
Die Suche neuer Basisfunktionen beginnt mit denjenigen, die bereits 
in \cref{skript:funktionsauswahl} erw"ahnt sind. Bei genauerer Betrachtung 
stellt man fest, dass diese auch ein wenig anders geschrieben werden k"onnen:
\begin{align*}
\sin(ax)\sin(y) &= \sin(1ax)\sin(1y),\\
\cos(ax)\sin(y) &= \cos(1ax)\sin(1y),\\
\sin(2y) &= \cos(0ax)\sin(2y).
\end{align*}
Wenn wir nun noch ein paar zus"atzliche, auf den ersten Blick etwas nutzlose 
Gleichungen hinzuf"ugen:
\begin{align*}
0 &= \sin(0ax)\sin(2y) \\
\sin(ax)\sin(y) &= \sin(1ax)\sin(1y)\\
0 &= \sin(2ax)\sin(0y) \\
\sin(2y) &= \cos(0ax)\sin(2y)
\cos(ax)\sin(y) &= \cos(1ax)\sin(1y)\\
0 &= \cos(2ax)\sin(0y)\\
\end{align*}
Daraus l"asst sich nun das Pattern
\begin{equation}
\begin{split}
\sin(\alpha_1 ax)\sin(\alpha_2 y) \\
\cos(\alpha_1 ax)\sin(\alpha_2 y)
\end{split}
\label{equation:lorenz2:basic-functions}
\end{equation}
ableiten, wobei $\alpha_1$ und $\alpha_2$ Teile eines Multi-Index sind und die 
Bedingung $|\alpha| = k = 2$ erf"ullen.

F"ur unser dreidimensionales Modell haben wir also Basisfunktionen zweiten 
Grades ($k = 2$) gefunden. Mit diesen Funktionen k"onnen wir nicht nur die uns 
bekannten generieren, sondern diese auch noch erweitern indem wir den Grad $k$ 
varieren.

Leicht zu erkennen ist, dass f"ur $k = 0$ nur die $0$-Funktion 
"ubrig bleibt, da einzig das $(0, 0)$-Tupel die Bedingung $|\alpha| = 0$ 
erf"ullt. $k = 1$ generiert die beiden Tupel $(0, 1)$ und $(1, 0)$, womit die 
Funktion $\sin(y)$ "ubrig bleibt. Somit k"onnen wir Grad $k \geq 1$ 
voraussetzen. So wird auch gleich das Problem einer m"oglichen Division durch 
$0$ elegant umgangen, wie wir sp"ater sehen werden.

\section{H"oherdimensionales Lorenzsystem\label{section:lorenz2:ho-model}}
\rhead{Lorenz-Modell h"oherer Ordnung}
Nun nehmen wir die Grundgleichungen aus \cref{equation:lorenz2:base} und unsere 
neuen Basisfunktionen aus \cref{equation:lorenz2:basic-functions} und bauen 
daraus ein h"oherdimensionales Lorenzsystem.

Als ersten Schritt werden die in \cref{skript:psiansatz} und 
\cref{skript:thetaansatz} gefundenen L"osungen erweitert und es resultiert
\begin{equation}
\psi(x,y,t) =
\sum_{k = 1}^{\infty}
\sum_{|\alpha| = k}
a_{\alpha}(t)
\sin(\alpha_1 ax) \sin(\alpha_2 y),
\label{equation:lorenz2:extendedpsi}
\end{equation}
sowie
\begin{equation}
\vartheta(x,y,t) =
\sum_{k = 1}^{\infty}
\sum_{|\alpha| = k}
b_{\alpha}(t)
\cos(\alpha_1 ax) \sin(\alpha_2 y).
\label{equation:lorenz2:extendedtheta}
\end{equation}

Als n"achstes betrachten wir nochmals die \cref{equation:lorenz2:base}
\begin{align*}
\frac{\partial\Delta\psi}{\partial t}
&=
\nu\Delta^2\psi 
+c\frac{\partial\vartheta}{\partial x}
-\frac{\partial(\psi,\Delta\psi)}{\partial(x,y)}
\\
\frac{\partial\vartheta}{\partial t}
&=
\kappa\Delta\vartheta
+ \frac{T_0}{\pi}\frac{\partial\psi}{\partial x}
- \frac{\partial(\psi,\vartheta)}{\partial(x,y)}.
\end{align*}
Um das Einsetzen und Vereinfachen der Terme einfacher zu gestalten, ist es 
sinnvoll, systematisch vorzugehen. Es ist leicht zu erkennen, dass wir sowohl 
partielle Ableitungen nach $x$ als auch nach $y$ brauchen werden. Daher fangen 
wir an, uns Bausteine zurechtzulegen, indem wir die 
\cref{equation:lorenz2:extendedpsi,equation:lorenz2:extendedtheta} 
je zwei Mal nach $x$ und $y$ ableiten.

F"ur $\psi(x,y,t)$ erhalten wir
\begin{align*}
\frac{\partial\psi}{\partial x}
&=
a
\sum_{k = 1}^{\infty}
\sum_{|\alpha| = k}
\alpha_1
a_{\alpha}(t)
\cos(\alpha_1 ax) \sin(\alpha_2 y),
\\
\frac{\partial\psi}{\partial y}
&=
\sum_{k = 1}^{\infty}
\sum_{|\alpha| = k}
\alpha_2
a_{\alpha}(t)
\sin(\alpha_1 ax) \cos(\alpha_2 y),
\\
\frac{\partial^2\psi}{\partial x^2}
&=
-
a^2
\sum_{k = 1}^{\infty}
\sum_{|\alpha| = k}
\alpha_1^2
a_{\alpha}(t)
\sin(\alpha_1 ax) \sin(\alpha_2 y),
\\
\frac{\partial^2\psi}{\partial y^2}
&=
-
\sum_{k = 1}^{\infty}
\sum_{|\alpha| = k}
\alpha_2^2
a_{\alpha}(t)
\sin(\alpha_1 ax) \sin(\alpha_2 y)
\end{align*}
und f"ur $\vartheta(x,y,t)$
\begin{align*}
\frac{\partial\vartheta}{\partial x}
&=
-
a
\sum_{k = 1}^{\infty}
\sum_{|\alpha| = k}
\alpha_1
b_{\alpha}(t)
\sin(\alpha_1 ax) \sin(\alpha_2 y),
\\
\frac{\partial\vartheta}{\partial y}
&=
\sum_{k = 1}^{\infty}
\sum_{|\alpha| = k}
\alpha_2
b_{\alpha}(t)
\cos(\alpha_1 ax) \cos(\alpha_2 y),
\\
\frac{\partial^2\vartheta}{\partial x^2}
&=
-
a^2
\sum_{k = 1}^{\infty}
\sum_{|\alpha| = k}
\alpha_1^2
b_{\alpha}(t)
\cos(\alpha_1 ax) \sin(\alpha_2 y),
\\
\frac{\partial^2\vartheta}{\partial y^2}
&=
-
\sum_{k = 1}^{\infty}
\sum_{|\alpha| = k}
\alpha_2^2
b_{\alpha}(t)
\cos(\alpha_1 ax) \sin(\alpha_2 y).
\end{align*}

Damit haben wir alles zusammen um den einfachen Laplace-Operator $\laplace$ 
f"ur $\psi(x,y,t)$
\begin{align*}
\laplace\psi
&= 
\frac{\partial^2\psi}{\partial x^2}
+
\frac{\partial^2\psi}{\partial y^2}
\\
&=
-
a^2
\sum_{i = 1}^{\infty}
\sum_{|\gamma| = i}
\gamma_1^2
a_{\gamma}(t)
\sin(\gamma_1 ax) \sin(\gamma_2 y)
-
\sum_{j = 1}^{\infty}
\sum_{|\delta| = j}
\delta_2^2
a_{\delta}(t) 
\sin(\delta_1 ax) \sin(\delta_2 y)
\\
&=
-
\sum_{k = 1}^{\infty}
\sum_{|\alpha| = k}
\left(\alpha_1^2 a^2 + \alpha_2^2 \right)
a_{\alpha}(t)
\sin(\alpha_1 ax) \sin(\alpha_2 y)
\end{align*}
und $\vartheta(x,y,t)$
\begin{align*}
\laplace\vartheta &= 
\frac{\partial^2\vartheta}{\partial x^2}
+
\frac{\partial^2\vartheta}{\partial y^2} \\
&=
-
a^2
\sum_{i = 1}^{\infty}
\sum_{|\gamma| = i}
\gamma_1^2
b_{\gamma}(t)
\cos(\gamma_1 ax) \sin(\gamma_2 y)
-
\sum_{j = 1}^{\infty}
\sum_{|\delta| = j}
\delta_2^2
b_{\delta}(t)
\cos(\delta_1 ax) \sin(\delta_2 y)
\\
&=
-
\sum_{k = 1}^{\infty}
\sum_{|\alpha| = k}
\left(\alpha_1^2 a^2 + \alpha_2^2\right)
b_{\alpha}(t)
\cos(\alpha_1 ax) \sin(\alpha_2 y)
\end{align*}
aufzul"osen. F"ur die \cref{equation:lorenz2:extendedpsi} brauchen wir zudem 
$\laplace^2\psi(x,y,t)$, was die zweifachen Ableitungen von 
$\laplace\psi(x,y,t)$ nach $x$ respektive $y$
\begin{align*}
\frac{\partial\laplace\psi}{\partial x} &=
-
a
\sum_{k = 1}^{\infty}
\sum_{|\alpha| = k}
\alpha_1
\left(\alpha_1^2 a^2 + \alpha_2^2 \right)
a_{\alpha}(t)
\cos(\alpha_1 ax) \sin(\alpha_2 y),
\\
\frac{\partial^2\laplace\psi}{\partial x^2}
&=
a^2
\sum_{k = 1}^{\infty}
\sum_{|\alpha| = k}
\alpha_1^2
\left(\alpha_1^2 a^2+\alpha_2^2\right)
a_{\alpha}(t)
\sin(\alpha_1 ax) \sin(\alpha_2 y),
\\
\frac{\partial\laplace\psi}{\partial y}
&=
-
\sum_{k = 1}^{\infty}
\sum_{|\alpha| = k}
\alpha_2
\left(\alpha_1^2 a^2 + \alpha_2^2 \right)
a_{\alpha}(t)
\sin(\alpha_1 ax) \cos(\alpha_2 y),
\\
\frac{\partial^2\laplace\psi}{\partial y^2}
&=
\sum_{k = 1}^{\infty}
\sum_{|\alpha| = k}
\alpha_2^2
\left(\alpha_1^2 a^2+\alpha_2^2\right)
a_{\alpha}(t)
\sin(\alpha_1 ax) \sin(\alpha_2 y)
\end{align*}
beinhaltet, womit wir dann
\begin{align*}
\laplace^2\psi
&= 
\frac{\partial^2\laplace\psi}{\partial x^2} + 
\frac{\partial^2\laplace\psi}{\partial y^2}
\\
&=
a^2
\sum_{i = 1}^{\infty}
\sum_{|\gamma| = i}
\gamma_1^2
\left(\gamma_1^2 a^2+\gamma_2^2\right)
a_{\gamma}(t)
\sin(\gamma_1 ax) \sin(\gamma_2 y)
\\
&\phantom{={}}
+
\sum_{j = 1}^{\infty}
\sum_{|\delta| = j}
\delta_2^2
\left(\delta_1^2 a^2+\delta_2^2\right)
a_{\delta}(t)
\sin(\delta_1 ax) \sin(\delta_2 y)
\\
&=
\sum_{k = 1}^{\infty}
\sum_{|\alpha| = k}
\left(\alpha_1^2 a^2+\alpha_2^2\right)^2
a_{\alpha}(t)
\sin(\alpha_1 ax) \sin(\alpha_2 y)
\end{align*}
zusammenbauen k"onnen.

Wie wir hier sehen, sind nicht nur die urspr"unglichen, sondern auch auch die 
erweiterten Basisfunktionen nat"urlich wieder Eigenfunktionen des 
Laplace-Operators. Es gilt dabei
\begin{equation*}
	\laplace^n f = \lambda^n f
	\qquad
	\text{mit }
	\lambda = -\left(\alpha_1^2 a^2+\alpha_2^2\right),
	n \in \mathbb{N}.
\end{equation*}

Wem bei den Gleichungen f"ur die Laplace Operatoren aufgrund der Addition 
zweier unendlicher Summen etwas mulmig war, soll unbesorgt sein. Diese 
Umformung sei uns erlaubt, da wir davon ausgehen m"ussen, dass die einzelnen 
Reihen konvergent sind. Zudem h"alt uns nichts davon ab, die 
Indices der beiden unabh"angigen Summen so umzubenennen, dass diese 
"ubereinstimmen. Damit ist klar, dass die gleichen Basisfunktionen generiert 
werden und die Summen somit vereinfacht werden k"onnen.

Mit unseren bisherigen Bausteinen haben wir jetzt fast alles zusammen, um diese 
in \cref{equation:lorenz2:base} einzusetzen. Einzig die Funktionaldeterminanten 
fehlen noch. Diese sind auch der haarige Teil des Gleichungssystems, denn es 
sorgt f"ur die Kopplung der beiden Terme.

Beginnen wir also damit die erste Funktionsdeterminante
\begin{align*}
\frac{\partial(\psi, \laplace\psi)}{\partial(x,y)}
&=
\frac{\partial\psi}{\partial x}
\frac{\partial\laplace\psi}{\partial y}
-
\frac{\partial\psi}{\partial y}
\frac{\partial\laplace\psi}{\partial x}
\\
&=
\left(
a
\sum_{i = 1}^{\infty}
\sum_{|\gamma| = i}
\gamma_1
a_{\gamma}(t)
\cos(\gamma_1 ax) \sin(\gamma_2 y)
\right)
\left(
-
\sum_{j = 1}^{\infty}
\sum_{|\delta| = j}
\delta_2
\left(\delta_1^2 a^2 + \delta_2^2 \right)
a_{\delta}(t)
\sin(\delta_1 ax) \cos(\delta_2 y)
\right)
\\
&\phantom{={}}
-
\left(
\sum_{q = 1}^{\infty}
\sum_{|\xi| = q}
\xi_2
a_{\xi}(t)
\sin(\xi_1 ax) \cos(\xi_2 y)
\right)
\left(
-
a
\sum_{r = 1}^{\infty}
\sum_{|\eta| = r}
\eta_1
\left(\eta_1^2 a^2 + \eta_2^2 \right)
a_{\eta}(t)
\cos(\eta_1 ax) \sin(\eta_2 y)
\right)
\\
&=
-
a
\sum_{i = 1}^{\infty}
\sum_{|\gamma| = i}
\gamma_1
a_{\gamma}(t)
\sum_{j = 1}^{\infty}
\sum_{|\delta| = j}
\delta_2
\left(\delta_1^2 a^2 + \delta_2^2 \right)
a_{\delta}(t)
\cos(\gamma_1 ax) \sin(\gamma_2 y)
\sin(\delta_1 ax) \cos(\delta_2 y)
\\
&\phantom{={}}
+
a
\sum_{q = 1}^{\infty}
\sum_{|\xi| = q}
\xi_2
a_{\xi}(t)
\sum_{r = 1}^{\infty}
\sum_{|\eta| = r}
\eta_1
\left(\eta_1^2 a^2 + \eta_2^2 \right)
a_{\eta}(t)
\sin(\xi_1 ax) \cos(\xi_2 y)
\cos(\eta_1 ax) \sin(\eta_2 y)
\\
&=
-
a
\sum_{i = 1}^{\infty}
\sum_{|\gamma| = i}
\gamma_1
a_{\gamma}(t)
\sum_{j = 1}^{\infty}
\sum_{|\delta| = j}
\delta_2
\left(\delta_1^2 a^2 + \delta_2^2 \right)
a_{\delta}(t) 
\cos(\gamma_1 ax) \sin(\gamma_2 y)
\sin(\delta_1 ax) \cos(\delta_2 y)
\\
&\phantom{={}}
+
a
\sum_{j = 1}^{\infty}
\sum_{|\delta| = j}
\delta_2
a_{\delta}(t)
\sum_{i = 1}^{\infty}
\sum_{|\gamma| = i}
\gamma_1
\left(\gamma_1^2 a^2 + \gamma_2^2 \right)
a_{\gamma}(t)
\sin(\delta_1 ax) \cos(\delta_2 y)
\cos(\gamma_1 ax) \sin(\gamma_2 y)
\\
&=
a
\sum_{i = 1}^{\infty}
\sum_{|\gamma| = i}
\gamma_1
a_{\gamma}(t)
\sum_{j = 1}^{\infty}
\sum_{|\delta| = j}
\delta_2
\left(
a^2 \left(\gamma_1^2 - \delta_1^2 \right)
+ \left(\gamma_2^2 - \delta_2^2 \right)
\right)
a_{\delta}(t)
\\[-2ex]
&\phantom{=abbbbb\sum\sum.....{}ccccccc}
\cdot
\cos(\gamma_1 ax) \sin(\gamma_2 y)
\sin(\delta_1 ax) \cos(\delta_2 y)
\\
&=
\frac{a}{4}
\sum_{i = 1}^{\infty}
\sum_{|\gamma| = i}
\gamma_1
a_{\gamma}(t)
\sum_{j = 1}^{\infty}
\sum_{|\delta| = j}
\delta_2
\left(
a^2 \left(\gamma_1 - \delta_1 \right)\left( \gamma_1 + \delta_1 \right)
+ \left(\gamma_2 - \delta_2 \right)\left( \gamma_2 + \delta_2 \right)
\right)
a_{\delta}(t)
\\[-2ex]
&\phantom{=abbbbb\sum\sum.....{}ccccccc}
\cdot
\Big(
\sin((\gamma_1 + \delta_1) ax)\sin((\gamma_2 + \delta_2) y)
\\[-1ex]
&\phantom{=abbbbb\sum\sum.....{}ccccccc}
+
\sin((\gamma_1 + \delta_1) ax)\sin((\gamma_2 - \delta_2) y)
\\[-1ex]
&\phantom{=abbbbb\sum\sum.....{}ccccccc}
-
\sin((\gamma_1 - \delta_1) ax)\sin((\gamma_2 + \delta_2) y)
\\[-1ex]
&\phantom{=abbbbb\sum\sum.....{}ccccccc}
-
\sin((\gamma_1 - \delta_1) ax)\sin((\gamma_2 - \delta_2) y)
\Big)
\\
&=
\frac{a}{4}
\sum_{i = 1}^{\infty}
\sum_{|\gamma| = i}
\gamma_1
a_{\gamma}(t)
\sum_{j = 1}^{\infty}
\sum_{|\delta| = j}
\delta_2
\left(
a^2 \left(\gamma_1 - \delta_1 \right)\left( \gamma_1 + \delta_1 \right)
+ \left(\gamma_2 - \delta_2 \right)\left( \gamma_2 + \delta_2 \right)
\right)
a_{\delta}(t)
\\[-2ex]
&\phantom{=abbbbb\sum\sum.....{}ccccccc}
\cdot
\Big(
\sgn(\gamma_1 + \delta_1)\sgn(\gamma_2 + \delta_2)
\sin(|\gamma_1 + \delta_1|ax)\sin(|\gamma_2 + \delta_2|y)
\\[-1ex]
&\phantom{=abbbbb\sum\sum.....{}ccccccc}
+
\sgn(\gamma_1 + \delta_1)\sgn(\gamma_2 - \delta_2)
\sin(|\gamma_1 + \delta_1|ax)\sin(|\gamma_2 - \delta_2|y)
\\[-1ex]
&\phantom{=abbbbb\sum\sum.....{}ccccccc}
-
\sgn(\gamma_1 - \delta_1)\sgn(\gamma_2 + \delta_2)
\sin(|\gamma_1 - \delta_1|ax)\sin(|\gamma_2 + \delta_2|y)
\\[-1ex]
&\phantom{=abbbbb\sum\sum.....{}ccccccc}
-
\sgn(\gamma_1 - \delta_1)\sgn(\gamma_2 - \delta_2)
\sin(|\gamma_1 - \delta_1|ax)\sin(|\gamma_2 - \delta_2|y)
\Big)
\end{align*}
und analog dazu kann auch die zweite
\begin{align*}
\frac{\partial(\psi, \vartheta)}{\partial(x,y)}
&=
\frac{\partial\psi}{\partial x}
\frac{\partial\vartheta}{\partial y}
-
\frac{\partial\psi}{\partial y}
\frac{\partial\vartheta}{\partial x}
\\
&=
\left(
a
\sum_{i = 1}^{\infty}
\sum_{|\gamma| = i}
\gamma_1
a_{\gamma}(t)
\cos(\gamma_1 ax) \sin(\gamma_2 y)
\right)
\left(
\sum_{j = 1}^{\infty}
\sum_{|\delta| = j}
\delta_2
b_{\delta}(t)
\cos(\delta_1 ax) \cos(\delta_2 y)
\right)
\\
&\phantom{={}}
-
\left(
\sum_{q = 1}^{\infty}
\sum_{|\xi| = q}
\xi_2
a_{\xi}(t)
\sin(\xi_1 ax) \cos(\xi_2 y)
\right)
\left(
-
a
\sum_{r = 1}^{\infty}
\sum_{|\eta| = r}
\eta_1
b_{\eta}(t)
\sin(\eta_1 ax) \sin(\eta_2 y)
\right)
\\
&=\phantom{+}
a
\sum_{i = 1}^{\infty}
\sum_{|\gamma| = i}
\gamma_1
a_{\gamma}(t)
\sum_{j = 1}^{\infty}
\sum_{|\delta| = j}
\delta_2
b_{\delta}(t)
\cos(\gamma_1 ax) \sin(\gamma_2 y)
\cos(\delta_1 ax) \cos(\delta_2 y)
\\
&\phantom{={}}
+
a
\sum_{q = 1}^{\infty}
\sum_{|\xi| = q}
\xi_2
a_{\xi}(t)
\sum_{r = 1}^{\infty}
\sum_{|\eta| = r}
\eta_1
b_{\eta}(t)
\sin(\xi_1 ax) \cos(\xi_2 y)
\sin(\eta_1 ax) \sin(\eta_2 y)
\\
&=\phantom{+}
a
\sum_{i = 1}^{\infty}
\sum_{|\gamma| = i}
\gamma_1
a_{\gamma}(t)
\sum_{j = 1}^{\infty}
\sum_{|\delta| = j}
\delta_2
b_{\delta}(t)
\cos(\gamma_1 ax) \sin(\gamma_2 y)
\cos(\delta_1 ax) \cos(\delta_2 y)
\\
&\phantom{={}}
+
a
\sum_{i = 1}^{\infty}
\sum_{|\gamma| = i}
\gamma_2
a_{\gamma}(t)
\sum_{j = 1}^{\infty}
\sum_{|\delta| = j}
\delta_1
b_{\delta}(t)
\sin(\gamma_1 ax) \cos(\gamma_2 y)
\sin(\delta_1 ax) \sin(\delta_2 y)
\\
&=
a
\sum_{i = 1}^{\infty}
\sum_{|\gamma| = i}
a_{\gamma}(t)
\sum_{j = 1}^{\infty}
\sum_{|\delta| = j}
b_{\delta}(t)
\Big(
\gamma_1 \delta_2
\cos(\gamma_1 ax) \sin(\gamma_2 y)
\cos(\delta_1 ax) \cos(\delta_2 y)
\\[-2ex]
&\phantom{=abbb\sum\sum.....{}ccccccc}
+
\gamma_2 \delta_1
\sin(\gamma_1 ax) \cos(\gamma_2 y)
\sin(\delta_1 ax) \sin(\delta_2 y)
\Big)	
\\
&=
\frac{a}{4}
\sum_{i = 1}^{\infty}
\sum_{|\gamma| = i}
a_{\gamma}(t)
\sum_{j = 1}^{\infty}
\sum_{|\delta| = j}
b_{\delta}(t)
\bigg(
\gamma_1 \delta_2
\Big(
\cos((\gamma_1 - \delta_1)ax)\sin((\gamma_2 + \delta_2)y)
\\[-2ex]
&\phantom{=abbb\sum\sum.....{}ccccccc}
+\cos((\gamma_1 - \delta_1)ax)\sin((\gamma_2 - \delta_2)y)
\\[-1ex]
&\phantom{=abbb\sum\sum.....{}ccccccc}
+\cos((\gamma_1 + \delta_1)ax)\sin((\gamma_2 + \delta_2)y)
\\[-1ex]
&\phantom{=abbb\sum\sum.....{}ccccccc}
+\cos((\gamma_1 + \delta_1)ax)\sin((\gamma_2 - \delta_2)y)
\Big)
\\[-1ex]
&\phantom{=abbb\sum\sum.....{}ccccccc}
+
\gamma_2 \delta_1
\Big(
\cos((\gamma_1 - \delta_1)ax)\sin((\gamma_2 + \delta_2)y)
\\[-1ex]
&\phantom{=abbb\sum\sum.....{}ccccccc}
-\cos((\gamma_1 - \delta_1)ax)\sin((\gamma_2 - \delta_2)y)
\\[-1ex]
&\phantom{=abbb\sum\sum.....{}ccccccc}
-\cos((\gamma_1 + \delta_1)ax)\sin((\gamma_2 + \delta_2)y)
\\[-1ex]
&\phantom{=abbb\sum\sum.....{}ccccccc}
+\cos((\gamma_1 + \delta_1)ax)\sin((\gamma_2 - \delta_2)y)
\Big)
\bigg)
\\
&=
\frac{a}{4}
\sum_{i = 1}^{\infty}
\sum_{|\gamma| = i}
a_{\gamma}(t)
\sum_{j = 1}^{\infty}
\sum_{|\delta| = j}
b_{\delta}(t)
\bigg(
\left(\gamma_1 \delta_2 + \gamma_2 \delta_1 \right)
\Big(
\cos((\gamma_1 - \delta_1)ax)\sin((\gamma_2 + \delta_2)y)
\\[-2ex]
&\phantom{=abbb\sum\sum.....{}ccccccc}
+
\cos((\gamma_1 + \delta_1)ax)\sin((\gamma_2 - \delta_2)y)
\Big)
\\[-1ex]
&\phantom{=abbb\sum\sum.....{}ccccccc}
+
\left(\gamma_1 \delta_2 - \gamma_2 \delta_1 \right)
\Big(
\cos((\gamma_1 - \delta_1)ax)\sin((\gamma_2 - \delta_2)y)
\\[-1ex]
&\phantom{=abbb\sum\sum.....{}ccccccc}
+
\cos((\gamma_1 + \delta_1)ax)\sin((\gamma_2 + \delta_2)y)
\Big)
\bigg)
\\
&=
\frac{a}{4}
\sum_{i = 1}^{\infty}
\sum_{|\gamma| = i}
a_{\gamma}(t)
\sum_{j = 1}^{\infty}
\sum_{|\delta| = j}
b_{\delta}(t)
\bigg(
\left(\gamma_1 \delta_2 + \gamma_2 \delta_1 \right)
\Big(
\sgn(\gamma_2 + \delta_2)
\cos(|\gamma_1 - \delta_1|ax)\sin(|\gamma_2 + \delta_2|y)
\\[-2ex]
&\phantom{=abbb\sum\sum.....{}ccccccc}
+
\sgn(\gamma_2 - \delta_2)
\cos(|\gamma_1 + \delta_1|ax)\sin(|\gamma_2 - \delta_2|y)
\Big)
\\[-1ex]
&\phantom{=abbb\sum\sum.....{}ccccccc}
+
\left(\gamma_1 \delta_2 - \gamma_2 \delta_1 \right)
\Big(
\sgn(\gamma_2 - \delta_2)
\cos(|\gamma_1 - \delta_1|ax)\sin(|\gamma_2 - \delta_2|y)
\\[-1ex]
&\phantom{=abbb\sum\sum.....{}ccccccc}
+
\sgn(\gamma_2 + \delta_2)
\cos(|\gamma_1 + \delta_1|ax)\sin(|\gamma_2 + \delta_2|y)
\Big)
\bigg)
\end{align*}
aufgel"ost werden. Jetzt haben wir alle Bausteine zusammen und k"onnen diese in 
die einzelnen Gleichungen einsetzen. Damit erhalten wir einerseits
\begin{align*}
\frac{\partial\Delta\psi}{\partial t}
&=
\nu\Delta^2\psi 
+c\frac{\partial\vartheta}{\partial x}
-\frac{\partial(\psi,\Delta\psi)}{\partial(x,y)}
\\
\Leftrightarrow \qquad
&-
\sum_{k = 1}^{\infty}
\sum_{|\gamma| = k}
\dot{a}_{\gamma}(t)
\left(\gamma_1^2 a^2 + \gamma_2^2 \right)
\sin(\gamma_1 ax) \sin(\gamma_2 y)
\\
&=
\nu
\sum_{q = 1}^{\infty}
\sum_{|\xi| = q}
a_{\xi}(t)
\left(\xi_1^2 a^2+\xi_2^2\right)^2
\sin(\xi_1 ax) \sin(\xi_2 y)
\\
&\phantom{={}}
-
ca
\sum_{s = 1}^{\infty}
\sum_{|\eta| = s}
b_{\eta}(t)
\eta_1
\sin(\eta_1 ax) \sin(\eta_2 y)
\\
&\phantom{={}}
-
\frac{a}{4}
\sum_{i = 1}^{\infty}
\sum_{|\zeta| = i}
\zeta_1
a_{\zeta}(t)
\sum_{j = 1}^{\infty}
\sum_{|\delta| = j}
\delta_2
\left(
a^2 \left(\zeta_1 - \delta_1 \right)\left( \zeta_1 + \delta_1 \right)
+ \left(\zeta_2 - \delta_2 \right)\left( \zeta_2 + \delta_2 \right)
\right)
a_{\delta}(t)
\\[-2ex]
&\phantom{=-abbbbb\sum\sum.....{}ccccccc}
\cdot
\Big(
\sgn(\zeta_1 + \delta_1)\sgn(\zeta_2 + \delta_2)
\sin(|\zeta_1 + \delta_1|ax)\sin(|\zeta_2 + \delta_2|y)
\\[-1ex]
&\phantom{=-abbbbb\sum\sum.....{}ccccccc}
+
\sgn(\zeta_1 + \delta_1)\sgn(\zeta_2 - \delta_2)
\sin(|\zeta_1 + \delta_1|ax)\sin(|\zeta_2 - \delta_2|y)
\\[-1ex]
&\phantom{=-abbbbb\sum\sum.....{}ccccccc}
-
\sgn(\zeta_1 - \delta_1)\sgn(\zeta_2 + \delta_2)
\sin(|\zeta_1 - \delta_1|ax)\sin(|\zeta_2 + \delta_2|y)
\\[-1ex]
&\phantom{=-abbbbb\sum\sum.....{}ccccccc}
-
\sgn(\zeta_1 - \delta_1)\sgn(\zeta_2 - \delta_2)
\sin(|\zeta_1 - \delta_1|ax)\sin(|\zeta_2 - \delta_2|y)
\Big)
\\
\Leftrightarrow \qquad
&
\sum_{k = 1}^{\infty}
\sum_{|\gamma| = k}
\dot{a}_{\gamma}(t)
\left(\gamma_1^2 a^2 + \gamma_2^2\right)
\sin(\gamma_1 ax) \sin(\gamma_2 y)
\\
&=
\sum_{i = 1}^{\infty}
\sum_{|\alpha| = i}
\Bigg(
\left(
-\nu
\left(\alpha_1^2 a^2+\alpha_2^2\right)^2
a_{\alpha}(t)
+
\alpha_1 c a
b_{\alpha}(t)
\right)
\sin(\alpha_1 ax) \sin(\alpha_2 y)
\\[-2ex]
&\phantom{=bbbb}
+
\frac{\alpha_1 a}{4}
a_{\alpha}(t)
\sum_{j = 1}^{\infty}
\sum_{|\beta| = j}
\beta_2
\left(
a^2 \left(\alpha_1 - \beta_1 \right)\left(\alpha_1 + \beta_1 \right)
+ \left(\alpha_2 - \beta_2 \right)\left(\alpha_2 + \beta_2 \right)
\right)
a_{\beta}(t)
\\[-2ex]
&\phantom{=bb-4paatabb\sum\sum.{}cc}
\cdot
\Big(
\sgn(\alpha_1 + \beta_1)\sgn(\alpha_2 + \beta_2)
\sin(|\alpha_1 + \beta_1|ax)\sin(|\alpha_2 + \beta_2|y)
\\[-1ex]
&\phantom{=bb-4paatabb\sum\sum.{}cc}
+
\sgn(\alpha_1 + \beta_1)\sgn(\alpha_2 - \beta_2)
\sin(|\alpha_1 + \beta_1|ax)\sin(|\alpha_2 - \beta_2|y)
\\[-1ex]
&\phantom{=bb-4paatabb\sum\sum.{}cc}
-
\sgn(\alpha_1 - \beta_1)\sgn(\alpha_2 + \beta_2)
\sin(|\alpha_1 - \beta_1|ax)\sin(|\alpha_2 + \beta_2|y)
\\[-1ex]
&\phantom{=bb-4paatabb\sum\sum.{}cc}
-
\sgn(\alpha_1 - \beta_1)\sgn(\alpha_2 - \beta_2)
\sin(|\alpha_1 - \beta_1|ax)\sin(|\alpha_2 - \beta_2|y)
\Big)
\Bigg)
\end{align*}
und andererseits
\begin{align*}
\frac{\partial\vartheta}{\partial t}
&=
\kappa\Delta\vartheta
+ \frac{T_0}{\pi}\frac{\partial\psi}{\partial x}
- \frac{\partial(\psi,\vartheta)}{\partial(x,y)}
\\
\Leftrightarrow \qquad
&
\sum_{k = 1}^{\infty}
\sum_{|\gamma| = k}
\dot{b}_{\gamma}(t)
\cos(\gamma_1 ax) \sin(\gamma_2 y)
\\
&=
-
\kappa
\sum_{q = 1}^{\infty}
\sum_{|\xi| = q}
b_{\xi}(t)
\left(\xi_1^2 a^2 + \xi_2^2\right)
\cos(\xi_1 ax) \sin(\xi_2 y)
\\
&\phantom{={}}
+
\frac{a T_0}{\pi}
\sum_{s = 1}^{\infty}
\sum_{|\eta| = s}
\eta_1
a_{\eta}(t)
\cos(\eta_1 ax) \sin(\eta_2 y)
\\
&\phantom{={}}
-
\frac{a}{4}
\sum_{i = 1}^{\infty}
\sum_{|\zeta| = i}
a_{\zeta}(t)
\sum_{j = 1}^{\infty}
\sum_{|\delta| = j}
b_{\delta}(t)
\bigg(
\left(\zeta_1 \delta_2 + \zeta_2 \delta_1 \right)
\Big(
\sgn(\zeta_2 + \delta_2)
\cos(|\zeta_1 - \delta_1|ax)\sin(|\zeta_2 + \delta_2|y)
\\[-2ex]
&\phantom{=-abbb\sum\sum.....{}ccccccc}
+
\sgn(\zeta_2 - \delta_2)
\cos(|\zeta_1 + \delta_1|ax)\sin(|\zeta_2 - \zeta_2|y)
\Big)
\\[-1ex]
&\phantom{=-abbb\sum\sum.....{}ccccccc}
+
\left(\zeta_1 \delta_2 - \zeta_2 \delta_1 \right)
\Big(
\sgn(\zeta_2 - \delta_2)
\cos(|\zeta_1 - \delta_1|ax)\sin(|\zeta_2 - \delta_2|y)
\\[-1ex]
&\phantom{=-abbb\sum\sum.....{}ccccccc}
+
\sgn(\zeta_2 + \delta_2)
\cos(|\zeta_1 + \delta_1|ax)\sin(|\zeta_2 + \delta_2|y)
\Big)
\bigg)
\\
\Leftrightarrow \qquad
&
\sum_{k = 1}^{\infty}
\sum_{|\gamma| = k}
\dot{b}_{\gamma}(t)
\cos(\gamma_1 ax) \sin(\gamma_2 y)
\\
&=
\sum_{i = 1}^{\infty}
\sum_{|\alpha| = i}
\Bigg(
\left(
-
\kappa
\left(\alpha_1^2 a^2 + \alpha_2^2\right)
b_{\alpha}(t)
+
\frac{\alpha_1 a T_0}{\pi}
a_{\alpha}(t)
\right)
\cos(\alpha_1 ax) \sin(\alpha_2 y)
\\
&\phantom{={}aaaa}
-
\frac{a}{4}
a_{\alpha}(t)
\sum_{j = 1}^{\infty}
\sum_{|\beta| = j}
b_{\beta}(t)
\bigg(
\left(\alpha_1 \beta_2 + \alpha_2 \beta_1 \right)
\Big(
\sgn(\alpha_2 + \beta_2)
\cos(|\alpha_1 - \beta_1|ax)\sin(|\alpha_2 + \beta_2|y)
\\[-2ex]
&\phantom{=-aat\sum\sum...{}ccccccc}
+
\sgn(\alpha_2 - \beta_2)
\cos(|\alpha_1 + \beta_1|ax)\sin(|\alpha_2 - \beta_2|y)
\Big)
\\[-1ex]
&\phantom{=-aat\sum\sum...{}ccccccc}
+
\left(\alpha_1 \beta_2 - \alpha_2 \beta_1 \right)
\Big(
\sgn(\alpha_2 - \beta_2)
\cos(|\alpha_1 - \beta_1|ax)\sin(|\alpha_2 - \beta_2|y)
\\[-1ex]
&\phantom{=-aat\sum\sum...{}ccccccc}
+
\sgn(\alpha_2 + \beta_2)
\cos(|\alpha_1 + \beta_1|ax)\sin(|\alpha_2 + \beta_2|y)
\Big)
\bigg)
\Bigg)
\end{align*}
als Gleichungen die es nun zu l"osen gilt.

Da wir die $x$ und $y$ Komponenten loswerden wollen, damit nur noch $t$ "ubrig 
bleibt, brauchen wir Gleichungen f"ur einzelne $\dot{a}_\gamma(t)$ f"ur 
$|\gamma| > 0$
\begin{align}
\left(\gamma_1^2 a^2 + \gamma_2^2\right)
\dot{a}_\gamma(t)
&=
\sum_{i = 1}^{\infty}
\sum_{|\alpha| = i}
\Bigg(
\left(
-\nu
\left(\alpha_1^2 a^2+\alpha_2^2\right)^2
a_{\alpha}(t)
+
\alpha_1 c a
b_{\alpha}(t)
\right)
f_\gamma(\alpha_1, \alpha_2) \nonumber
\\[-2ex]
&\phantom{=aaaa{}}
+
\frac{\alpha_1 a}{4}
a_{\alpha}(t)
\sum_{j = 1}^{\infty}
\sum_{|\beta| = j}
\beta_2
\left(
a^2 \left(\alpha_1 - \beta_1 \right)\left(\alpha_1 + \beta_1 \right)
+ \left(\alpha_2 - \beta_2 \right)\left(\alpha_2 + \beta_2 \right)
\right)
a_{\beta}(t) \nonumber
\\[-2ex]
&\phantom{=-4aapaata..\sum\sum.....{}a.}
\cdot
\left(
f_\gamma(\alpha_1 + \beta_1, \alpha_2 + \beta_2)
+
f_\gamma(\alpha_1 + \beta_1, \alpha_2 - \beta_2)
\right. \nonumber
\\[-1ex]
&\phantom{=-4aapaata..\sum\sum.....{}a.}
\left.
-
f_\gamma(\alpha_1 - \beta_1, \alpha_2 + \beta_2)
-
f_\gamma(\alpha_1 - \beta_1, \alpha_2 - \beta_2)
\right)
\Bigg) \nonumber
\\
\Leftrightarrow \qquad
\dot{a}_\gamma(t)
&=
\left(
-\nu
\left(\gamma_1^2 a^2+\gamma_2^2\right)
a_{\gamma}(t)
+
\frac{\gamma_1 ca}{\gamma_1^2 a^2 + \gamma_2^2}
b_{\gamma}(t)
\right)
\sgn(\gamma_1)\sgn(\gamma_2)
+
\frac{a}{4 \left(\gamma_1^2 a^2 + \gamma_2^2\right)}
\nonumber
\\[-1ex]
&\phantom{={}}
\cdot
\sum_{i = 1}^{\infty}
\sum_{|\alpha| = i}
\alpha_1
a_{\alpha}(t) \nonumber
\sum_{j = 1}^{\infty}
\sum_{|\beta| = j}
\beta_2
\left(
a^2 \left(\alpha_1 - \beta_1 \right)\left(\alpha_1 + \beta_1 \right)
+ \left(\alpha_2 - \beta_2 \right)\left(\alpha_2 + \beta_2 \right)
\right)
a_{\beta}(t) \nonumber
\\[-2ex]
&\phantom{=a.-4aapaata...a\sum\sum{}}
\cdot
\left(
f_\gamma(\alpha_1 + \beta_1, \alpha_2 + \beta_2)
+
f_\gamma(\alpha_1 + \beta_1, \alpha_2 - \beta_2)
\right. \nonumber
\\
&\phantom{=a.-4aapaata...a\sum\sum{}}
\left.
-
f_\gamma(\alpha_1 - \beta_1, \alpha_2 + \beta_2)
-
f_\gamma(\alpha_1 - \beta_1, \alpha_2 - \beta_2)
\right)
\label{equation:lorenz2:dota}
\end{align}
und analog f"ur
\begin{align}
\dot{b}_\gamma(t)
&=
\sum_{i = 1}^{\infty}
\sum_{|\alpha| = i}
\Bigg(
\left(
-
\kappa
\left(\alpha_1^2 a^2 + \alpha_2^2\right)
b_{\alpha}(t)
+
\frac{\alpha_1 a T_0}{\pi}
a_{\alpha}(t)
\right)
g_\gamma(\alpha_1, \alpha_2) 
\nonumber
\\[-1ex]
&\phantom{={}}
-
\frac{a}{4}
a_{\alpha}(t)
\sum_{j = 1}^{\infty}
\sum_{|\beta| = j}
b_{\beta}(t)
\bigg(
\left(\alpha_1 \beta_2 + \alpha_2 \beta_1 \right)
\left(
g_\gamma(\gamma_1 - \beta_1, \alpha_2 + \beta_2)
+
g_\gamma(\gamma_1 + \beta_1, \alpha_2 - \beta_2)
\right) \nonumber
\\[-2ex]
&\phantom{=-aat\sum\sum...{}ccc}
+
\left(\alpha_1 \beta_2 - \alpha_2 \beta_1 \right)
\left(
g_\gamma(\gamma_1 - \beta_1, \alpha_2 - \beta_2)
+
g_\gamma(\gamma_1 + \beta_1, \alpha_2 + \beta_2)
\right)
\bigg)
\Bigg) \nonumber
\\
\Leftrightarrow \qquad
\dot{b}_\gamma(t)
&=
\left(
-
\kappa
\left(\gamma_1^2 a^2 + \gamma_2^2\right)
b_{\gamma}(t)
+
\frac{\gamma_1 a T_0}{\pi}
a_{\gamma}(t)
\right)
\sgn(\gamma_2)
-
\frac{a}{4}
\sum_{i = 1}^{\infty}
\sum_{|\alpha| = i}
a_{\alpha}(t)
\nonumber
\\[-1ex]
&\phantom{={}}
\cdot
\sum_{j = 1}^{\infty}
\sum_{|\beta| = j}
b_{\beta}(t)
\bigg(
\left(\alpha_1 \beta_2 + \alpha_2 \beta_1 \right)
\left(
g_\gamma(\alpha_1 - \beta_1, \alpha_2 + \beta_2)
+
g_\gamma(\alpha_1 + \beta_1, \alpha_2 - \beta_2)
\right)
\nonumber
\\[-2ex]
&\phantom{=-aat\sum..}
+
\left(\alpha_1 \beta_2 - \alpha_2 \beta_1 \right)
\left(
g_\gamma(\alpha_1 - \beta_1, \alpha_2 - \beta_2)
+
g_\gamma(\alpha_1 + \beta_1, \alpha_2 + \beta_2)
\right)
\bigg).
\label{equation:lorenz2:dotb}
\end{align}

Die beiden Hilfsfunktionen $f_\gamma(q, s)$ und $g_\gamma(q, s)$ sind wie folgt 
definiert:
\begin{align*}
f_\gamma(q, s)
&=
\begin{cases}
\sgn(q)\sgn(s) & |q| = \gamma_1, |s| = \gamma_2 \\
0 & \text{sonst}
\end{cases}
\\
g_\gamma(q, s)
&=
\begin{cases}
\sgn(s) & |q| = \gamma_1, |s| = \gamma_2 \\
0 & \text{sonst}.
\end{cases}
\end{align*}

\section{Lorenzsystem vierten Grades\label{section:lorenz2:4degreelorenz}}
\rhead{Lorenzsystem vierten Grades}
Anhand des Lorenzsystems vierten Grades wollen wir nun zeigen, welche 
Information verloren geht, wenn wir uns alleinig auf ein Systme zweiten Grades 
konzentrieren.

Zuerst brauchen wir allerdings alle Gleichungen von $k = 1$
\begin{align*}
|\gamma| = 1
\qquad &
\dot{a}_{(1,0)}(t) = 0
\\
&
\dot{a}_{(0,1)}(t) = 0
\\
\\
&
\dot{b}_{(1,0)}(t) = 0
\\
&
\dot{b}_{(0,1)}(t)
=
-
\kappa
b_{(0,1)}(t)
\\
&\phantom{aaaaaaaaaa}
\color{blue}
+
\frac{a}{4} a_{(1,1)}(t) b_{(1,2)}(t)
+
\frac{a}{4} a_{(1,2)}(t) b_{(1,1)}(t)
\\
&\phantom{aaaaaaaaaa}
\color{red}
+
\frac{a}{4} a_{(1,2)}(t) b_{(1,3)}(t)
+
\frac{a}{4} a_{(1,3)}(t) b_{(1,2)}(t)
\\
&\phantom{aaaaaaaaaa}
\color{red}
+
\frac{a}{2} a_{(2,1)}(t) b_{(2,2)}(t)
+
\frac{a}{2} a_{(2,2)}(t) b_{(2,1)}(t)
\end{align*}
"uber $k = 2$
\begin{align*}
|\gamma| = 2
\qquad &
\dot{a}_{(2,0)}(t) = 0
\\
&
\dot{a}_{(1,1)}(t)
=
-
(a^2+1)
\nu
a_{(1,1)}(t)
+
\frac{a c}{a^2+1} b_{(1,1)}(t)
\\
&\phantom{aaaaaaaaaa}
\color{blue}
+
\frac{9 a (a^2 - 1)}{4 (a^2+1)} a_{(1,2)}(t) a_{(2,1)}(t)
\\
&\phantom{aaaaaaaaaa}
\color{red}
+
\frac{a (3 a^2 - 5)}{a^2+1} a_{(1,3)}(t) a_{(2,2)}(t)
+
\frac{a (5 a^2 - 3)}{a^2+1} a_{(2,2)}(t) a_{(3,1)}(t)
\\
&
\dot{a}_{(0,2)}(t) = 0
\\
\\
&
\dot{b}_{(2,0)}(t) = 0
\\
&
\dot{b}_{(1,1)}(t)
=
-
(a^2+1)
\kappa
b_{(1,1)}(t)
+
\frac{a T_{0}}{\pi} a_{(1,1)}(t)
\\
&\phantom{aaaaaaaaaa}
+
a
a_{(1,1)}(t) b_{(0,2)}(t)
\\
&\phantom{aaaaaaaaaa}
\color{blue}
-
\frac{a}{2} a_{(1,2)}(t) b_{(0,1)}(t)
+
\frac{3 a}{2} a_{(1,2)}(t) b_{(0,3)}(t)
\\
&\phantom{aaaaaaaaaa}
\color{blue}
+
\frac{3 a}{4} a_{(1,2)}(t) b_{(2,1)}(t)
+
\frac{3 a}{4} a_{(2,1)}(t) b_{(1,2)}(t)
\\
&\phantom{aaaaaaaaaa}
\color{red}
-
a
a_{(1,3)}(t) b_{(0,2)}(t)
+
a
a_{(1,3)}(t) b_{(2,2)}(t)
+
2 a
a_{(1,3)}(t) b_{(0,4)}(t)
\\
&\phantom{aaaaaaaaaa}
\color{red}
+
a
a_{(2,2)}(t) b_{(1,3)}(t)
+
a
a_{(2,2)}(t) b_{(3,1)}(t)
+
a
a_{(3,1)}(t) b_{(2,2)}(t)
\\
&
\dot{b}_{(0,2)}(t)
=
-
4
\kappa
b_{(0,2)}(t)
\\
&\phantom{aaaaaaaaaa}
-
\frac{a}{2} a_{(1,1)}(t) b_{(1,1)}(t)
\\
&\phantom{aaaaaaaaaa}
\color{blue}
-
a
a_{(2,1)}(t) b_{(2,1)}(t)
\\
&\phantom{aaaaaaaaaa}
\color{red}
+
\frac{a}{2} a_{(1,1)}(t) b_{(1,3)}(t)
+
\frac{a}{2} a_{(1,3)}(t) b_{(1,1)}(t)
-
\frac{3 a}{2} a_{(3,1)}(t) b_{(3,1)}(t)
\end{align*}
mit $k = 3$
\begin{align*}
|\gamma| = 3
\qquad &
\dot{a}_{(3,0)}(t) = 0
\\
&
\dot{a}_{(2,1)}(t)
=
-
(4 a^2+1)
\nu
a_{(2,1)}(t)
+
\frac{2 a c}{4 a^2+1} b_{(2,1)}(t)
\\
&\phantom{aaaaaaaaaa}
+
\frac{9 a}{4 (4 a^2+1)} a_{(1,1)}(t) a_{(1,2)}(t)
\\
&\phantom{aaaaaaaaaa}
\color{red}
+
\frac{5 a (8 a^2 - 3)}{4 (4 a^2+1)} a_{(1,2)}(t) a_{(3,1)}(t)
+
\frac{25 a}{4 (4 a^2+1)} a_{(1,2)}(t) a_{(1,3)}(t)
\\
&
\dot{a}_{(1,2)}(t)
=
-
(a^2+4)
\nu
a_{(1,2)}(t)
+
\frac{a c}{a^2+4} b_{(1,2)}(t)
\\
&\phantom{aaaaaaaaaa}
-
\frac{9 a^3}{4 (a^2+4)} a_{(1,1)}(t) a_{(2,1)}(t)
\\
&\phantom{aaaaaaaaaa}
\color{red}
+
\frac{5 a (3 a^2 - 8)}{4 (a^2+4)} a_{(1,3)}(t) a_{(2,1)}(t)
-
\frac{25 a^3}{4 (a^2+4)} a_{(2,1)}(t) a_{(3,1)}(t)
\\
&
\dot{a}_{(0,3)}(t) = 0
\\
\\
&
\dot{b}_{(3,0)}(t) = 0
\\
&
\dot{b}_{(2,1)}(t)
=
-
(4 a^2+1)
\kappa
b_{(2,1)}(t)
+
\frac{2 a T_{0}}{\pi} a_{(2,1)}(t)
\\
&\phantom{aaaaaaaaaa}
+
\frac{3 a}{4} a_{(1,1)}(t) b_{(1,2)}(t)
-
\frac{3 a}{4} a_{(1,2)}(t) b_{(1,1)}(t)
+
2 a
a_{(2,1)}(t) b_{(0,2)}(t)
\\
&\phantom{aaaaaaaaaa}
\color{red}
+
\frac{5 a}{4} a_{(1,2)}(t) b_{(1,3)}(t)
-
\frac{5 a}{4} a_{(1,3)}(t) b_{(1,2)}(t)
-
a
a_{(2,2)}(t) b_{(0,1)}(t)
\\
&\phantom{aaaaaaaaaa}
\color{red}
+
\frac{5 a}{4} a_{(1,2)}(t) b_{(3,1)}(t)
+
\frac{5 a}{4} a_{(3,1)}(t) b_{(1,2)}(t)
+
3 a
a_{(2,2)}(t) b_{(0,3)}(t)
\\
&
\dot{b}_{(1,2)}(t)
=
-
(a^2+4)
\kappa
b_{(1,2)}(t)
+
\frac{a T_{0}}{\pi} a_{(1,2)}(t)
\\
&\phantom{aaaaaaaaaa}
-
\frac{a}{2} a_{(1,1)}(t) b_{(0,1)}(t)
\\
&\phantom{aaaaaaaaaa}
+
\frac{3 a}{2} a_{(1,1)}(t) b_{(0,3)}(t)
-
\frac{3 a}{4} a_{(1,1)}(t) b_{(2,1)}(t)
-
\frac{3 a}{4} a_{(2,1)}(t) b_{(1,1)}(t)
\\
&\phantom{aaaaaaaaaa}
\color{red}
-
\frac{5 a}{4} a_{(2,1)}(t) b_{(3,1)}(t)
-
\frac{5 a}{4} a_{(3,1)}(t) b_{(2,1)}(t)
+
\frac{5 a}{4} a_{(1,3)}(t) b_{(2,1)}(t)
\\
&\phantom{aaaaaaaaaa}
\color{red}
+
\frac{5 a}{4} a_{(2,1)}(t) b_{(1,3)}(t)
-
\frac{a}{2} a_{(1,3)}(t) b_{(0,1)}(t)
+
2 a
a_{(1,2)}(t) b_{(0,4)}(t)
\\
&
\dot{b}_{(0,3)}(t)
=
-
9
\kappa
b_{(0,3)}(t)
\\
&\phantom{aaaaaaaaaa}
-
\frac{3 a}{4} a_{(1,1)}(t) b_{(1,2)}(t)
-
\frac{3 a}{4} a_{(1,2)}(t) b_{(1,1)}(t)
\\
&\phantom{aaaaaaaaaa}
\color{red}
-
\frac{3 a}{2} a_{(2,1)}(t) b_{(2,2)}(t)
-
\frac{3 a}{2} a_{(2,2)}(t) b_{(2,1)}(t)
\end{align*}
und $k = 4$
\begin{align*}
|\gamma| = 4
\qquad &
\dot{a}_{(4,0)}(t) = 0
\\
&
\dot{a}_{(3,1)}(t)
=
-
(9 a^2+1)
\nu
a_{(3,1)}(t)
+
\frac{3 a c}{9 a^2+1} b_{(3,1)}(t)
\\
&\phantom{aaaaaaaaaa}
+
\frac{15 a (1 - a^2)}{4 (9 a^2+1)} a_{(1,2)}(t) a_{(2,1)}(t)
\\
&\phantom{aaaaaaaaaa}
+
\frac{12 a (1 + a^2)}{4 (9 a^2+1)} a_{(1,1)}(t) a_{(2,2)}(t)
+
\frac{8 a (5 - 3 a^2)}{4 (9 a^2+1)} a_{(1,3)}(t) a_{(2,2)}(t)
\\
&
\dot{a}_{(2,2)}(t)
=
-
4
(a^2+1)
\nu
a_{(2,2)}(t)
+
\frac{a c}{2 (a^2+1)} b_{(2,2)}(t)
\\
&\phantom{aaaaaaaaaa}
+
\frac{2 a}{a^2+1} a_{(1,1)}(t) a_{(1,3)}(t)
-
\frac{2 a^3}{a^2+1} a_{(1,1)}(t) a_{(3,1)}(t)
\\
&\phantom{aaaaaaaaaa}
+
\frac{4 a^3}{a^2+1} a_{(1,3)}(t) a_{(3,1)}(t)
-
\frac{4 a}{a^2+1} a_{(1,3)}(t) a_{(3,1)}(t)
\\
&
\dot{a}_{(1,3)}(t)
=
-
(a^2+9)
\nu
a_{(1,3)}(t)
+
\frac{a c}{a^2+9} b_{(1,3)}(t)
\\
&\phantom{aaaaaaaaaa}
+
\frac{15 a (1 - a^2)}{4 (a^2+9)} a_{(1,2)}(t) a_{(2,1)}(t)
\\
&\phantom{aaaaaaaaaa}
-
\frac{12 a (1 + a^2)}{4 (a^2+9)} a_{(1,1)}(t) a_{(2,2)}(t)
+
\frac{8 a (3 - 5 a^2)}{4 (a^2+9)} a_{(2,2)}(t) a_{(3,1)}(t)
\\
&
\dot{a}_{(0,4)}(t) = 0
\\
\\
&
\dot{b}_{(4,0)}(t) = 0
\\
&
\dot{b}_{(3,1)}(t)
=
-
(9 a^2+1)
\kappa
b_{(3,1)}(t)
+
\frac{3 a T_{0}}{\pi} a_{(3,1)}(t)
\\
&\phantom{aaaaaaaaaa}
+
\frac{5 a}{4} a_{(2,1)}(t) b_{(1,2)}(t)
-
\frac{5 a}{4} a_{(1,2)}(t) b_{(2,1)}(t)
\\
&\phantom{aaaaaaaaaa}
+
3 a
a_{(3,1)}(t) b_{(0,2)}(t)
+
a
a_{(1,1)}(t) b_{(2,2)}(t)
-
a
a_{(2,2)}(t) b_{(1,1)}(t)
\\
&\phantom{aaaaaaaaaa}
-
2 a
a_{(1,3)}(t) b_{(2,2)}(t)
+
2 a
a_{(2,2)}(t) b_{(1,3)}(t)
\\
&
\dot{b}_{(2,2)}(t)
=
-
(4 a^2+4)
\kappa
b_{(2,2)}(t)
+
\frac{2 a T_{0}}{\pi} a_{(2,2)}(t)
\\
&\phantom{aaaaaaaaaa}
-
a
a_{(2,1)}(t) b_{(0,1)}(t)
+
3 a
a_{(2,1)}(t) b_{(0,3)}(t)
\\
&\phantom{aaaaaaaaaa}
+
a
a_{(1,1)}(t) b_{(1,3)}(t)
-
a
a_{(1,3)}(t) b_{(1,1)}(t)
-
a
a_{(1,1)}(t) b_{(3,1)}(t)
-
a
a_{(3,1)}(t) b_{(1,1)}(t)
\\
&\phantom{aaaaaaaaaa}
+
2 a
a_{(1,3)}(t) b_{(3,1)}(t)
+
2 a
a_{(3,1)}(t) b_{(1,3)}(t)
+
4 a
a_{(2,2)}(t) b_{(0,4)}(t)
\\
&
\dot{b}_{(1,3)}(t)
=
-
(a^2+9)
\kappa
b_{(1,3)}(t)
+
\frac{a T_{0}}{\pi} a_{(1,3)}(t)
\\
&\phantom{aaaaaaaaaa}
-
a
a_{(1,1)}(t) b_{(0,2)}(t)
\\
&\phantom{aaaaaaaaaa}
-
\frac{a}{2} a_{(1,2)}(t) b_{(0,1)}(t)
-
\frac{5 a}{4} a_{(1,2)}(t) b_{(2,1)}(t)
-
\frac{5 a}{4} a_{(2,1)}(t) b_{(1,2)}(t)
\\
&\phantom{aaaaaaaaaa}
-
a
a_{(1,1)}(t) b_{(2,2)}(t)
-
a
a_{(2,2)}(t) b_{(1,1)}(t)
-
2 a
a_{(2,2)}(t) b_{(3,1)}(t)
-
2 a
a_{(3,1)}(t) b_{(2,2)}(t)
\\
&\phantom{aaaaaaaaaa}
+
2 a
a_{(1,1)}(t) b_{(0,4)}(t)
\\
&
\dot{b}_{(0,4)}(t)
=
-
16
\kappa
b_{(0,4)}(t)
\\
&\phantom{aaaaaaaaaa}
-
a
a_{(1,2)}(t) b_{(1,2)}(t)
\\
&\phantom{aaaaaaaaaa}
-
a
a_{(1,1)}(t) b_{(1,3)}(t)
-
a
a_{(1,3)}(t) b_{(1,1)}(t)
-
2a
a_{(2,2)}(t) b_{(2,2)}(t).
\end{align*}
Terme die verloren gehen, wenn man beim jeweiligen $k$ stoppen w"urde, sind 
$\color{blue}{blau}$ (dritter Grad), beziehungsweise $\color{red}{rot}$ 
(vierter Grad) hervorgehoben.

Vergleicht man die Resultate f"ur $k = 2$ mit denjenigen aus 
\cref{skript:lorenz:dim} stellt man fest, dass diese "ubereinstimmen, womit 
auch wieder gezeigt ist, dass unsere neuen Basisfunktionen eine echte 
Erweiterung sind. Bereits jetzt ist aber ersichtlich, dass die Anzahl 
zu l"osenden Gleichungen, in $\text{O}(k^2)$ mit dem Grad $k$ w"achst 
(\cref{table:lorenz2:degree}). Beispielsweise muss f"ur $k = 10$ ein 
Gleichungssystem mit
\begin{equation*}
	2\left(\frac{(10 + 1)(10 + 2)}{2} - 1\right) = 11 \cdot 12 - 2 = 130
\end{equation*}
Gleichungen gel"ost werden, das zudem noch aus Gleichungen besteht, die "uber 
etliche Kopplungen miteinander verbunden sind.

\begin{table}
	\centering
	\begin{tabular}{c | l}
		Grad $k$ & Anzahl Gleichungen \\
		\hline
		1 & $2$ \\
		2 & $2 + 3$ \\
		3 & $2 + 3 + 4$\\
		4 & $2 + 3 + 4 + 5$\\
		\dots & \dots \\
		$n$ & $\dfrac{(n + 1)((n + 1) + 1)}{2} - 1
		= \dfrac{(n + 1)(n + 2)}{2} - 1$
	\end{tabular}
	\caption{Wachstun der Anzahl Gleichung mit dem Grad $k$}
	\label{table:lorenz2:degree}
\end{table}

\section{Numerische L"osung\label{section:lorenz2:numeric-solution}}
\rhead{Numerische L"osung}
Mit den \cref{equation:lorenz2:dota,equation:lorenz2:dotb} aus 
\cref{section:lorenz2:ho-model} haben wir Gleichungen gefunden, die wir mit 
Hilfe eines Computers berechnen k"onnen. Wobei wir dabei nat"urlich 
unterschiedlich vorgehen k"onnen. Was wir sicher brauchen, ist ein Programm mit 
dessen Hilfe wir gew"ohnliche Differentialgleichungen l"osen k"onnen. In 
unserem Fall traf die Wahl auf \texttt{octave}, nicht nur weil es auf allen 
g"angigen Plattformen verf"ugbar und open-source ist, sondern weil es mit 
\texttt{lsode} einen m"achtigen  ODE-Solver beinhaltet.

Wie sich herausstellt, ist eine direkte Implementation unserer gefundenen 
gew"ohnlichen Differentialgleichungen \cref{equation:lorenz2:dota} und 
\cref{equation:lorenz2:dotb} in \texttt{octave} keine sonderlich effiziente 
L"osung, da der Algorithmus selbst f"ur kleine Grad $k$, auf dem Testsystem mit 
einer Intel\textregistered Core\texttrademark\, i7-3930K CPU mit 3.20 GHz, sehr 
lange dauert.

Um das ganze etwas zu optimieren, nehmen wir also am besten ein 
Computer-Algebra Programm, mit dessen wir unsere Gleichungen bereits soweit 
vereinfachen k"onnen, dass nur noch ein Vektor mit einfachen Summen "ubrig 
bleibt. Mit \texttt{maxima} haben wir dazu wieder ein open-source Programm das 
auf allen Plattformen zur Verf"ugung steht.

Wir haben jetzt alles zusammen um die Gleichungen mit \texttt{maxima} zu 
vereinfachen und uns eine Inputfunktion f"ur \texttt{octave} zu generieren, die 
halbwegs performant ist. Trotz unseren Optimierungen ist das L"osen der 
Gleichungen immer noch "ausserst Rechen- und Zeitintensiv, was die 
Zeitmessdaten einmal f"ur das Zeitintervall $t = [0,100]$ und einmal f"ur $t = 
[0,40]$ auf dem Testsystem in \cref{figure:lorenz2:timings} zeigt.

Eine der Ursachen f"ur die stark wachsende Laufzeit kann die fehlende 
Parallelisierung des Algorithmus sein, was dazu f"uhrt dass f"ur jeden Grad $k$ 
nur ein CPU Core verwendet wird. Das erlaubt es uns zwar die Berechnung 
einzelner Grade $k$ gleichzeitig durchzuf"uhren, indem wir unterschiedliche 
\texttt{octave}-Instanzen f"ur die jeweiligen $k$ starten, doch ist der Nutzen 
aufgrund der rasant wachsenden Laufzeit nicht sonderlich gross. Zudem hat auch 
die Wahl des $t$-Intervalls und auch die an \texttt{lsode} gestellten 
Toleranzen Auswirkungen auf die Laufzeit. So kann man durch Lockerung unserer 
absoluten und relativen Toleranzabweichung von $10^{-12}$ respektive $10^{-13}$ 
mit Sicherheit noch etwas an Laufzeit gewinnen, was dann aber mit Verlust an 
Genauigkeit einhergeht w"are.

Noch gar nicht angesprochen wurden die gew"ahlten Anfangsbedingungen und 
Parameterwerte. Bei den Anfangswerten setzen wir alles bis auf $b_{(0,2)}$ und 
$b_{(1,1)}$ gleich $0$, dies da diese Bereits f"ur Grad $2$ Verf"ugbar sind. 
Die Parameter sind so gew"ahlt, dass sie chaotisches Verhalten f"ur ein 
dreidimensionales, also Grad $2$, Lorenzsystem hervorrufen. Es gilt dabei
\begin{align*}
	a &= \sqrt{1/2} \\
	T_0 &= 5 \\
	c &= 82 \\
	\nu &= 1.43 \\
	\kappa &= 0.143
\end{align*}
wobei gesagt werden muss, dass diese experimentell bestimmt sind.

Mit all dem haben wir Lorenzsysteme mit Grad $k = \{2,3,\dots,22\}$ erfolgreich 
berechnet und kommen teilweise zu erstaunlichen Ergebnissen, wie 
\cref{figure:lorenz2:systemdeg2,figure:lorenz2:systemdeg3,figure:lorenz2:systemdeg4,figure:lorenz2:systemdeg5,figure:lorenz2:systemdeg6,figure:lorenz2:systemdeg7,figure:lorenz2:systemdeg8,figure:lorenz2:systemdeg9,figure:lorenz2:systemdeg10,figure:lorenz2:systemdeg11,figure:lorenz2:systemdeg12,figure:lorenz2:systemdeg13,figure:lorenz2:systemdeg14,figure:lorenz2:systemdeg15,figure:lorenz2:systemdeg16,figure:lorenz2:systemdeg17,figure:lorenz2:systemdeg18,figure:lorenz2:systemdeg19,figure:lorenz2:systemdeg20,figure:lorenz2:systemdeg21-40,figure:lorenz2:systemdeg22-40}
zeigen. Eines der augenf"alligsten ist, das ab $k = 4$ das chaotische Verhalten 
verschwindet. Das kann einerseits an unsere Reduktion des Ursprungssystems 
liegen oder der fehlenden Untersuchung des Anfangsbedingungs- und 
Parameterraumes.

\begin{figure}
	\centering
	\includegraphics[width=0.49\linewidth]{lorenz2/03-images/timing_100}
	\includegraphics[width=0.49\linewidth]{lorenz2/03-images/timing_40}
	\caption{Laufzeit f"ur Berechnung von eines Lorenzsystems mit Grad $k$}
	\label{figure:lorenz2:timings}
\end{figure}

\begin{figure}
	\centering
	\includegraphics[width=0.49\linewidth]{{lorenz2/03-images/ord2.X}.pdf}
	\includegraphics[width=0.49\linewidth]{{lorenz2/03-images/ord2.butterfly}.pdf}
	\caption{Lorenzssystem mit Grad 2}
	\label{figure:lorenz2:systemdeg2}
\end{figure}

\begin{figure}
	\centering
	\includegraphics[width=0.49\linewidth]{{lorenz2/03-images/ord3.X}.pdf}
	\includegraphics[width=0.49\linewidth]{{lorenz2/03-images/ord3.butterfly}.pdf}
	\caption{Lorenzssystem mit Grad 3}
	\label{figure:lorenz2:systemdeg3}
\end{figure}

\begin{figure}
	\centering
	\includegraphics[width=0.49\linewidth]{{lorenz2/03-images/ord4.X}.pdf}
	\includegraphics[width=0.49\linewidth]{{lorenz2/03-images/ord4.butterfly}.pdf}
	\caption{Lorenzssystem mit Grad 4}
	\label{figure:lorenz2:systemdeg4}
\end{figure}

\begin{figure}
	\centering
	\includegraphics[width=0.49\linewidth]{{lorenz2/03-images/ord5.X}.pdf}
	\includegraphics[width=0.49\linewidth]{{lorenz2/03-images/ord5.butterfly}.pdf}
	\caption{Lorenzssystem mit Grad 5}
	\label{figure:lorenz2:systemdeg5}
\end{figure}

\begin{figure}
	\centering
	\includegraphics[width=0.49\linewidth]{{lorenz2/03-images/ord6.X}.pdf}
	\includegraphics[width=0.49\linewidth]{{lorenz2/03-images/ord6.butterfly}.pdf}
	\caption{Lorenzssystem mit Grad 6}
	\label{figure:lorenz2:systemdeg6}
\end{figure}

\begin{figure}
	\centering
	\includegraphics[width=0.49\linewidth]{{lorenz2/03-images/ord7.X}.pdf}
	\includegraphics[width=0.49\linewidth]{{lorenz2/03-images/ord7.butterfly}.pdf}
	\caption{Lorenzssystem mit Grad 7}
	\label{figure:lorenz2:systemdeg7}
\end{figure}

\begin{figure}
	\centering
	\includegraphics[width=0.49\linewidth]{{lorenz2/03-images/ord8.X}.pdf}
	\includegraphics[width=0.49\linewidth]{{lorenz2/03-images/ord8.butterfly}.pdf}
	\caption{Lorenzssystem mit Grad 8}
	\label{figure:lorenz2:systemdeg8}
\end{figure}

\begin{figure}
	\centering
	\includegraphics[width=0.49\linewidth]{{lorenz2/03-images/ord9.X}.pdf}
	\includegraphics[width=0.49\linewidth]{{lorenz2/03-images/ord9.butterfly}.pdf}
	\caption{Lorenzssystem mit Grad 9}
	\label{figure:lorenz2:systemdeg9}
\end{figure}

\begin{figure}
	\centering
	\includegraphics[width=0.49\linewidth]{{lorenz2/03-images/ord10.X}.pdf}
	\includegraphics[width=0.49\linewidth]{{lorenz2/03-images/ord10.butterfly}.pdf}
	\caption{Lorenzssystem mit Grad 10, $t = [0,100]$}
	\label{figure:lorenz2:systemdeg10}
\end{figure}

\begin{figure}
	\centering
	\includegraphics[width=0.49\linewidth]{{lorenz2/03-images/ord11.X}.pdf}
	\includegraphics[width=0.49\linewidth]{{lorenz2/03-images/ord11.butterfly}.pdf}
	\caption{Lorenzssystem mit Grad 11, $t = [0,100]$}
	\label{figure:lorenz2:systemdeg11}
\end{figure}

\begin{figure}
	\centering
	\includegraphics[width=0.49\linewidth]{{lorenz2/03-images/ord12.X}.pdf}
	\includegraphics[width=0.49\linewidth]{{lorenz2/03-images/ord12.butterfly}.pdf}
	\caption{Lorenzssystem mit Grad 12, $t = [0,100]$}
	\label{figure:lorenz2:systemdeg12}
\end{figure}

\begin{figure}
	\centering
	\includegraphics[width=0.49\linewidth]{{lorenz2/03-images/ord13.X}.pdf}
	\includegraphics[width=0.49\linewidth]{{lorenz2/03-images/ord13.butterfly}.pdf}
	\caption{Lorenzssystem mit Grad 13, $t = [0,100]$}
	\label{figure:lorenz2:systemdeg13}
\end{figure}

\begin{figure}
	\centering
	\includegraphics[width=0.49\linewidth]{{lorenz2/03-images/ord14.X}.pdf}
	\includegraphics[width=0.49\linewidth]{{lorenz2/03-images/ord14.butterfly}.pdf}
	\caption{Lorenzssystem mit Grad 14, $t = [0,100]$}
	\label{figure:lorenz2:systemdeg14}
\end{figure}

\begin{figure}
	\centering
	\includegraphics[width=0.49\linewidth]{{lorenz2/03-images/ord15.X}.pdf}
	\includegraphics[width=0.49\linewidth]{{lorenz2/03-images/ord15.butterfly}.pdf}
	\caption{Lorenzssystem mit Grad 15, $t = [0,100]$}
	\label{figure:lorenz2:systemdeg15}
\end{figure}

\begin{figure}
	\centering
	\includegraphics[width=0.49\linewidth]{{lorenz2/03-images/ord16.X}.pdf}
	\includegraphics[width=0.49\linewidth]{{lorenz2/03-images/ord16.butterfly}.pdf}
	\caption{Lorenzssystem mit Grad 16, $t = [0,100]$}
	\label{figure:lorenz2:systemdeg16}
\end{figure}

\begin{figure}
	\centering
	\includegraphics[width=0.49\linewidth]{{lorenz2/03-images/ord17.X}.pdf}
	\includegraphics[width=0.49\linewidth]{{lorenz2/03-images/ord17.butterfly}.pdf}
	\caption{Lorenzssystem mit Grad 17, $t = [0,100]$}
	\label{figure:lorenz2:systemdeg17}
\end{figure}

\begin{figure}
	\centering
	\includegraphics[width=0.49\linewidth]{{lorenz2/03-images/ord18.X}.pdf}
	\includegraphics[width=0.49\linewidth]{{lorenz2/03-images/ord18.butterfly}.pdf}
	\caption{Lorenzssystem mit Grad 18, $t = [0,100]$}
	\label{figure:lorenz2:systemdeg18}
\end{figure}

\begin{figure}
	\centering
	\includegraphics[width=0.49\linewidth]{{lorenz2/03-images/ord19.X}.pdf}
	\includegraphics[width=0.49\linewidth]{{lorenz2/03-images/ord19.butterfly}.pdf}
	\caption{Lorenzssystem mit Grad 19, $t = [0,100]$}
	\label{figure:lorenz2:systemdeg19}
\end{figure}

\begin{figure}
	\centering
	\includegraphics[width=0.49\linewidth]{{lorenz2/03-images/ord20.X}.pdf}
	\includegraphics[width=0.49\linewidth]{{lorenz2/03-images/ord20.butterfly}.pdf}
	\caption{Lorenzssystem mit Grad 20, $t = [0,100]$}
	\label{figure:lorenz2:systemdeg20}
\end{figure}

\begin{figure}
	\centering
	\includegraphics[width=0.49\linewidth]{{lorenz2/03-images/ord21.X.40}.pdf}
	\includegraphics[width=0.49\linewidth]{{lorenz2/03-images/ord21.butterfly.40}.pdf}
	\caption{Lorenzssystem mit Grad 21, $t = [0,40]$}
	\label{figure:lorenz2:systemdeg21-40}
\end{figure}

\begin{figure}
	\centering
	\includegraphics[width=0.49\linewidth]{{lorenz2/03-images/ord22.X.40}.pdf}
	\includegraphics[width=0.49\linewidth]{{lorenz2/03-images/ord22.butterfly.40}.pdf}
	\caption{Lorenzssystem mit Grad 22, $t = [0,40]$}
	\label{figure:lorenz2:systemdeg22-40}
\end{figure}

\section{Schlussfolgerungen}
\rhead{Schlussfolgerungen}
Wir haben somit gezeigt, dass es m"oglich ist, ein Lorenzsystem h"oherer 
Ordnung 
mittels Erweiterung der bekannten Basisfunktionen zu bestimmen. Wir k"onnen 
weiter die ben"otigten Gleichungen mittels Computer generieren und auswerten, 
auch wenn dies mit gr"osser werdendem Grad $k$ "ausserst lange dauert. Eine 
m"ogliche Verbesserung w"are Beispielsweise eine parallele Berechnung der 
einzelnen ODE Gleichungen.

Erstaunlich ist auch, dass ab Grad $k = 4$ das chaotische Verhalten nicht mehr 
aufzutreten scheint. Das k"onnte einerseits darauf hinweise, dass dies durch 
die Reduktion des originalen Systems entstanden ist. Andererseits k"onnte dies 
auch einfach daran liegen, dass wir den Raum unserer Anfangsbedingungen und 
Parameter nicht weiter untersucht haben. So kann es durchaus sein, dass es 
etwaige Werte gibt, die uns wieder ein chaotisches Lorenzsystem liefern.


\printbibliography[heading=subbibliography]
\end{refsection}
