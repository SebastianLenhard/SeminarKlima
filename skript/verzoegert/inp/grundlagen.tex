%
% main.tex -- Paper zum Thema verzoegerte Differentialgleichung
%
% (c) 2018 Raphael Unterer, Hochschule Rapperswil
%
\section{Grundlagen verzögerte Differentialgleichungen}
\rhead{Grundlagen DDE}
\subsection{Definitionen}
Verzögerte Differentialgleichungen werden als DDE (engl. "\textbf{D}elayed \textbf{D}ifferential \textbf{E}quation") abgekürzt.

Die allgemeine DDE 1. Ordnung sieht folgendermaßen aus:
\begin{equation}
	\dot{x}(t) = f(x(t),x(t-\tau_1),\dots,x(t-\tau_n))
\end{equation}
Dabei ist $f$ eine beliebige Funktion. Die Verzögerungen $\tau_1,\dots,\tau_n$ sind gegeben und nach der Grösse geordnet, also $0<\tau_1<\dots<\tau_n$.

Im Unterschied zu einer gewöhnliche Differentialgleichung ist das Anfangswertproblem nicht mehr eindimensional, d.h. es genügt nicht mehr den Anfangszustand zu kennen.
Alle Werte von $-\tau_n$ bis $0$ müssen gegeben sein. 
Es braucht nicht nur einen einzelnen Anfangswertvektor, sondern eine ganze Funktion von Anfangswertvektoren im Intervall $[-\tau, 0]$.

\subsection{Analytische Lösungsverfahren an einem Beispiel}
Um die analytischen Lösungsverfahren zu verstehen, werden diese zunächst an einem einfachen Beispiel erläutert.
In den folgenden Betrachtungen analysieren wir die DDE:
\begin{equation}\label{bsp}
\dot{y}(t)=ky(t-\tau)
\end{equation}

\subsubsection{Schrittweises Lösen}
Beim schrittweisen wird die DDE immer in Schritten von einem $\tau$ gelöst.
Wir nehmen an, dass $y$ im Bereich von $-\tau$ bis $0$ immer Konstant bleibt
\begin{equation}
	y(t)=1 \quad\text{wenn}\quad -1\le t<0
\end{equation}
Daraus folgt, dass im Bereich von $0\le t<\tau$ die Ableitung
\begin{equation}\label{abl}
	\dot{y}(t)=k
\end{equation}
wird. Durch integrieren von \eqref{abl} erhalten wir
\begin{equation}\label{schritt1}
	y(t)=1+kt
\end{equation}
für den Bereich $0\le t<\tau$. 
Dieses $y(t)$ kann als Anfangswert für den nächsten Schritt genommen werden.
Wir erhalten für den zweiten Schritt  $\tau\le t<2\tau$ 
\begin{equation}\label{abl2}
	\dot{y}(t)=k(1+k(t-\tau))=k+k^2(t-\tau)
\end{equation}
Es ist offensichtlich das diese Methode nur für kurze Zeiten, einfache Anfangswerte und einfache Formeln funktioniert. 
Bereits \eqref{abl2} ist nicht mehr ganz einfach zu integrieren. 
Für längere Zeiten werden die Integrale immer komplexer.

In Abbildung \ref{fig:bsp} sieht man das Resultat des Beispiels. 
Die lineare Funktion \eqref{schritt1} ist als erster Schritt gut sichtbar.
Ab Schritt 2 sieht man die Nichtlinearität. 
\begin{figure}
	\centering
	% This file was created by matlab2tikz.
%
%The latest updates can be retrieved from
%  http://www.mathworks.com/matlabcentral/fileexchange/22022-matlab2tikz-matlab2tikz
%where you can also make suggestions and rate matlab2tikz.
%
\begin{tikzpicture}

\begin{axis}[%
width=4.521in,
height=3.566in,
at={(0.758in,0.481in)},
scale only axis,
xmin=-1,
xmax=3,
ymin=-1,
ymax=2,
ylabel={$y$},
xlabel={$t$},
axis background/.style={fill=white}
]
\addplot [color=blue, forget plot]
  table[row sep=crcr]{%
0	1\\
0.03000300030003	0.970000000000003\\
0.06000600060006	0.940000000000007\\
0.09000900090009	0.91000000000001\\
0.12001200120012	0.880000000000013\\
0.15001500150015	0.850000000000017\\
0.18001800180018	0.82000000000002\\
0.21002100210021	0.790000000000023\\
0.24002400240024	0.760000000000026\\
0.27002700270027	0.73000000000003\\
0.3000300030003	0.700000000000033\\
0.33003300330033	0.670000000000036\\
0.36003600360036	0.64000000000004\\
0.39003900390039	0.610000000000043\\
0.42004200420042	0.580000000000046\\
0.45004500450045	0.55000000000005\\
0.48004800480048	0.520000000000053\\
0.51005100510051	0.490000000000054\\
0.54005400540054	0.460000000000052\\
0.57005700570057	0.43000000000005\\
0.6000600060006	0.400000000000048\\
0.63006300630063	0.370000000000045\\
0.66006600660066	0.340000000000043\\
0.69006900690069	0.310000000000041\\
0.72007200720072	0.280000000000039\\
0.75007500750075	0.250000000000036\\
0.78007800780078	0.220000000000037\\
0.81008100810081	0.190000000000037\\
0.84008400840084	0.160000000000038\\
0.87008700870087	0.130000000000038\\
0.9000900090009	0.100000000000039\\
0.93009300930093	0.0700000000000395\\
0.96009600960096	0.0400000000000395\\
0.99009900990099	0.0100000000000394\\
1.02010201020102	-0.0198069499999607\\
1.05010501050105	-0.0487674499999608\\
1.08010801080108	-0.076827949999961\\
1.11011101110111	-0.103988449999961\\
1.14011401140114	-0.130248949999962\\
1.17011701170117	-0.155609449999962\\
1.2001200120012	-0.180069949999963\\
1.23012301230123	-0.203630449999964\\
1.26012601260126	-0.226290949999964\\
1.29012901290129	-0.248051449999965\\
1.32013201320132	-0.268911949999966\\
1.35013501350135	-0.288872449999967\\
1.38013801380138	-0.307932949999969\\
1.41014101410141	-0.32609344999997\\
1.44014401440144	-0.343353949999971\\
1.47014701470147	-0.359714449999973\\
1.5001500150015	-0.375174949999974\\
1.53015301530153	-0.389735449999976\\
1.56015601560156	-0.403395949999978\\
1.59015901590159	-0.416156449999979\\
1.62016201620162	-0.42801694999998\\
1.65016501650165	-0.438977449999982\\
1.68016801680168	-0.449037949999983\\
1.71017101710171	-0.458198449999984\\
1.74017401740174	-0.466458949999985\\
1.77017701770177	-0.473819449999987\\
1.8001800180018	-0.480279949999988\\
1.83018301830183	-0.485840449999989\\
1.86018601860186	-0.49050094999999\\
1.89018901890189	-0.494261449999991\\
1.92019201920192	-0.497121949999992\\
1.95019501950195	-0.499082449999993\\
1.98019801980198	-0.500142949999995\\
2.01020102010201	-0.500303583919996\\
2.04020402040204	-0.499574065819996\\
2.07020702070207	-0.497979917719998\\
2.1002100210021	-0.495548139619999\\
2.13021302130213	-0.49230573152\\
2.16021602160216	-0.488279693420001\\
2.19021902190219	-0.483497025320003\\
2.22022202220222	-0.477984727220004\\
2.25022502250225	-0.471769799120005\\
2.28022802280228	-0.464879241020006\\
2.31023102310231	-0.457340052920007\\
2.34023402340234	-0.449179234820008\\
2.37023702370237	-0.440423786720009\\
2.4002400240024	-0.43110070862001\\
2.43024302430243	-0.421237000520011\\
2.46024602460246	-0.410859662420012\\
2.49024902490249	-0.399995694320012\\
2.52025202520252	-0.388672096220013\\
2.55025502550255	-0.376915868120014\\
2.58025802580258	-0.364754010020014\\
2.61026102610261	-0.352213521920015\\
2.64026402640264	-0.339321403820016\\
2.67026702670267	-0.326104655720016\\
2.7002700270027	-0.312590277620017\\
2.73027302730273	-0.298805269520017\\
2.76027602760276	-0.284776631420018\\
2.79027902790279	-0.270531363320018\\
2.82028202820282	-0.256096465220018\\
2.85028502850285	-0.241498937120018\\
2.88028802880288	-0.226765779020019\\
2.91029102910291	-0.211923990920019\\
2.94029402940294	-0.197000572820019\\
2.97029702970297	-0.182022524720019\\
};
\addplot [color=blue, forget plot]
  table[row sep=crcr]{%
-1	1\\
0	1\\
};
\addplot [dashed,color=black, forget plot]
table[row sep=crcr]{%
	0	-1\\
	0	2\\
};
\addplot [dashed,color=black, forget plot]
table[row sep=crcr]{%
	1	-1\\
	1	2\\
};
\addplot [dashed,color=black, forget plot]
table[row sep=crcr]{%
	2	-1\\
	2	2\\
};
\end{axis}
\end{tikzpicture}%
	\caption{Beispiel mit $k=-1$ und $\tau=1$}
	\label{fig:bsp}
\end{figure}

\subsubsection{Charakteristisches Polynom}
Bei gewöhnlichen Differentialgleichungen können Lösungen mit Hilfe des charakteristischen Polynoms gefunden werden. 
Bei DDEs wird das Polynom zu einer Gleichung. 
Wir betrachten wiederum die Gleichung \ref{bsp} und verwenden als Lösungsansatz
\begin{equation}\label{ansatz}
	y(t) = ce^{\lambda t}
\end{equation}
Dieser klassische Ansatz eignet sich (fast) immer, da die Exponentialfunktion beim differenzieren erhalten bleibt. 
\eqref{ansatz} eingesetzt in \eqref{bsp} ergibt
\begin{equation}
	\lambda ce^{\lambda t} = kce^{\lambda (t-\tau )}
\end{equation} 
Diese Gleichung kann durch $ce^{\lambda t}$ gekürzt werden zu
\begin{equation}\label{chareq}
	\lambda  - ke^{-\lambda \tau}= 0
\end{equation} 
Damit können nun verschiedene Werte für die Konstante $k$ berechnet werden, je nachdem wie die Lösung $\lambda$ aussehen soll.
Wir nehmen an, dass $\lambda$ komplex ist und setzen $\lambda = \lambda_r + i\lambda_i$ in die Gleichung \eqref{chareq} ein.
\begin{equation}\label{chareq_kompl}
	\lambda_r + i\lambda_i  - ke^{-(\lambda_r + i\lambda_i) \tau}= 0
\end{equation} 
Diese Gleichung stellen wir um als Gleichungssystem mit Imaginär- und Realteil\footnote{Mithilfe der Eulerschen Formel $e^{iy}=\cos(y)+i\sin(y)$}
\begin{align}
	\lambda_r - ke^{-\lambda_r\tau}\cos(\lambda_i\tau)&=0 \label{realchar} \\
	\lambda_i + ke^{-\lambda_r\tau}\sin(\lambda_i\tau)&=0 \label{imagchar}
\end{align}
Nimmt man den zweiten Term auf die rechte Seite und dividiert die beiden Gleichungen, erhält man
\begin{equation} \label{chareq_kompl_frac}
	-\frac{\lambda_r}{\lambda_i} = \cot(\lambda_i\tau)
\end{equation}
Damit lassen sich nun beliebig viele Kombinationen von $\lambda$ und $k$ generieren.
Als Beispiel wollen wir eine periodische Schwingung mit gleichbleibender Amplitude erreichen.
Daraus ergibt sich ein $\lambda_r = 0$. 
Die Verzögerung ist gegeben als $\tau = 1$.
Eingesetzt in \eqref{chareq_kompl_frac} erhalten wir:
\begin{equation}
	\cot(\lambda_i) = 0 \longrightarrow \lambda_i = (2n-1)\frac{\pi}{2}
\end{equation}
Mit $n\in \mathbb{N}$ können wir z.B. die kleinste Frequenz $n=1$ einsetzen und erhalten $\lambda_i = \frac{\pi}{2}$.
Damit können wir aus \eqref{imagchar} die Konstante $k$ bestimmen
\begin{equation}
	\frac{\pi}{2} + k = 0 \longrightarrow k = -\frac{\pi}{2}
\end{equation}
Zur überprüfung wurde mit diesen Werten die Gleichung numerisch berechnet (vgl. Abbildung \ref{fig:bsp_chareq}).
Man sieht dort gut die gleichbleibende Amplitude und die Periodendauer von $\frac{2\pi}{\frac{\pi}{2}}=4$.

\begin{figure}
	\centering
	% This file was created by matlab2tikz.
%
%The latest updates can be retrieved from
%  http://www.mathworks.com/matlabcentral/fileexchange/22022-matlab2tikz-matlab2tikz
%where you can also make suggestions and rate matlab2tikz.
%
\begin{tikzpicture}

\begin{axis}[%
width=4.521in,
height=3.566in,
at={(0.758in,0.481in)},
scale only axis,
xmin=-1,
xmax=20,
ymin=-2,
ymax=2,
ylabel={$y$},
xlabel={$t$},
axis background/.style={fill=white}
]
\addplot [color=blue, forget plot]
  table[row sep=crcr]{%
0	1\\
0.2000200020002	0.685840734641018\\
0.4000400040004	0.371681469282036\\
0.6000600060006	0.0575222039230558\\
0.8000800080008	-0.256637061435924\\
1.000100010001	-0.570796326794905\\
1.2001200120012	-0.8370781412042\\
1.4001400140014	-1.004673781207\\
1.6001600160016	-1.07357337719891\\
1.8001800180018	-1.04377692917992\\
2.000200020002	-0.91528443715004\\
2.2002200220022	-0.692811831767233\\
2.4002400240024	-0.401719394778529\\
2.6002600260026	-0.0730132788391235\\
2.8002800280028	0.262300239370685\\
3.000300030003	0.573214883170596\\
3.2003200320032	0.829061882795288\\
3.4003400340034	1.00375096899054\\
3.6003600360036	1.08044888928263\\
3.8003800380038	1.05206329884011\\
4.000400040004	0.921242761934929\\
4.2004200420042	0.700358046775638\\
4.4004400440044	0.410894972395743\\
4.6004600460046	0.0811695702016152\\
4.8004800480048	-0.256698001198943\\
5.000500050005	-0.56984365282742\\
5.2005200520052	-0.827718726631608\\
5.4005400540054	-1.00508644212531\\
5.6005600560056	-1.08457724034126\\
5.8005800580058	-1.05837694192613\\
6.000600060006	-0.928955575193748\\
6.2006200620062	-0.708834845900301\\
6.4006400640064	-0.419399587351868\\
6.6006600660066	-0.0888208309815821\\
6.8006800680068	0.250697886833838\\
7.000700070007	0.566058621725787\\
7.2007200720072	0.826489603116091\\
7.4007400740074	1.0065466558522\\
7.6007600760076	1.08860232292087\\
7.8007800780078	1.06457222915751\\
8.000800080008	0.936709891836633\\
8.2008200820082	0.717395204334429\\
8.4008400840084	0.427936675271025\\
8.6008600860086	0.0965014634190242\\
8.8008800880088	-0.244625951676723\\
9.000900090009	-0.562188709016319\\
9.2009200920092	-0.825199586034987\\
9.4009400940094	-1.007964533518\\
9.6009600960096	-1.09259284835939\\
9.8009800980098	-1.0707485030828\\
10.00100010001	-0.944471114793201\\
10.2010201020102	-0.725985834900832\\
10.4010401040104	-0.436519896039718\\
10.6010601060106	-0.104240311052822\\
10.8010801080108	0.238486952445908\\
11.001100110011	0.558249593995494\\
11.2011201120112	0.823846299489424\\
11.4011401140114	1.00933122336974\\
11.6011601160116	1.09654858079859\\
11.8011801180118	1.0769097468055\\
12.001200120012	0.952238916780887\\
12.2012201220122	0.734604074564064\\
12.4012401240124	0.445148892015901\\
12.6012601260126	0.112038616450404\\
12.8012801280128	-0.232280502236563\\
13.001300130013	-0.554241579885249\\
13.2013201320132	-0.822429433166693\\
13.4013401340134	-1.01064587389975\\
13.6013601360136	-1.10046887369065\\
13.8013801380138	-1.08305561027834\\
14.001400140014	-0.960012927274474\\
14.2014201420142	-0.743249530817551\\
14.4014401440144	-0.453823470266721\\
14.6014601460146	-0.119896436679787\\
14.8014801480148	0.226006393370834\\
15.001500150015	0.550164351925213\\
15.2015201520152	0.82094854735172\\
15.4015401540154	1.01190794785818\\
15.6015601560156	1.10435316900647\\
15.8015801580158	1.08918557888102\\
16.001600160016	0.967792713012642\\
16.2016201620162	0.751921888084314\\
16.4016401640164	0.462543469866669\\
16.6016601660166	0.127813784848457\\
16.8016801680168	-0.219664442808302\\
17.001700170017	-0.546017576930293\\
17.2017201720172	-0.819403189643811\\
17.4017401740174	-1.01311691645\\
17.6017601760176	-1.10820091373403\\
17.8017801780178	-1.09529912996655\\
18.001800180018	-0.97557783239324\\
18.2018201820182	-0.760620828561267\\
18.4018401840184	-0.471308728047667\\
18.6018601860186	-0.135790669436177\\
18.8018801880188	0.213254471562456\\
19.001900190019	0.541800922830334\\
19.2019201920192	0.817792907116875\\
19.4019401940194	1.01427224989574\\
19.6019601960196	1.11201155284536\\
19.8019801980198	1.10139573739401\\
};
\addplot [color=blue, forget plot]
  table[row sep=crcr]{%
-1	1\\
0	1\\
};
\end{axis}
\end{tikzpicture}%
	\caption{Beispiel mit $k=-\frac{pi}{2}$ und $\tau=1$}
	\label{fig:bsp_chareq}
\end{figure}
