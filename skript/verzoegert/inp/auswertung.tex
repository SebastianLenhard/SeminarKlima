\section{Auswertung}
\rhead{Auswertung}
\subsection{Simulation El-Niño mit realen Werten}
Der Matlab-Code zur Simulation des El-Niño DDE ist nun vorhanden.
Damit die Simulation sinnvoll ist, wird eine Zeitperiode in der Vergangenheit berechnet.
Ausgewählt zur Simulation wird die Zeitperiode ab September 1995. 
Also werden die Daten von Januar bis September 1995 als History genommen.

Simuliert werden 1 Jahr, 3 Jahre und 10 Jahre.
%todo Grafiken einfügen
Es ist offensichtlich, dass die Genauigkeit einer Vorhersage ziemlich schnell abnimmt.
%todo ev. Analyse der Genauigkeit
%todo ev. Andere Zeitspanne zum Vergleich, wo Vorhersage failt

\subsection{Chaotisches Verhalten}
Wenn der Dämpfungsterm $\varepsilon$ weggelassen wird, tendieren die Lösungen zu chaotischem Verhalten.
Chaotisches Verhalten bedeutet, dass kleinste Änderungen der Konstanten extrem unterschiedliche Lösungen verursachen können. 
Wir simulieren über zehn Jahre und ändern dabei die Konstante $a$ von 0 bis 5.
%todo Grafik
Ein ähnlich chaotisches Verhalten sehen wir auch bei einer Änderung der anderen Konstanten.
Zum Vergleich die identische Simulation mit einem $\varepsilon = 0.1$. 
%todo Grafik
Hier sehen wird die korrekte Auswirkung einer Änderung von $a$.
Die Konstante $a$ bewirkt eine Änderung der Amplitude und der Frequenz.


\subsection{Fazit}
%todo Fazit
%todo Literaturverzeichnis