%
% burgertraining.tex -- Trainingsdaten für Burgers Gleichung
%
% (c) 2018 Prof Dr Andreas Müller, Hochschule Rapperswil
%
\subsection{Trainingsdaten für die Gleichung von Burgers\label{burgers:training}}
Damit die Gleichung von Burgers mit einem Machine-Learning-Ansatz gelöst
werden kann, müssen geeignete Trainingsdaten bereitgestellt werden.


\subsubsection{Lösung mit Charakteristiken}
Glatte Lösungen existieren nur für eine beschränkte Zeit.
Um das Verhalten für glatte Lösungen zu trainineren, generieren
wir Lösungen mit Hilfe der Charakteristiken-Methode.
Wir müssen aber sicherstellen, dass die Lösungen nur so lange verwendet
werden, bis sich Sprungstellen entwickeln.
Wir wählen daher das folgende Vorgehen.
Zunächst erzeugen wählen wir eine zufällige Anfangsfunktion.
Dann bestimmen wir die Zeit, bis zu der die zu dieser Anfangsbedingungen
gehörende Funktion keine Sprungstelle entwickelt.
Für diese Zeit berechnen wir die Lösung mit der Charakteristiken-Methode.

\subsubsection{Sprungstellen}
Um das Verhalten bei Sprungstellen zu trainieren generieren wir
einfache Lösungen mit genau einer Sprungstelle.
Wir verwenden dazu die Hugoniot-Rankine-Bedingung \eqref{burgers:hugonito-rankine}.





