%
% compmod.tex -- Beispiel zum computational mode
%
% (c) 2018 Prof Dr Andreas Müller, Hochschule Rapperswil
%
\documentclass[tikz]{standalone}
\usepackage{times}
\usepackage{txfonts}
\usepackage[utf8]{inputenc} 
\usepackage{graphics}
\usepackage{ifthen}
\usepackage{color}
\usetikzlibrary{arrows,intersections}
\usetikzlibrary{math}
\begin{document} 
\begin{tikzpicture}[>=latex,thick]

\input{bw.tex}

\begin{scope}[yshift=0cm]
\pfada
\draw[->] (-0.1,0)--(12.5,0) coordinate[label=$x$];
\draw[->] (0,-0.5)--(0,3.0) coordinate[label={right:$u$}];
\end{scope}

\begin{scope}[yshift=-3.6cm]
\pfadb
\draw[->] (-0.1,0)--(12.5,0) coordinate[label=$x$];
\draw[->] (0,-0.5)--(0,3.0) coordinate[label={right:$u$}];
\end{scope}

\begin{scope}[yshift=-7.2cm]
\pfadc
\draw[->] (-0.1,0)--(12.5,0) coordinate[label=$x$];
\draw[->] (0,-0.5)--(0,3) coordinate[label={right:$u$}];
\end{scope}

\begin{scope}[yshift=-10.8cm]
\pfadd
\draw[->] (-0.1,0)--(12.5,0) coordinate[label=$x$];
\draw[->] (0,-0.5)--(0,3) coordinate[label={right:$u$}];
\end{scope}

%\begin{scope}[yshift=-16cm]
%\pfade
%\draw[->] (-0.1,0)--(12.5,0) coordinate[label=$x$];
%\draw[->] (0,-0.5)--(0,3) coordinate[label={right:$u$}];
%\end{scope}

\end{tikzpicture}
\end{document}
