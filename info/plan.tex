%
% plan.tex -- Themenplan für die 
%
% (c) 2018 Prof Dr Andreas Müller, Hochschule Rapperswil
%
\documentclass[a4paper,12pt]{article}
\usepackage{times}
\usepackage{amssymb}
\usepackage{amsmath}
\usepackage{german}
\usepackage{fancyhdr}
\usepackage{longtable}
\usepackage{geometry}
\usepackage{txfonts}
\usepackage[utf8]{inputenc}
\usepackage[T1]{fontenc}
\geometry{papersize={210mm,297mm},total={170mm,260mm},top=21mm}
\usepackage{color}
\begin{document}
\pagestyle{fancy}
\lhead{Mathematisches Seminar 2018}
\rhead{Themenplan}
\begin{longtable}{|l|p{14.0cm}|}
\hline
Datum&Thema
\\
\hline
\endhead
\hline
\endfoot
19.~Februar&
Einführung, Ziele, Vorstellung der Themen, Bücher
\\
\hline
26.~Februar&
Wetter und Klima, Modell von Budyko, Snowball Earth,
Bedeutung von Differentialgleichungen für die Klimamodellierung,
Parameterabhängigkeit und Bifurkationen
\\
\hline
\phantom{0}5.~März&
Differentialgleichungen, Gleichgewichtslösungen, Bifurkation.
Dieses Kapitel ist nur eine Übersicht, Details werden im Skript bzw.~im
Seminarbuch {\em Differentialgleichungen} behandelt und müssen bei Bedarf
dort nachgelesen werden.
\\
\hline
12.~März&
Ozeane und Klima, Box-Modelle, Modell von Stommel für die thermohaline
Zirkulation, Koppelungen über grosse Distanzen
\\
\hline
19.~März&
Strömungsdynamik. Herleitung der Grundgleichungen der Strömungsdynamik,
die wesentlich sind für das Verständnis des Klimasystems.
Insbesondere auch der Einfluss der Coriolis-Kraft, welche für das
Verständnis des El Niño-Phänomens wesentlich ist.
\\
\hline
26.~März.&
Aufbau der Atmosphäre, Globale Zirkulation, Energietransport in der
Atmosphäre.
\\
\hline
\phantom{0}9.~April&
Lorenz-System, Herleitung der Gleichung, Motivation aus Zirkulationsmodell,
chaotisches Verhalten der Lösungen des Lorenz-Systems
\\
\hline
16.~April&
Strahlung und Treibhausgase
\\
\hline
23.~April&
El Niño, Koppelung über grosse Distanzen, verzögerte
Differentialgleichungen, Kelvin- und Rossby-Wellen
\\
\hline
30.~April&
Fourier-Theorie, Sonnenfleckenzyklus, Milankovic-Zyklen der Eiszeiten,
langfristige astronomische Einflüsse auf das Klimasystem.
\\
\hline
\phantom{0}7.~Mai&
Datenassimilation mit Hilfe des Kalman-Filters
\end{longtable}

\end{document}
